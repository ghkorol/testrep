\chapter{Reconstruction of the top-pair Kinematics}\label{chapt:kinReco}

The two neutrinos are not detected, thus additional
assumptions are needed to define the full final state kinematics of the $t\bar{t}$ reconstructed in dilepton decay channel.
This section describes the method used for the full kinematics reconstruction of this system. The mathematical
background (sec. \ref{sec:MatBg}) as well as a performance description of the method (sec. \ref{sec:SolSer} -- \ref{sec:kinRecPerf}) are revealed.

\section{Mathematical Background}\label{sec:MatBg}

The presence of two undetected neutrinos introduces six unknowns  (three momentum components of each neutrino)
for the $t\bar{t}$ system in the dilepton final state.
The following constraints are being used:

\begin{itemize}
 \item \textit{t and $\bar{t}$ masses} ($m_{t}$ and $m_ {\bar{t}}$) are assumed to be equal and constrained to the same value of 172.5 GeV\cite{PDG-2012};
 \item The whole missing transverse energy $E_{T}^{miss}$ of the event is assumed to arise entirely
 from the two neutrinos from the $t\bar{t}$ decay;
 \item \textit{The $W^{\pm}$ masses} ($m_{W\pm}$) are assumed to be known. As $W^{\pm}$ are resonances with a very small lifetime, their masses 
 are distributed according to the Breit-Wigner function. For this reconstruction, the $W^{\pm}$ masses are constrained to values randomly taken 
 from the generator Breit-Wigner $W^{\pm}$ mass spectrum.
\end{itemize}

These assumptions lead to a system of six equations which describe the conservation of energies and momenta:

\begin{align}\label{alg:LS1}
 E^{miss}_{T_{x}} & =  p_{\nu_{x}} + p_{\bar{\nu}_{x}} \\
 %
 E^{miss}_{T_{y}} & =  p_{\nu_{y}} + p_{\bar{\nu}_{y}} \\
 %
 m^{2}_{W^{+}} & = (E_{l^{+}} + E_{\nu})^{2} - (p_{l^{+}_{x}} + p_{\nu_{x}})^{2} - (p_{l^{+}_{y}} + p_{\nu_{y}})^{2} - (p_{l^{+}_{z}} + p_{\nu_{z}})^2 \\
 %
 m^{2}_{W^{-}} & = (E_{l^{-}} + E_{\bar{\nu}})^{2} - (p_{l^{-}_{x}} + p_{\bar{\nu}_{x}})^{2} - (p_{l^{-}_{y}} + p_{\bar{\nu}_{y}})^{2} - (p_{l^{-}_{z}} + p_{\bar{\nu}_{z}})^2 \\
 % 
 m_{t}^{2} & = (E_{b} + E_{l^{+}} + E_{\nu})^{2} - (p_{b_{x}} + p_{l^{+}_{x}} + p_{\nu_{x}})^2 \nonumber \\
           & - (p_{b_{y}} + p_{l^{+}_{y}} + p_{\nu_{y}})^2 - (p_{b_{z}} + p_{l^{+}_{z}} + p_{\nu_{z}})^2 \\
 %
 m_{\bar{t}}^{2} & = (E_{\bar{b}} + E_{l^{-}} + E_{\bar{\nu}})^{2} - (p_{\bar{b}_{x}} + p_{l^{-}_{x}} + p_{\bar{\nu}_{x}})^2 \nonumber \\
                 & - (p_{\bar{b}_{y}} + p_{l^{-}_{y}} + p_{\bar{\nu}_{y}})^2 - (p_{\bar{b}_{z}} + p_{l^{-}_{z}} + p_{\bar{\nu}_{z}})^2\label{alg:LS6} 
\end{align}

Here the $E_{l^{\pm}}$ and $p_{l^{\pm}_{x,y,z}}$ correspond to the lepton(antilepton) energy and momentum components respectively; 
$E_{b/\bar{b}}$ and $p_{b/\bar{b}_{x,y,z}}$ are the $b$/$\bar{b}$-jet energy and momentum components respectively; the $E^{miss}_{T_{x,y}}$ are
the two components of the missing transverse energy; the $p_{\nu/\bar{\nu}_{x,y,z}}$ are the neutrino (antineutrino) momenta components.
The neutrino energies $ E_{\nu/\bar{\nu}}$ are composed from momenta:

\begin{equation}
 E_{\nu/\bar{\nu}}^{2} = p_{\nu/\bar{\nu}_{x}}^{2} + p_{\nu/\bar{\nu}_{y}}^{2} + p_{\nu/\bar{\nu}_{z}}^{2}
\end{equation}

The quantities $E_{l^{\pm}}$, $p_{l^{\pm}_{x,y,z}}$, $E_{b/\bar{b}}$, $p_{b/\bar{b}_{x,y,z}}$ and $E^{miss}_{T_{x,y}}$ are reconstructed from the detector (as described in chapter \ref{chapt:event_selection})
and $p_{\nu/\bar{\nu}_{x,y,z}}$ are the unknowns.

An analytical solution of the system of equations (\ref{alg:LS1}--\ref{alg:LS6}) was proposed in \cite{LSpaper}. After a number of transformations
the system is reduced to a fourth order polynomial equation for the neutrino momentum component $p_{\nu_{x}}$:

\begin{equation}\label{eq:eqLSf}
 0 = h_{0} p_{\nu_{x}}^{4} + h_{1} p_{\nu_{x}}^{3} + h_{2} p_{\nu_{x}}^{2} + h_{3} p_{\nu_{x}} + h_{4},
\end{equation}

where the coefficients $h_{0} - h_{4}$ \cite{LSpaper, LSerrat} depend on the missing transverse energy $E_{T}^{miss}$ and four-momenta of the
leptons, antileptons, $b$- and $\bar{b}$-jets. 

There can be up to four solutions for the equation \ref{eq:eqLSf}. This equation (in case if the coefficients $h$ are such that the equation can't be simplified to the third, second
or first order), as it was constructed, can not have analytically three or one real solutions, as proven in \cite{LSpaper}. Thus, it is expected to get either two, or four solutions
of the eq. \ref{eq:eqLSf}. However, due to the limited computing accuracy, there may be cases when two solutions are indistinguishable and treated as one. This phenomenon can create
three out of four or one out of two solutions.

The distribution of the number of solutions for the generated \MG + \PYTHIA $t\bar{t}$ signal sample 
is shown in Figure \ref{fig:LSNsol}. Two (four) solutions per event are expected in approx. 80$\%$ (20\%) of the cases.
The cases, when there are one or three solutions found, have a rate of about 0.1\%.

\begin{figure}[t]
  \centering
  \includegraphics[width=0.8\textwidth]{05_kinReco/plots/KinReco_Nsolutions.png}
  \caption{Number of solutions of the equation \ref{eq:eqLSf}. The distribution is normalized to unity. The information used for this plot is taken
  from the generated \MG + \PYTHIA $t\bar{t}$ signal for this analysis.}
  \label{fig:LSNsol}
\end{figure}


% Under the real conditions a sizable amount of the events will find no solutions of the kinematic equations (\ref{alg:LS1}--\ref{alg:LS6})
% because of an imperfect reconstruction of the detector objects. But in the other cases there is a possibility to find up to four solutions.

\section{Ambiguity and Detector Effects Treatment}\label{sec:SolSer}

There are several problems arising during the $t\bar{t}$ dilepton final state kinematics reconstruction:

\begin{itemize}
 \item \textit{Fluctuations of measurement}. There might be no solutions of the equation \ref{eq:eqLSf} found for a combination
 of leptons, jets and missing transverse energy arising from the $t\bar{t}$ system due to reconstruction effects, e.g. detector resolution,
 jet overlapping, badly reconstructed missing transverse energy, etc.
 %
 \item \textit{Multiple solutions of the kinematic equations}. As discussed in section \ref{sec:MatBg} the equation \ref{eq:eqLSf}
 has up to four mathematical solutions while only one of them corresponds to the correct kinematics of the neutrino.
 %
 \item \textit{Multiple combinations of leptons and jets}. An event with a $t\bar{t}$ decaying to a dilepton final state has at least two leptons and two
 $b$-jets. However there is a priori no information if a $b$-jet originates from the $t$ or the $\bar{t}$. For this reason each $b$-jet candidate is being 
 paired to one of the leptons, and then to another
 to form a $t$ or $\bar{t}$ candidate. Thus an event with two leptons and two $b$-jets has two possible $t\bar{t}$ candidates. In case of further jets
 in the event, the number of $t\bar{t}$ candidates can be up to $N_{jet}!$, where $N_{jet}$ is the jet multiplicity.
\end{itemize}

% The last two challenges have different nature but boths cause the same issue of multiple possible $t\bar{t}$ candidates, thus they can be treated together.

\subsection{Fluctuations of measurements}\label{ssec:smear}

The problem of rescuing the events which are lost due to fluctuations is solved by varying the measured objects energies and
momentum directions. This increases the efficiency of finding a solution of the system of equations (\ref{alg:LS1}--\ref{alg:LS6}) (see Appendix \ref{appendix:smear}). 
The idea was implemented by reconstructing each event 100 times, each time varying relevant observables according to their resolution determined from the Monte Carlo simulation.
The energies and directions of leptons and jets are smeared. All variations are done randomly, independently and simultaneously for each quantity.

The energy variation was performed through multiplication of the reconstructed energy value
by a correction factor determined from the distribution of $f = \frac{E_{true}}{E_{reco}}$. Here $E_{reco}$ is the reconstructed lepton or jet energy taken from the MC signal simulation
and $E_{true}$ is the true energy of the same object on the particle level. The distributions for $f$ which are used for the random choice
of the correction factors are shown in Figure \ref{fig:fE}. The correction factors are determined from the signal MC simulation using jets and 
leptons matched to the particle level $b$-quarks and leptons arising from the decay chain $t \rightarrow WB \rightarrow l\nu b$. 

If, due to the energy smearing, the lepton or jet $p_{T}$ is reduced to the value beyond the selection criteria (see sec. \ref{sec:sel}), they are still accepted.

\begin{figure}[t]
\centering
\begin{subfigure}
  \centering
  \includegraphics[width=0.48\textwidth]{05_kinReco/plots/fE_jet.png}
\end{subfigure}
\begin{subfigure}
  \centering
  \includegraphics[width=0.48\textwidth]{05_kinReco/plots/fE_lep.png}
\end{subfigure}
\caption{Distributions of the energy correction factors used for the energy smearing in the kinematic
reconstruction of the top-quark kinematics. The distribution of the factor for jets is shown on the left and for
leptons on the right.}
\label{fig:fE}
\end{figure}

The directional smearing is applied by rotating the actual momentum vectors relatively to the nominal vector direction
as shown in Figure \ref{fig:angleRot}. 
\begin{figure}[h]
 \centering
 \includegraphics[width=0.2\textwidth]{05_kinReco/plots/angle_rot.pdf}
 \caption{Sketch of the directional smearing of leptons and jets as applied in the $t\bar{t}$ kinematic reconstruction.}
 \label{fig:angleRot}
\end{figure}

A smearing of the polar angle $\alpha$ is performed by choosing a random value from
the MC distributions presented in Figure \ref{fig:dAngle}. The azimuthal angle $\omega$ is taken randomly from 0 to $2\pi$.

\begin{figure}[t]
\centering
\begin{subfigure}
  \centering
  \includegraphics[width=0.48\textwidth]{05_kinReco/plots/dan_jet.png}
\end{subfigure}
\begin{subfigure}
  \centering
  \includegraphics[width=0.48\textwidth]{05_kinReco/plots/dan_lep.png}
\end{subfigure}
\caption{Distributions of the angle between the particle level direction and the detector level direction.
The angle distribution for the $b$-quarks is shown on the left and for the leptons on the right.}
\label{fig:dAngle}
\end{figure}

In each of the 100 variations of the jet and lepton kinematics, the transverse missing energy $E_{T}^{miss}$ has to be recalculated. This is done
assuming the transverse energy component, which does not refer to the leptons and jets forming a $t\bar{t}$ candidate, to be constant. Thus the
missing transverse energy for the $i^{th}$ smearing will be expressed as following:

\begin{equation}
 E^{miss\;i}_{T_{x,y}} = E^{miss \; from \, reconstruction}_{T_{x,y}} + p^{jets \; not\,smeared}_{x,y} + p^{leptons\;not\,smeared}_{x,y} - (p^{jets\;i}_{x,y} + p^{leptons\;i}_{x,y}).
\end{equation}

Here the $E^{miss \; from \, reconstruction}_{T_{x,y}}$, $p^{jets \; not\,smeared}_{x,y}$, $p^{leptons\;not\,smeared}_{x,y}$ are the missing transverse energy and momenta
components taken directly from the detector reconstruction (see chapt. \ref{chapt:event_selection} and \ref{chapt:event_sel}) without applying any variations; the $p^{jets\;i}_{x,y}$ and $p^{leptons\;i}_{x,y}$
are the smeared jets and leptons momenta respectively on the $i^{th}$ of the 100 variation step.

\subsection{Single solution choice}

The equation \ref{eq:eqLSf} is solved for every of the 100 event reconstructions. Each equation may have up to four solutions, thus each event
has up to ($100 \otimes 4 \otimes N_{jet}!$) reconstructed $t\bar{t}$ candidates ($N_{jet}!$ component has already been discussed is sec. \ref{sec:SolSer}). 
To obtain one $t\bar{t}$ pair out of this candidate variety, the following steps are undertaken:

\begin{itemize}

 \item [--] First, the single combination of leptons and jets is selected (getting rid of the combinatorics part $N_{jet}!$). The priority goes to
 the lepton-jet combination where there are two $b$-tagged jets (see sec. \ref{ssec:bTag}). The further indeterminacy connected with the single lepton-jet
 combination is resolved by requiring the combination with the highest weight $\omega = \omega_{m^{\bar{l}b}} \cdot \omega_{m^{l\bar{b}}}$. Here 
 $m^{\bar{l}b}$ and $m^{l\bar{b}}$ are the reconstructed invariant masses of the lepton-jet pairs from the top and the anti-top 
 decays, respectively. The distribution of the $m(l\bar{b}) \& m(\bar{l}b)$ is shown in Fig. \ref{fig:mlb}. 
 These weights are calculated from the spectrum of the lepton-jet pairs in top decays obtained from the signal MC on the 
 particle level after all kinematic cuts described in chapter \ref{chapt:event_sel}. This reduces the number of the possible candidates to maximum
 $100 \otimes 4$.
 
 \begin{figure}[t]
  \centering
  \includegraphics[width=0.8\textwidth]{05_kinReco/plots/mlb_distr.png}
  \caption{The distribution of the invariant mass of the lepton -- $b$-jet system originating from one $t/\bar{t}$-quark. This distribution is obtained on the
  generator level.}
  \label{fig:mlb}
 \end{figure}
 
 \item [--] For each of the four solutions of equation \ref{eq:eqLSf} the invariant mass of the $t\bar{t}$ pair, $m(t\bar{t})$, is calculated. Only
 the solution with the smallest $m(t\bar{t})$ is taken. The detailed study of the smallest $m(t\bar{t})$ criterion is presented in Appendix \ref{appendix:mtt}. 
%  This criterion was prior introduced in \cite{PhysRevD.73.112006}, where it was shown to
%  deliver for the correct lepton-jet assignment in most cases the correct solution.
 %
 \item [--] For the remaining up to 100 candidates (these number is related only to the energy and directional smearing, see sec. \ref{ssec:smear}), the $t$ ($\bar{t}$) 
 momentum is constructed as a weighted average of all smeared solutions as following:
 \begin{equation}
  \langle{\vec{p}(t,\bar{t})}\rangle = \frac{\sum\limits_{i=1}^{100} \omega_{i} \vec{p}(t,\bar{t})_i}{\sum\limits_{i=1}^{100} \omega_i}.
 \end{equation}
 Here $\omega_i$ is a weight and $\vec{p}(t, \bar{t})_{i}$ is a $t$ or $\bar{t}$ momentum three vector for the $i^{th}$ variation in the event. 
 In case that for the $i^{th}$ variation no solution of the kinematic equations is found, both weight and momentum three vector are set to zero. To complete the kinematics,
 the $t$ and $\bar{t}$ energies are calculated taking $\langle{\vec{p}(t,\bar{t})}\rangle$ and assuming the top mass $m(t) = m(\bar{t}) = $ 172.5 GeV.
\end{itemize}

\section{Performance}\label{sec:kinRecPerf}

Only the events in which solutions of kinematic equations are found are accepted for this analysis. 
Figure \ref{fig:EffSF} shows the efficiencies and scale factors for the $t\bar{t}$ kinematic reconstruction procedure depending on jet multiplicity in the event, 
lepton and $b$-jets kinematics and missing transverse energy. The integrated efficiency of the kinematic reconstruction method is 93 $\%$. The remaining 7\%
inefficiency possibly originates from the events, where the kinematic reconstruction failed due to the wrong jet choice or due to the detector resolution.

Overall, the scale factors 
show a flat behavior depending on different variables, thus the value of 0.9909 for the scale factor is used for the analysis. This value was determined 
out of the histogram with only one bin for the whole kinematic range.

\begin{figure}[t]
\centering
\begin{subfigure}
  \centering
  \includegraphics[width=0.49\textwidth]{05_kinReco/plots/eff_SF/KinRecoEff_JetMult.png}
\end{subfigure}
\begin{subfigure}
  \centering
  \includegraphics[width=0.49\textwidth]{05_kinReco/plots/eff_SF/KinRecoEff_MET.png}
\end{subfigure}
\begin{subfigure}
  \centering
  \includegraphics[width=0.49\textwidth]{05_kinReco/plots/eff_SF/KinRecoEff_LeptonEta.png}
\end{subfigure}
\begin{subfigure}
  \centering
  \includegraphics[width=0.49\textwidth]{05_kinReco/plots/eff_SF/KinRecoEff_LeptonpT.png}
\end{subfigure}
\begin{subfigure}
  \centering
  \includegraphics[width=0.49\textwidth]{05_kinReco/plots/eff_SF/KinRecoEff_JetEta.png}
\end{subfigure}
\begin{subfigure}
  \centering
  \includegraphics[width=0.49\textwidth]{05_kinReco/plots/eff_SF/KinRecoEff_JetpT.png}
\end{subfigure}
\caption{The efficiencies and scale factors for the $t\bar{t}$ kinematics reconstruction procedure in bins of the jet multiplicity in the event (top left),
         missing transverse energy (top right), lepton pseudorapidity (middle left) and transverse momentum (middle right) and $b$-jet pseudorapidity (bottom left) 
         and transverse momentum (bottom right). The efficiencies in the data 
         are marked with black dots, efficiencies in MC simulations are red lined and the scale factors are plotted with the blue dots.}
\label{fig:EffSF}
\end{figure}

The scatter plots on Figure \ref{fig:ScatterPl} show the correlation between the kinematic variables after the selection and kinematic reconstruction in the simulated
signal and the actual characteristics of the $t\bar{t}$ system and decay on the particle level. There are no shifts and trends observed, thus showing 
the trustful behavior of the kinematic reconstruction algorithm.

\begin{figure}[t]
\centering
\begin{subfigure}
  \centering
  \includegraphics[width=0.45\textwidth]{05_kinReco/plots/scatter/pt-t.png}
\end{subfigure}
\begin{subfigure}
  \centering
  \includegraphics[width=0.45\textwidth]{05_kinReco/plots/scatter/pTtt-scaterPlot.png}
\end{subfigure}
\begin{subfigure}
  \centering
  \includegraphics[width=0.45\textwidth]{05_kinReco/plots/scatter/y-t.png}
\end{subfigure}
\begin{subfigure}
  \centering
  \includegraphics[width=0.45\textwidth]{05_kinReco/plots/scatter/y-tt.png}
\end{subfigure}
\begin{subfigure}
  \centering
  \includegraphics[width=0.45\textwidth]{05_kinReco/plots/scatter/mtt-scaterPlot.png}
\end{subfigure}
\caption{Plots which show the reconstructed vs generated variables: the transverse momentum of the $t$ and $t\bar{t}$ (top left and right respectively), 
        the rapidity of the $t$ and $t\bar{t}$ (middle left and right respectively) and the invariant mass of the $t\bar{t}$ system, obtained from the 
        signal MC.}
\label{fig:ScatterPl}
\end{figure}

\section{Control Distributions}

The kinematic reconstruction method shows desirable results over the complete kinematic range of the top system. The figure \ref{fig:CPkinTop} shows
the control distributions of the kinematic quantities of the top quark system. The distribution of the transverse momentum of the top quark
shows a good agreement between experimental data and simulation in the small and middle $p_{T}$. However the higher $p_{T}$ range is not well
described by the MC. A slight slope is observed in the data-to-MC ratio plot. The rapidity is well described by the simulation in the central region, 
whereas for the edges MC slightly underestimates the data.
An overall good agreement is observed in the  $x_{1}$\footnote{ $x_{1}$ is a fraction of a parton momentum transfered to a $t$ quark} 
distribution.

\begin{figure}[h]
\centering
\begin{subfigure}
  \centering
  \includegraphics[width=0.49\textwidth]{05_kinReco/plots/CP_top_pt.png}
\end{subfigure}
\begin{subfigure}
  \centering
  \includegraphics[width=0.49\textwidth]{05_kinReco/plots/CP_top_rapidity.png}
\end{subfigure}
\begin{subfigure}
  \centering
  \includegraphics[width=0.49\textwidth]{05_kinReco/plots/CP_x1.png}
\end{subfigure}
\caption{Control distributions of the top quark $p_{T}$ (top left), rapidity (top right) and  $x_{1}$ in the events 
 after the kinematic reconstruction. The experimental data with the error bars corresponding to the statistical uncertainties
 and simulated distributions of signal and different background are plotted. The bottom subplots represent the data-to-MC yield ratio distributions. 
 The distributions are presented for the $t$ quark only.}
\label{fig:CPkinTop}
\end{figure}

In figure \ref{fig:CPkinTTbar} the control distributions of the kinematic variables of the $t\bar{t}$ system are presented. The transverse momentum is overall
well described by the simulation. A slight data excess is observed in central rapidity $y^{t\bar{t}}$ and the edges have also slightly worse agreement between
experimental data and MC. The invariant mass of the $t\bar{t}$ system is well described by the simulation for the lower mass values. A worse agreement is
observed for the higher mass values starting from 1400 GeV.

\begin{figure}[h]
\centering
\begin{subfigure}
  \centering
  \includegraphics[width=0.49\textwidth]{05_kinReco/plots/CP_ttbar_pt.png}
\end{subfigure}
\begin{subfigure}
  \centering
  \includegraphics[width=0.49\textwidth]{05_kinReco/plots/CP_ttbar_rapidity.png}
\end{subfigure}
\begin{subfigure}
  \centering
  \includegraphics[width=0.49\textwidth]{05_kinReco/plots/CP_ttbar_mass.png}
\end{subfigure}
\caption{Control distributions of the $p_{T}$ (top left), rapidity (top right) and invariant mass of the $t\bar{t}$ system in the events 
 after the kinematic reconstruction. The experimental data with the error bars corresponding to the statistical uncertainties
 and simulated distributions of signal and different backgrounds are plotted. The bottom subplots represent the data-to-MC yield ratio distributions.}
\label{fig:CPkinTTbar}
\end{figure}

\section{Usage of the Kinematic Reconstruction in TOP-12-028}

The kinematic reconstruction as described in this chapter has also been implemented into the analysis, which measured the single differential $t\bar{t}$
production cross sections in the dilepton channel at $\sqrt{s} = $8 TeV, TOP-12-028 \cite{TWikiTOP12028} \cite{Khachatryan:2015oqa}. 
It was an alternative to the kinematic reconstruction utilized in the measurement of the single differential $t\bar{t}$
production cross sections in the dilepton channel at $\sqrt{s} = $7 TeV \cite{Chatrchyan:2012saa}, where no smearing of the leptons and 
jets energies and directions were performed, and on the other hand the top-quark mass was scanned.

The results obtained in TOP-12-028 exploiting the kinematic reconstruction are presented in fig. \ref{fig:xsec_top12028}.

\begin{figure}[h]
\centering
\begin{subfigure}
  \centering
  \includegraphics[width=0.49\textwidth]{05_kinReco/plots/DiffXS_HypToppT.png}
\end{subfigure}
\begin{subfigure}
  \centering
  \includegraphics[width=0.49\textwidth]{05_kinReco/plots/DiffXS_HypTTBarpT.png}
\end{subfigure}
\caption{Normalised differential $t\bar{t}$ production cross section as a function of the $p_{T}$ of the top quarks or antiquarks and $t\bar{t}$-pairs. 
         The superscript 't' refers to both top quarks and antiquarks. The inner (outer) error bars indicate the statistical (combined statistical and 
         systematic) uncertainty. The measurement is compared to predictions from \MG, \Powheg and \MCNLO Monte Carlo generators. The \MG prediction is 
         shown both as a curve and as a binned histogram. The figures are taken from \cite{TWikiTOP12028} \cite{Khachatryan:2015oqa}.}
\label{fig:xsec_top12028}
\end{figure}