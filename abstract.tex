\thispagestyle{empty}
\vspace{-3cm}
\section*{\centering Abstract}

\vspace{\baselineskip}

The high energy scale of the $pp$ collisions at the Large Hadron Collider (LHC) at CERN makes this facility
to a real factory for the production of $t\bar{t}$ pairs. This enables to study the top-quark 
properties and its production and decay mechanisms in unprecedent detail. The dileptonic decay channel of the
top-quark pair, in which both $W$ bosons, produced from the top-quark decay, decay into a lepton and neutrino,
is studied in this analysis. The limitation to one electron and one muon in final state 
used in this work allows to strongly suppress the possible background processes and leads to a higher
signal purity.
About 40k events with a 
top-quark pair have been selected using the $\sqrt{s}$ = 8 TeV data recorded with the CMS detector 
in the year 2012. Exploiting this large sample, double differential top-quark pair production 
cross sections are measured for the first time. The cross sections are studied as functions of
various observables which describe the top and top-pair kinematics. 

To obtain the full kinematics of the $t\bar{t}$ final state, which contains two undetected neutrinos,
a kinematic reconstruction procedure was developed and exploited in this work. The new procedure makes
use of all available constraints and is based on a repeated reconstruction of each event with detector 
observables smeared according to their resolutions in order to obtain for each event solutions for the
kinematic constraint equations. In order to obtain
double differential cross sections, the distributions of reconstructed observables
are then corrected for detector effects by using a double differential unfolding procedure, which is
based on a $\chi^{2}$ minimization.

The double differential cross sections presented in this work allow to test the Standard Model 
in detail and investigate previously seen disagreements between measured and predicted single differential cross sections.
The results of this work are compared to four different Standard Model predictions (up to next-to-leading order
of the hard interaction description), implemented in various Monte Carlo event generators.

In general, the results agree with the predictions, except for some cases, which are discussed
in detail.

%\newpage
%\thispagestyle{empty}
%\mbox{}
\clearpage