\thispagestyle{empty}
\vspace{-3cm}
\section*{\centering Abstract}

\vspace{\baselineskip}

The energy scale of the $pp$ collisions at the Large Hadron Collider (LHC) allows an intense
production of $t\bar{t}$ pairs. This enables for the first time to accurately measure the top-quark 
properties. About 40k events with a top-quark pair have been selected with high purity in the 
$e^{\mp}\mu^{\pm}$ decay channel using the $\sqrt{s}$ = 8 TeV data recorded with the CMS detector 
in the year 2012. Exploiting this large sample, double differential top-quark pair production 
cross sections are measured for the first time. The cross sections are studied as functions of
various observables, which describe top and top-pair kinematics. 

To obtain the full kinematics of the $t\bar{t}$ final state, which contains two undetected neutrinos,
a kinematic reconstruction procedure was developed and exploited in this work. The reconstructed distributions
are corrected for detector effects by using the double differential unfolding procedure, which is
based on $\chi^{2}$ minimization.

The double differential cross sections presented in this work allow to test Standard Model 
in detail and investigate the disagreements between measured and predicted single differential cross sections.
The results of this work are compared to four different Standard Model predictions (up to next-to-leading order
of the hard interaction description), implemented in various Monte Carlo event generators.

In general, the results agree with the predictions, except for some cases, which are discussed
in detail.
% A measurement of charm and beauty production in Deep Inelastic Scattering at HERA is presented.
% The analysis is based on the data sample collected by the ZEUS detector in the period from 2003 to 2007 corresponding to an integrated luminosity of \SI{354}{\per\pico\barn}.
% The kinematic region of the measurement is given by $\num{5}<\Qsq<\SI{1000}{\GeV\squared}$ and $0.02<y<0.7$,
% where $Q^2$ is the photon virtuality and $y$ is the inelasticity.
% A lifetime technique is used to tag the production of charm and beauty quarks.
% Secondary vertices due to decays of charm and beauty hadrons are reconstructed, in association with jets.
% The jet kinematics is defined by $\ETjet>\SI{4.2(5)}{\GeV}$ for charm (beauty) and $-1.6<\etajet<2.2$ for both charm and beauty,
% where \ETjet and \etajet are the transverse energy and pseudorapidity of the jet, respectively.
% The significance of the decay length and the invariant mass of charged tracks associated with the secondary vertex are used as discriminating variables to distinguish between signal and background.
% Differential cross sections of jet production in charm and beauty events as a function of \Qsq, $y$, \ETjet and \etajet are measured.
% Results are compared to Next-to-Leading Order (NLO) predictions from Quantum Chromodynamics (QCD) in the fixed flavour number scheme.
% Good agreement between  data and theory is observed.
% Contributions of the charm and beauty production to the inclusive proton structure function, \Ftwoc and \Ftwob,
% are determined by extrapolating the double differential cross sections using NLO QCD predictions.
% 
% \noindent
% Contributions to the test beam program for the Insertable B-Layer upgrade project of the ATLAS pixel detector are discussed. 
% The test beam data analysis software package EUTelescope was extended, which allowed an efficient analysis of ATLAS pixel sensors.
% The USBPix DAQ system was integrated into the EUDET telescope allowing test beam measurements with the front end chip FE-I4.
% Planar and 3D ATLAS pixel sensors were studied at the first IBL test beam at the CERN SPS.

%\newpage
%\thispagestyle{empty}
%\mbox{}
\clearpage