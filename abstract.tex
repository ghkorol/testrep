\thispagestyle{empty}
\vspace{-3cm}
\section*{\centering Abstract}

\vspace{\baselineskip}

The high energy scale of the $pp$ collisions at the Large Hadron Collider (LHC) at CERN makes this facility
to a factory for the production of $t\bar{t}$ pairs. This enables to study the top-quark 
properties and its production and decay mechanisms in unprecedent detail. About 40k events with a 
top-quark pair have been selected with high purity in the 
$e^{\mp}\mu^{\pm}$ decay channel using the $\sqrt{s}$ = 8 TeV data recorded with the CMS detector 
in the year 2012. Exploiting this large sample, double differential top-quark pair production 
cross sections are measured for the first time. The cross sections are studied as functions of
various observables which describe the top and top-pair kinematics. 

To obtain the full kinematics of the $t\bar{t}$ final state, which contains two undetected neutrinos,
a kinematic reconstruction procedure was developed and exploited in this work. The reconstructed distributions
are corrected for detector effects by using a double differential unfolding procedure, which is
based on a $\chi^{2}$ minimization.

The double differential cross sections presented in this work allow to test the Standard Model 
in detail and investigate previously seen disagreements between measured and predicted single differential cross sections.
The results of this work are compared to four different Standard Model predictions (up to next-to-leading order
of the hard interaction description), implemented in various Monte Carlo event generators.

In general, the results agree with the predictions, except for some cases, which are discussed
in detail.

%\newpage
%\thispagestyle{empty}
%\mbox{}
\clearpage