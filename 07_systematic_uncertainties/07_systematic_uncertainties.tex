\chapter{Systematic Uncertainties}

Due to the limited knowledge of the detector features and theory predictions some assumptions and corrections are
made to obtain the results, which may lead to systematic deviations of the analysis outcome.
Those are the sources of systematic uncertainties.

To determine the systematic uncertainty the analysis is repeated with changed assumption or correction and compared
to the nominal result. The systematic variations are applied to the simulated data.

As the measurement of the normalized double differential production cross section $\frac{1}{\sigma} \frac{d\sigma}{dx\:dy}$
is performed, where $x$ and $y$ are the corresponding kinematic observables, many systematic uncertainties cancel
out. Only those effects which may influence the distributions shape are relevant to be taken to account.

The systematic uncertainties may be divided into two classes:
\begin{itemize}
 \item Experimental uncertainties: variations of the correction factors connected with some reconstruction procedure.
 \item Model uncertainties: variations of the assumption entering the simulation model.
\end{itemize}

This chapter gives a detailed overview of the systematic variations performed in the analysis. In the end the way
of determining the total systematic uncertainty is presented.

%%%%%%%%%%%%%%%%%%%%%%%%%%%%%%%%%%%%%%%%%%
%%%%%%%%%%%%%%%%%%%%%%%%%%%%%%%%%%%%%%%%%%
%%%%%%%%%%%%%%%%%%%%%%%%%%%%%%%%%%%%%%%%%%
\section{Experimental Uncertainties}

The experimental systematic uncertainties are determined by varying the correction and scale factors applied during the
analysis procedure within their uncertainty values.

\subsection{Trigger Efficiency}

The trigger scale factor is described in sec. \ref{sec:sel}. The typical precision of their determination is of the order
of 1\%. The variation of the  is performed within this uncertainty.

\subsection{Pile-up Correction}

The pile-up influences the reconstruction of all the objects. The removal of the pile-up is thus an important procedure
which has to be taken to account in the systematic uncertainties. Following the recommendations of \cite{TWikiSystPU} the
pile-up distribution was varied by $\pm$5\%.

\subsection{Luminosity}

The systematic uncertainty due to the luminosity value is canceled out in the normalized double differential cross section,
while it is relative for the total cross section determination. The luminosity is varied within it's uncertainty of 2.6\%. 
The luminosity uncertainty was determined in \cite{CMS-PAS-LUM-13-001}.

\subsection{Uncertainty on the Lepton Selection}

The lepton scale factors described in sec. \ref{sec:sel} have the uncertainty of 0.3\%. An uncertainty of 1\% is added to account for
the differences between Drell-Yan and $t\bar{t}$ event topologies \cite{AN-2012-389}.

\subsection{Jet Energy Scale}

The jet energy scale correction $p_{T}$ and $\eta$ dependent uncertainties are taken from \cite{CMS-PAS-JME-10-010}, which are of the
order of few percent. The simulated jet energy was scaled up and down by the values of these uncertainties. The missing energy in the 
event is recalculated correspondingly to the change of the jet energies values.

\subsection{Jet Energy Resolution}

The jet energy resolution has been rescaled by the $\eta$ dependent factors recommended by \cite{TWikiSystJER}.

\subsection{$b$-tagging Efficiency Uncertainty}

The uncertainties due to the $b$-tagging algorithm is taken into account by the $b$-tag scale factors variations within their
uncertainties \cite{CMS-PAS-BTV-13-001}. The variations are performed antagonistically depending on the $p_{T}$ and $|\eta|$ of the jets.
If the $p_{T}$ of the jet was greater than a median value (65$\;$GeV for $b$- and $c$-jets and 45$\;$GeV for light-jets) then 
the $b$-tagging scale factor was varied down by it's uncertainty and vice versa. The same strategy is performed also versus $|\eta|$
of the jets. The median value for the jet $|\eta|$ is 0.75 for all the jets flavours.

The variations of the $b$-tagging scale factors depending on the jet kinematics are considered as fully uncorrelated and are added
quadratically to the systematic uncertainty. The variations depending on the jet flavours $c$ and $b$ are considered to be 
fully correlated and the largest of them is taken, while the variation depending on heavy or light jet flavour are treated as uncorrelated.

\subsection{Missing Transverse Energy Uncertainty}

No specific MET variation for the systematic uncertainty was performed. The reason for that is the recalculation of the missing energy
after each variation connected to the jets or leptons kinematics. This matches with the recommendations of the experts \cite{CMS-PAS-JME-12-002}.

\subsection{Uncertainty Related to the $t\bar{t}$ Kinematic Reconstruction}

The scale factors related to the $t\bar{t}$ kinematic reconstruction were presennted in sec. \ref{sec:kinRecPerf}. They were varied 
within their errors to estimate the related systematic uncertainty.

%%%%%%%%%%%%%%%%%%%%%%%%%%%%%%%%%%%%%%%%%%
%%%%%%%%%%%%%%%%%%%%%%%%%%%%%%%%%%%%%%%%%%
%%%%%%%%%%%%%%%%%%%%%%%%%%%%%%%%%%%%%%%%%%
\section{Determination of the Total Systematic Uncertainties}

The total systematic uncertainty consists of the different experimental and model variations described above.
Some of them give the results with the lower values compared to the nominal, and some of them are higher.
The deviations to lower and to higher cross section values are treated independently as follows:

\begin{align}
 \delta_{syst.\,total,\:ij}^{pos} = \sum_{s}\sqrt{\delta_{s,\;ij}^{high^{2}}}, \\
 \delta_{syst.\,total,\:ij}^{neg} = \sum_{s} \sqrt{\delta_{s,\;ij}^{low^{2}}}.
\end{align}

Here the $\delta_{syst.\,total,\:ij}$ is a total systematic uncertainty in the bin $ij$ which consists of the
$s$ different sources. The $\delta_{s,\;ij}^{high/low^{2}}$ are expressed the following way:

\begin{equation}
 \delta_{s,\;ij}^{m} = \textrm{Nominal result}_{ij} - \textrm{Varied Result}_{s,\;ij}, 
\end{equation}
where $m = high$ if $\delta_{s,\;ij}^{m} \geq 0$ and $m = low$ if $\delta_{s,\;ij}^{m} < 0$ . 

In case all the variations of a certain correction factor or assumption result only in higher (or lower) values 
compared to the nominal result, the largest deviation is taken for the uncertainty.

The final result is presented the following way:

\begin{equation}
 \textrm{Result}_{ij}\: = \: \textrm{Nominal Value}_{ij}\: \pm \: \textrm{Stats. uncertainty}_{ij}\:\: { }^{+\:\delta_{syst.\,total,\:ij}^{pos}}_{-\:\delta_{syst.\,total,\:ij}^{neg}}.
\end{equation}

% \section{Model Uncertainties}