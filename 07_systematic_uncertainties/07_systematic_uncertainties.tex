\chapter{Systematic Uncertainties}\label{chapt:syst}

Due to the limited knowledge of detector features and theory predictions some assumptions and corrections are
made to obtain the results, which may lead to systematic deviations of the analysis outcome.
These are the sources of systematic uncertainties.

To determine the systematic uncertainties the analysis is repeated with changed assumptions or corrections and compared
to the nominal result. The systematic variations are applied on the MC only.

The systematic uncertainties may be divided into two classes:
\begin{itemize}
 \item Experimental uncertainties: variations of the correction factors connected with some reconstruction procedure.
 \item Model uncertainties: variations of the assumption entering the simulation model.
\end{itemize}

This chapter gives a detailed overview of the systematic variations performed in the analysis which closely follow the
procedure applied in the measurement of single differential $t\bar{t}$ production cross sections \cite{Asin2014Auth}. 
The way of determining the total systematic uncertainty is presented. The systematic uncertainties, which are taken into
account in this analysis, cover all the sources of the systematics for the normalized differential cross sections 
measurements. 

For every variation discussed in the following the full cross section determination procedure including background subtraction 
and unfolding (using the regularization strength $\tau$ obtained in the nominal measurement) was repeated.
%However, this list may be incomplete for the unnormalized cross sections measurements.

%%%%%%%%%%%%%%%%%%%%%%%%%%%%%%%%%%%%%%%%%%
%%%%%%%%%%%%%%%%%%%%%%%%%%%%%%%%%%%%%%%%%%
%%%%%%%%%%%%%%%%%%%%%%%%%%%%%%%%%%%%%%%%%%
\section{Experimental Uncertainties}

The experimental systematic uncertainties are determined by varying the correction and scale factors applied during the
analysis procedure within their uncertainty values.

\subsection{Trigger Efficiency}

The trigger scale factor is described in sec. \ref{sec:sel}. The typical precision of its determination is of the order
of 1\%. The variation of the factor is performed within this uncertainty.

\subsection{Pileup Correction}

% The pile-up influences the reconstruction of the detector objects. The removal of the pile-up is thus an important procedure
% which has to be taken to account in the systematic uncertainties.
For the pileup correction (described in sec. \ref{sec:sel}, vertex requirement) the total $pp$ inelastic cross section was varied by $\pm$5\% \cite{TWikiSystPU}.
% This conservative variation accounts for the uncertainty of the cross section itself and the luminosity uncertainties.

\subsection{Luminosity}

The measurement of the luminosity delivered to the CMS experiment in 2012 was done utilizing the pixel cluster counting algorithm \cite{CMS-PAS-LUM-13-001, CMS:2012rua}.
The uncertainty of this measurement reaches 2.6\%\cite{CMS-PAS-LUM-13-001}. For the systematic variation, the luminosity value was varied up and down by it's uncertainty.

\subsection{Uncertainty on the Lepton Selection}

The lepton scale factors (described in sec. \ref{sec:sel}, lepton isolation) were determined using a tag and probe method which results in an uncertainty of 0.3\%\cite{Asin2014Auth}. 
However, this method was performed using a Drell-Yan sample. To account for the uncertainty of the method and the possible differences between the Drell-Yan and $t\bar{t}$ 
event topologies, the scale factors were conservatively varied up and down by 1\% \cite{AN-2012-389}.

\subsection{Jet Energy Scale}

The $p_{T}$ and $\eta$ dependent jet energy scale correction uncertainties are taken from \cite{CMS-PAS-JME-10-010} and are of the
order of a few percent. The simulated jet energy was scaled up and down by the values of these uncertainties. The missing energy in the 
event is recalculated correspondingly to the change of the jet energy values.

\subsection{Jet Energy Resolution}

The jet energy resolution has been rescaled up by $\eta$ dependent factors (1.052, 1.057, 1.096, 1.134, 1.288 for the $|\eta|$ ranges [0.0, 0.5, 1.1, 1.7, 2.3, 5.0]) following 
the prescriptions of the Jet/MET group \cite{TWikiSystJER}.

\subsection{$b$-tagging Efficiency Uncertainty}

The uncertainties due to the $b$-tagging algorithm is taken into account by the $b$-tag scale factors (described in sec. \ref{sec:sel}, 
$b$-jet selection) variations within their estimated
uncertainties \cite{CMS-PAS-BTV-13-001}. The variations are performed depending on the $p_{T}$ and $|\eta|$ of the jets. The variation ``up'' 
is done such that if the $p_{T}$ of the jet was greater than a median value (65$\;$GeV for $b$- and $c$-jets and 45$\;$GeV for light-jets) then 
the $b$-tagging scale factor was varied up by it's uncertainty, and if the $p_{T}$ of the jet is lower than the median then the $b$-tagging 
scale factor was varied down by it's uncertainty. The inverse logics was applied for the systematic ``down'' variation of the $b$-tagging 
scale factors. Analogous variations were performed versus the $|\eta|$ of the jets. The median value for the jet $|\eta|$ is 0.75 for all the jets flavours.

The variations of the $b$-tagging scale factors in $p_{T}$ and $\eta$ are considered as fully uncorrelated, which means that they were added
quadratically to the systematic uncertainty. The variations were also done separately for different jet flavours. The uncertainties for $c$ and $b$ 
flavours are considered to be fully correlated and they are varied simultaneously, while the variations for the light jet flavour are treated as 
uncorrelated to $c$- and $b$-jets.

\subsection{Missing Transverse Energy Uncertainty}

No specific MET variation for the systematic uncertainty was performed. The reason for this is that the missing energy is recalculated
after each variation connected to the jets or leptons kinematics. This matches with the recommendations of the experts \cite{CMS-PAS-JME-12-002}.

\subsection{Uncertainty Related to the $t\bar{t}$ Kinematic Reconstruction}

The uniform scale factor related to the $t\bar{t}$ kinematic reconstruction was presented in sec. \ref{sec:kinRecPerf}. It was varied 
up and down by 1\%, which corresponds to the value of its error, to estimate the related systematic uncertainty.

\subsection{Uncertainty on the Background Normalization}

The normalization of each background is scaled up and down by 30$\%$\cite{Asin2014Auth}. This conservative variation covers both the 
total normalization uncertainty and local deficiencies due to mismodelling of the shapes of the distributions. 

\subsection{Branching Ratio}

The uncertainties on the branching ratios of the $W$ decays are propagated from their individual uncertainties \cite{PDG-2012}.
They cancel for the normalized cross sections.
%%%%%%%%%%%%%%%%%%%%%%%%%%%%%%%%%%%%%%%%%%
%%%%%%%%%%%%%%%%%%%%%%%%%%%%%%%%%%%%%%%%%%
%%%%%%%%%%%%%%%%%%%%%%%%%%%%%%%%%%%%%%%%%%
\section{Model Uncertainties}

The model uncertainty is defined by using for the unfolding of double differential cross sections different $t\bar{t}$ signal simulations 
obtained from different generators and/or setting
different parameters for the generator input\footnote{Please note that for each discussed variation the full unfolding procedure was
repeated}. A summary of all the uncertainties taken to account in this analysis is presented in this section.

\subsection{Uncertainties Related to PDFs}\label{sec:syst_PDF}

The systematic uncertainty due to the chosen PDF model is estimated by reweighting the $t\bar{t}$ signal sample according to the 
44 errors of the CTEQ66 PDF set evaluated at the 90\% confidence level\cite{Lai:2010vv}. The effect of each of these variations
of the same sign\footnote{The sign of the systematic variation is determined by the deviation caused by this
variation. If the variation gave higher cross section results than the nominal ones, then the sign of this variation is "+``. 
If the variation resulted in lower cross section values, then the sign of this variation is "-".} are added in quadrature.

% There were 26 parameter variations
% performed and the uncertainties of the same sign of each variation were added in quadrature.

\subsection{Uncertainties Related to the Hard Scattering Model}

The nominal $t\bar{t}$ MC sample used for this analysis is generated with $\MG+\PYTHIA$. The hard scattering model is simulated
in \MG with LO precision. A systematic variation was performed using \Powheg instead of \MG for the hard scattering simulation.
\Powheg is also interfaced with \PYTHIA for the showering simulation. All the differences between the results produced with $\MG+\PYTHIA$
and $\Powheg+\PYTHIA$ are assumed to originate only from the difference in the hard scattering model. The performance of \PYTHIA is assumed 
to be the same in both samples.

\subsection{Hadronization and Parton Showering Model Uncertainties}

As discussed in chapter \ref{chapt:MC}, \PYTHIA and \HERWIG generators have different algorithms for parton showering and hadronization.
Thus, the generated $t\bar{t}$ signal samples interfaced with \PYTHIA and \HERWIG are compared to measure this uncertainty. In this work
the samples generated with $\Powheg+\HERWIG$ and $\Powheg+\PYTHIA$ were compared. All the differences in the results measured exploiting the
$\Powheg+\HERWIG$ and $\Powheg+\PYTHIA$ $t\bar{t}$ signal samples are assumed to originate from the differences between parton showering
and hadronization models.

\subsection{Top Quark Mass Assumption}

The nominal $t\bar{t}$ simulated sample used in this analysis was generated with the top mass $m(t) = 172.5$ GeV. The experimentally measured
value (world average) of this mass is $m_{exp}(t) = 172.4 \pm 0.74$ GeV \cite{ATLAS:2014wva}. To account for the experimental uncertainty of the top-quark
mass, two additional $t\bar{t}$ signal samples were generated, in which the masses of the top-quark were assumed to be $m(t) = 171.5$ GeV and
$m(t) = 173.5$ GeV. The results were recalculated using these samples and the differences to the nominal results are assumed to
be the uncertainties related to the top quark mass assumption.

The procedure of the full kinematic reconstruction of the $t\bar{t}$ system applied in this analysis (see chapter \ref{chapt:kinReco})
assumes a fixed top-quark mass of 172.5 GeV. This mass was varied separately for the kinematic reconstruction procedure by $\pm$ 1 GeV. 
This variation produced only minor difference to the nominal results, which can be neglected.
% compared to the statistical uncertainties.

\subsection{Matching Scale Variation}\label{ssec:matchS_sys}

The scale at which the parton shower is matched to the hard process was varied up and down by factors 2 and 0.5 compared to the nominal
value of 20 GeV. The resulting variations in cross sections are the corresponding systematic uncertainties.

\subsection{Hard Scale Variation}\label{ssec:hardS_sys}

The hard scale Q was varied by factors
2 and 0.5 up and down compared to the nominal value. This variation follows the convention 
adopted by the CMS experiment.

%%%%%%%%%%%%%%%%%%%%%%%%%%%%%%%%%%%%%%%%%%
%%%%%%%%%%%%%%%%%%%%%%%%%%%%%%%%%%%%%%%%%%
%%%%%%%%%%%%%%%%%%%%%%%%%%%%%%%%%%%%%%%%%%
\section{Determination of the Total Systematic Uncertainties}\label{sec:syst_det}

The total systematic uncertainty consists of the different experimental and model variations described above.
Some of them give results with lower values compared to the nominal ones, and some of them higher ones.
The deviations to lower and to higher cross section values are independently summed up in quadrature as follows:

\begin{align}
 \delta_{syst.\,total,\:ij}^{pos} = \sqrt{\sum_{s}\delta_{s,\;ij}^{high^{2}}}, \\
 \delta_{syst.\,total,\:ij}^{neg} = \sqrt{\sum_{s}\delta_{s,\;ij}^{low^{2}}}.
\end{align}

Here $\delta_{syst.\,total,\:ij}$ is the total systematic uncertainty in the bin $ij$ from
$s$ different sources. The $\delta_{s,\;ij}^{high/low}$ are expressed the following way:

\begin{equation}
 \delta_{s,\;ij}^{m} = \textrm{Varied Result}_{s,\;ij} - \textrm{Nominal result}_{ij}, 
\end{equation}
where $m = high$ if $\delta_{s,\;ij}^{m} \geq 0$ and $m = low$ if $\delta_{s,\;ij}^{m} < 0$ . 

In case all the variations of a certain correction factor or assumption result only in higher (or lower) values 
compared to the nominal result, the largest deviation is taken for the higher (lower) uncertainty and the corresponding 
lower (higher) uncertainty is set to zero. These numbers are presented
in tables from appendix \ref{appendix:xsec_table}.

% The final result is presented the following way:
% 
% \begin{equation}
%  \textrm{Result}_{ij}\: = \: \textrm{Nominal Value}_{ij}\: \pm \: \textrm{Stats. uncertainty}_{ij}\:\: { }^{+\:\delta_{syst.\,total,\:ij}^{pos}}_{-\:\delta_{syst.\,total,\:ij}^{neg}}.
% \end{equation}

% \subsection{Correlated and Uncorrelated Uncertainties}
% 
% The information about the bin-by-bin correlation of the systematic uncertainties is important if the results presented in this work
% will be used to the other measurement. Additionally to the knowledge of the value and the sign of each uncertainty, it is important 
% to know, which uncertainties are bin-by-bin correlated and which are not.
% 
% The following uncertainties could be treated as uncorrelated:
% 
% \begin{itemize}
%  \item Jet Energy Scale uncertainties
%  \item Jet Energy Resolution uncertainties
%  \item Hadrd scattering model uncertainties
%  \item Hadronization and showering model uncertainties
% \end{itemize}
% 
% All the other uncertainties should be treated as correlated.

\section{Summary of Systematic Uncertainties}

An example of the systematic uncertainties is presented in the table \ref{tab:ex}. It shows the 
systematic uncertainties for the double differential $t\bar{t}$ production cross sections in bins of $p_{T}(t)$ and $|y(t)|$.
The total uncertainties result in mainly symmetric positive and negative uncertainties.
The table also shows the statistical uncertainties in the same bins for comparison. All the systematic variations are smaller
or of the order of statistical uncertainties.

The biggest contributions to the total systematic uncertainties are the JES (up to $\sim 5\%$), top quark mass variation (up to
$\sim 13\%$), hard scale variation (up to $\sim 12\%$), matching scale variation (up to $\sim 8\%$), hadronization model
(up to $\sim 13\%$) and hard scattering model ($\sim 22\%$) uncertainties. A large value of the uncertainty related to the $b$-tagging 
in the high $p_{T}(t)$ bin is caused by statistical effects.

The uncertainties related to the kinematic reconstruction scale factor variation, background variation and PDF variation are 
small for each cross section bin.

The following uncertainties are slightly increasing with $p_{T}(t)$: JER (up to $\sim 7\%$), JES (up to $\sim 5\%$), pile up
reweighting (up to $\sim 1.5\%$), matching scale (up to $\sim 5-8\%$). The uncertainties related to the top quark mass variation,
hard scale variation and hadronization model variation are more increasing with the $p_{T}(t)$. On the other hand, the uncertainties
related to the trigger and lepton scale factor variations are slightly decreasing with $p_{T}(t)$ -- from 0.4\% to $\sim0\%$.

\input{/home/dolinska/Dropbox/desy_plots/Thesis/Jenya/xSec/sysTable/fullSysTableNorm_top_pt-top_arapidity.tex}

All the uncertainties derived from the experimental and model systematic sources described in this chapter are listed in the tables in appendix \ref{appendix:syst_unnorm},
for all the normalized cross sections described in sec. \ref{ssec:xsec_mes} and unnormalized cross sections shown in appendix \ref{appendix:unnorm_XSec}. 
% 
% 
% \input{/home/dolinska/Dropbox/desy_plots/Thesis/Jenya/xSec/sysTable/fullSysTableNorm_top_pt-top_arapidity.tex}
% \input{/home/dolinska/Dropbox/desy_plots/Thesis/Jenya/xSec/sysTable/fullSysTableNorm_top_arapidity-top_pt.tex}
% 
% \input{/home/dolinska/Dropbox/desy_plots/Thesis/Jenya/xSec/sysTable/fullSysTableNorm_ttbar_pt-ttbar_arapidity.tex}
% \input{/home/dolinska/Dropbox/desy_plots/Thesis/Jenya/xSec/sysTable/fullSysTableNorm_ttbar_arapidity-ttbar_pt.tex}
% 
% \input{/home/dolinska/Dropbox/desy_plots/Thesis/Jenya/xSec/sysTable/fullSysTableNorm_ttbar_mass-top_arapidity.tex}
% \input{/home/dolinska/Dropbox/desy_plots/Thesis/Jenya/xSec/sysTable/fullSysTableNorm_top_arapidity-ttbar_mass.tex}
% 
% \input{/home/dolinska/Dropbox/desy_plots/Thesis/Jenya/xSec/sysTable/fullSysTableNorm_ttbar_mass-top_pt.tex}
% \input{/home/dolinska/Dropbox/desy_plots/Thesis/Jenya/xSec/sysTable/fullSysTableNorm_top_pt-ttbar_mass.tex}
% 
% \input{/home/dolinska/Dropbox/desy_plots/Thesis/Jenya/xSec/sysTable/fullSysTableNorm_ttbar_mass-ttbar_delta_eta.tex}
% \input{/home/dolinska/Dropbox/desy_plots/Thesis/Jenya/xSec/sysTable/fullSysTableNorm_ttbar_delta_eta-ttbar_mass.tex}
% 
% \input{/home/dolinska/Dropbox/desy_plots/Thesis/Jenya/xSec/sysTable/fullSysTableNorm_ttbar_mass-ttbar_delta_phi.tex}
% \input{/home/dolinska/Dropbox/desy_plots/Thesis/Jenya/xSec/sysTable/fullSysTableNorm_ttbar_delta_phi-ttbar_mass.tex}
% 
% \input{/home/dolinska/Dropbox/desy_plots/Thesis/Jenya/xSec/sysTable/fullSysTableNorm_ttbar_mass-ttbar_arapidity.tex}
% \input{/home/dolinska/Dropbox/desy_plots/Thesis/Jenya/xSec/sysTable/fullSysTableNorm_ttbar_arapidity-ttbar_mass.tex}
% 
% \input{/home/dolinska/Dropbox/desy_plots/Thesis/Jenya/xSec/sysTable/fullSysTableNorm_ttbar_mass-ttbar_pt.tex}
% \input{/home/dolinska/Dropbox/desy_plots/Thesis/Jenya/xSec/sysTable/fullSysTableNorm_ttbar_pt-ttbar_mass.tex}
% 
% \input{/home/dolinska/Dropbox/desy_plots/Thesis/Jenya/xSec/sysTable/fullSysTableNorm_ttbar_mass-x1.tex}
% \input{/home/dolinska/Dropbox/desy_plots/Thesis/Jenya/xSec/sysTable/fullSysTableNorm_x1-ttbar_mass.tex}
% 
% \input{/home/dolinska/Dropbox/desy_plots/Thesis/Jenya/xSec/sysTable/fullSysTableNorm_ttbar_pt-x1.tex}
% \input{/home/dolinska/Dropbox/desy_plots/Thesis/Jenya/xSec/sysTable/fullSysTableNorm_x1-ttbar_pt.tex}
