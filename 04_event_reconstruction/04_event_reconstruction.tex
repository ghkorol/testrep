\chapter{Event Reconstruction and Section}\label{chapt:event_selection}

The reconstruction procedure is interpreting the detector data as a set of physical objects.

In general, for this analysis each object of the $t\bar{t}$ dileptonic final state was reconstructed using the \textit{Particle Flow} (PF)
algorithm \cite{Beaudette:2014cea}. It is an iterative process which reconstructs the particles and jets exploiting the information from all parts
of the detector. After an object is reconstructed in one iteration, all the detector signals assigned to it are blinded for the further iterations.

The reconstruction with the PF algorithm following \cite{Beaudette:2014cea} starts with muons as they are the most unambiguous particles to identify.
After blinding the muon signals in the detector, the charged hadrons are reconstructed. The next step is the electron reconstruction and afterwards
all the remaining signals are assigned to neutral hadrons and photons. After all particles are reconstructed and identified, the missing transverse
energy ($E_{T}^{miss}$) is constructed using all available information. A complete overview of the PF algorithm can be found in \cite{CMS-PAS-PFT-09-001}. 

Different algorithms and methods are used to reconstruct different objects and some of them (relevant for this analysis -- leptons, jets and missing transverse
energy, responsible for neutrinos) are discussed in this chapter. Finally the selection, which aims at suppressing the background events
with a similar final state signature as the $t\bar{t}$ signal events and increase the event quality, is presented. 
The selection is based on the CMS Top-Quark-Physics-Analysis group recommendations \cite{TopPAGreco}.

%%%%%%%%%%%%%%%%%%%%%%%%%%%%%%%%%%%%%%%%%%%%%%%
\section{Track and Vertex Reconstruction}

Tracks\footnote{A \textit{track} is a set of information about the charged particles trajectory and momentum.} and vertices\footnote{A 
\textit{vertex} is an estimate of a point in space where the particles arise from. It can be either related to the 
collision point or to the place  where some secondary interaction or decay happened.} are the essential objects used for all the particles reconstruction. 
However they are not the primary detector output and they have to be reconstructed.

\subsection{Track Reconstruction}\label{ssec:trkReco}

The track reconstruction takes place using the information from the inner tracking detectors. The neighboring pixels and strips which
produce signals are grouped to clusters. The position of a cluster defines a hit which is an estimate of the point where a particle crossed the
detector material. A particle trajectory can be tracked by fitting the corresponding sequence of hits. The curvature 
of the trajectory within the known magnetic field gives a transverse momentum estimate of a particle.

The track reconstruction procedure consists of four main steps as following \cite{Chatrchyan:2014fea}:

\begin{itemize}
 \item \textbf{Seed generation}: all the possible combinations of three points (3 hits in the first layers of the pixel tracker or
 two hits from the first tracker layers an then beam spot center\footnote{The \textit{beam spot} center is 
 defined as an average value of the $p\bar{p}$ interaction points over many events.}) are used to determine a helical trajectory 
 assuming a quasi-uniform magnetic field. Every resulted track is required to have some minimum $p_{T}$ and maximum beam spot impact 
 parameter\footnote{An \textit{impact parameter} $d_{xy}$ of the track is a minimum distance of the track to a certain vertex or 
 beam spot in the transverse $x-y$ plane.} to be accepted as an initial track candidate or seed.
 %
 \item \textbf{Track finding}: extrapolates the seeding candidates along the expected flight paths to the consecutive tracking detector layers searching 
 for further hits and reestimating the track parameters respectively. On each consecutive layer all the hits from a 3 sigma region around the seed trajectory
 are taken and fitted with a Kalman filter\cite{Fruhwirth:1987fm}. The best $\chi^{2}$ hit is taken and the trajectory
 continues extrapolation to the next layer until the end of tracker is reached. In case no hit is found on some layer a ghost hit is assumed. The track is not
 accepted if it contains more then one ghost hit.
 %
 \item \textbf{Track fitting}: after finding the hits in all the tracker layers with the track finding, a track candidate is refitted using the Kalman filter. 
 This procedure is performed releasing the constraints set on the seeding stage.
 %
 \item \textbf{Track selection}: reduces the rate of fake tracks\footnote{A \textit{fake track} is a track produced by the fit but is actually there was
 no real charged particle which produced this track.} by setting physical and quality constrains on the object produced at the previous step. The requirements
 are based on a good $\chi^{2}$ of the fit and a reasonable value of the impact parameter with respect to the beam spot.
\end{itemize}

The tracking procedure is repeated in several iterations.
%Each iteration differs in the set of $p_{T}$ and beam spot impact parameter
%requirements for the seed tracks generation\cite{Chatrchyan:2014fea}. 
The first iteration has the hardest requirements - high $p_{T}$ of the tracks
and the smallest impact parameters (such objects are usually easier to reconstruct). After reconstructing the tracks in the first iteration,
their hits are removed and the second iteration starts with a bit softer requirements on the seed tracks. In total, six iterations are performed.
% 
% For the HLT (see sec.\ref{sec:trig}) track reconstruction the information from pixel tracker only is used, repeating the seed generation procedure\cite{Chatrchyan:2014fea}.
% This procedure is fast enough for the trigger performance. However the tracks produced this way have a higher probability to be fake and have worse resolution, they
% are not used as the track candidates, but rather for the vertex reconstruction at the trigger level. To define track candidates a simplistic procedure
% involving pixel and strip detector information is performed on the HLT\cite{Chatrchyan:2014fea}.

\subsection{Primary Vertex Reconstruction}\label{ssec:vtxReco}

The primary vertex reconstruction aims to measure the location of all proton-proton collision points in the event (see Figure \ref{fig:HugePU}). The reconstruction procedure has three
phases described as follows\cite{Chatrchyan:2014fea}:

\begin{figure}[t]
  \centering
  \includegraphics[width=1.0\textwidth]{04_event_reconstruction/plots/url.png}
  \caption{Magnified view of the event showing 29 distinct vertices reconstructed corresponding to 29 distinct collisions within a single crossing of the LHC beam.}
  \label{fig:HugePU}
\end{figure}
\begin{itemize}

 \item \textbf{Track selection}: the tracks for the primary vertex reconstruction are chosen with certain requirements such as to be consistent with being produced 
 in the primary interaction region (the significance of the beam spot impact parameter -- it's value over it's uncertainty -- has to be maximum 5), having hits in at least two pixel 
 layers and at least five pixel and strip layers associated with the track and having a $\chi^{2}$ per degree of freedom for the track fit of not higher then 20.
 %
 \item \textbf{Track clustering}: the main task of the clustering algorithm is to separate the groups of tracks emerging from different primary vertices, not splitting 
 the tracks from one vertex to two, but also not merging the tracks from different vertices to one. The tracks are clustered according 
 to their $z$-coordinate at the point of closest approach to the beam spot center. 
 \\
 The algorithm used for clustering for the CMS 2012 data is the \textit{deterministic annealing} described in detail in\cite{rose_ieee1998}. A system with many tracks has many degrees
 of freedom and finding the primary vertex regions according to the $z$-coordinates at the closest approach to the beam spot center is treated analogically to the problem of a physical 
 system approaching a state of minimal energy through a series of gradual temperature reductions.
 %
 \item \textbf{Fitting the position of the vertex}: each clustered group of tracks forms a vertex candidate. Only the candidates with at least two tracks are taken and are fitted
 using an \textit{adaptive vertex fitter}\cite{Frühwirth:1027031}. The output of the fit is the vertex position, it's uncertainty and a $\chi^{2}$. The determination of the probability
 of each track to arise from the reconstructed vertex is performed.
\end{itemize}

The information on all primary vertices in the event is useful not only for other objects reconstruction but also for separating the signal vertex (with a hard interaction) 
from the others with no useful physics information.

%%%%%%%%%%%%%%%%%%%%%%%%%%%%%%%%%%%%%%%%%%%%%%%
%%%%%%%%%%%%%%%%%%%%%%%%%%%%%%%%%%%%%%%%%%%%%%%
%%%%%%%%%%%%%%%%%%%%%%%%%%%%%%%%%%%%%%%%%%%%%%%
\section{Objects Reconstruction}

The reconstruction of the muons, electrons, jets and missing transverse energy (which are needed for the $t\bar{t}$ dileptonic final
state reconstruction) is presented in this section.

%%%%%%%%%%%%%%%%%%%%%%%%%%%%%%%%%%%%%%%%%%%%%%%
\subsection{Muon Reconstruction}

As discussed in section \ref{sec:CMS} the CMS experiment has a well established setup for the muon reconstruction.
Muons are the only particles expected to appear in the muon sub-detector system, thus their identification is unambiguous.
Figure \ref{fig:PFmuons} shows a muon path in the detector compared to other particles.
One can get three types of muons out of the reconstruction:

\begin{figure}[t]
  \centering
  \includegraphics[width=1.0\textwidth]{04_event_reconstruction/plots/CMS_Slice.png}
  \caption{Track reconstruction using different sub-detector information in combination in Particle Flow algorithm. The different
  charged particles are shown as curved lines: muon (blue), electron (red) and charged hadron (green). Neutral particles are shown
  with the strait dashed lines.}
  \label{fig:PFmuons}
\end{figure}

\begin{itemize}
 \item \textbf{Standalone}: muon tracks are reconstructed using the information from the muon system only. The track reconstruction
 algorithm \cite{TWiki:GlobalMuon} is similar to the one used in the tracking system.
% \\
%  The first stage, \textit{seed generation}, needs 3 parameters to fit a muon seed (as in sec.\ref{ssec:trkReco}). These can be either a beam spot position, segment position and segment
%  bending angle with respect to the vertex direction or two segments from MB3 and MB4 (see figure \ref{fig:muond} in sec.\ref{ssec:muonDet}) and the difference
%  between their bending angles. Also, the algorithm allows to take two segments from the first muon stations, MB1 and MB2, beam spot position and their
%  bending angles, determine a seed for each of them and take an average. In the CSC region (see sec.\ref{ssec:muonDet}) the two segments from first and second
%  or first and third stations are taken with either the difference between their $\phi$ coordinate or the direction of the higher quality segment as a bending parameter.
%  The muon seed generation algorithm is used for the HLT muon track definition.
%  \\
%  The further procedure is to extrapolate the tracks to the other layers of the muon system (it is flexible enough to prolong the tracks from the outermost layers inside or
%  from the innermost layers outside, depending on how the seed was created). The track propagation is performed using the Kalman filter. Each time a new point of the 
%  track is found, the track is refitted (taking into account the effects from the multiple interaction with matter). In case on any step the $\chi^2$ per degree of freedom 
%  of the fit turns out to be higher then 20, this track is rejected. 
 
 \item \textbf{Tracker}: the tracks, reconstructed in the tracking detector only as described in sec. \ref{ssec:trkReco}, 
 are assigned as muons in case they match at least one hit in the muon sub-detector \cite{TWiki:GlobalMuon}.
 
 \item \textbf{Global}: the muons are reconstructed using 
 the combined fit of tracks from the muon system and from the inner tracker. Here the standalone and tracker muons are fitted together. The details are given
 in \cite{TWiki:GlobalMuon}.
\end{itemize}

%%%%%%%%%%%%%%%%%%%%%%%%%%%%%%%%%%%%%%%%%%%%%%%
\subsection{Electrons reconstruction}\label{ssec:ElRec}

The electrons are reconstructed by combining tracks from the tracker with clusters from the ECAL. The electron seeds
are created making use of two complementary algorithms as following \cite{CMS-PAS-EGM-10-004}: \textit{tracker driven seeding}
and \textit{ECAL driven seeding}. The former is performed for the tracks with $p_{T} < 5$ GeV, matching them to the ECAL superclusters\footnote
{A \textit{supercluster} is a group of one or more associated clusters of energy deposits in the ECAL. The transverse energy $E_{T}$
of the supercluster has to be not lower then 4 GeV.}. The ECAL driven seeding is performed for the tracks with $p_{T} \geq 5$ GeV
fitting the ECAL superclusters to the tracker tracks. All the seeds are fitted with the Gaussian Sum Fitter (GSM) \cite{GSF_Electron_Reconstruction_CMS}
using the hypothesis that each track is an electron and assuming the specific energy losses.

On the next reconstruction step the fitted seeding tracks are preselected. The tracker seeds are preselected by a multivariate analysis 
(MVA) as described in \cite{CMS:2010byl}. For the ECAL seeds the restrictions on the GSM matching in $\eta$ and $\phi$ are applied \cite{Baffioni:2006cd}.

The collection of the tracks formed after this selection is assigned to electrons.

%%%%%%%%%%%%%%%%%%%%%%%%%%%%%%%%%%%%%%%%%%%%%%%
\subsection{Jet Reconstruction}

A quark or a gluon as a colorful object can't exist singly due to the confinement property (see sec. \ref{sec:quark}).
A quark thus starts to hadronize and form a group of particles (primarily only hadrons) flowing in similar direction. These are the jets. Special algorithms are developed
to find and reconstruct jets.

\subsubsection{Jet Finder Algorithms}\label{sec:JetAlgo}

The idea which lies behind any algorithm of jet finding and reconstruction is the merging of objects which are measured nearby in the detector.
Generally a jet can be reconstructed using two strategies \cite{Salam:2007xv}:

\begin{itemize}
 \item \textit{Sequential clustering}: the particles are sequentially recombined until the closest (regarding some specific distance measure between the objects) combination is found.
 %
 \item \textit{Cone algorithms}: a jet is defined as a cone around some direction of dominant energy flow, or seed. Each or some of the particles are tried 
 in a role of the dominant direction seeds. The next step was to define a trial cone around the seed and accept all the particles which enter this cone.
 The sum of the four-momenta of all objects inside the jet cone is calculated. This jet candidate is assumed to be a new seed. This iterative process continues
 until stable seeds are found.
\end{itemize}

Although the cone algorithms are fast and simplistic they are not collinear and infrared safe by default.

The sequential clustering algorithms are both infrared and collinear safe. They don't rely on a stable cone. The procedure of constructing
a jet starts with defining two distances -- $d_{ij}$ (the distance between two objects, particles or pseudojets, $i$ and $j$ in the detector) and $d_{iB}$ (the distance between the object $i$
and the beam direction). These two distances are compared:

\begin{itemize}
 \item [--] $d_{ij} < d_{iB}$: the objects $i$ and $j$ are combined together to a pseudojet which enters the algorithm again as a single object;
 \item [--] $d_{ij} > d_{iB}$: the object $i$ is taken as a final jet.
\end{itemize}

The different sequential clustering algorithms differ at the level of the distance definition. In general the distances are defined as follows \cite{Cacciari:2008gp}:

\begin{align}
 d_{ij} & = min(k_{Ti}^{2p}, k_{Tj}^{2p}) \frac{\Delta_{ij}^{2}}{R^{2}},\label{eq:ktDist} \\
 d_{iB} & = k_{Ti}^{2p}, \\
 \Delta_{ij}^{2} & = (y_{i} - y_{j})^{2} + (\phi_{i} - \phi_{j})^{2}.
\end{align}

Here the $k_{Ti}$, $y_{i}$ and $\phi_i$ are the transverse momentum, rapidity and azimuth angle of an object $i$. The $R$ is a cone radius parameter and 
$p$ is the parameter which varies the power of the energetic term in comparison to the geometrical scale $\Delta_{ij}$.
There are three different sequential clustering algorithms depending on the $p$ value:

\begin{itemize}
 \item $p = 1$ defines the $k_{T}$ algorithm, where the energetic and spatial term are of the same power;
 \item $p = 0$ defines the Cambridge/Aachen algorithm, where the energetic term in eq.\ref{eq:ktDist} is removed thus the spatial part plays the only role;
 \item $p = -1$ defines the anti-$k_{T}$ algorithm.
\end{itemize}

The jets for this analyses were constructed using the anti-$k_{T}$ sequential clustering algorithm. 

\subsubsection{Jet Energy Calibration}\label{ssec:JCal}

The reconstructed jet energy should be corrected for the non-linear and non-uniform responses of the calorimeter. For this a factorized jet calibration
method is used \cite{2011JInst...611002C}. The calibration is performed sequentially in several steps:

\begin{itemize}
 \item The \textit{Offset correction} removes extra energy due to electronics noise and pile-up. It is applied on the real and simulated reconstructed data.
 %
 \item The \textit{Monte Carlo calibration} corrects the energy of the reconstructed jets such that it is equal to the average energy of the generated 
 particle jets. This correction is performed in bins of the jets $p_{T}$ and $\eta$. It is applied on the simulated reconstructed data.
 %
 \item The \textit{Relative jet energy correction} ensures a flat response as a function of $\eta$. It is applied on the simulated reconstructed data.
 %
 \item The \textit{Absolute jet energy correction} balances the $p_{T}$ response to be linear. It is calibrated on a sample of well reconstructed $Z^{0}$ bosons 
 decaying into two jets.
\end{itemize}


First calibration step deals with the additional energy in the jet which does not occur from a hard process but rather from the detector noise or pile up. 
Obviously this correction produces the factor always smaller then 1 thus the jet energy reduces on this step. Further correction steps aim to make the jet response
flat as a function of $|\eta|$ and $p_{T}$. The correction on the geometrical position dependence corrects all the energies as if they were measured 
in the most efficient barrel region using the dijet data events. The $p_{T}$ dependence correction makes use of the Drell-Yan events to exploit a good
lepton energy and momentum resolutions. Finally the residual corrections are applied on data to correct for some minor disagreement with the simulation.

In general all the correction factors in the kinematic region of interest for this analysis are smaller than $5\%$.

\subsubsection{b-Jets identification}\label{ssec:bTag}

The task of the $b$-jet identification is to distinguish a $b$-quark jet from light flavour-, $c$- and gluon-jets. In particular,
the long life time of the beauty hadrons (in the order of $10^{-12}$ s) is exploited which allows them to travel far enough from the primary interaction point 
(about 500 $\mu$m) before decaying. The point in space, which corresponds to the place of the beauty hadron decay can be reconstructed in the pixel tracking 
detector as a \textit{secondary vertex}. An important parameter which can be used to identify the jet origin is the impact parameter of the tracks arising 
from the secondary vertex with respect to the primary vertex (see Figure \ref{fig:SV}).

\begin{figure}[t]
  \centering
  \includegraphics[width=0.5\textwidth]{04_event_reconstruction/plots/btagging_cartoon.png}
  \caption{A sketch of the event with a reconstructed secondary vertex. A light blue circle is the primary vertex and a red circle is the secondary vertex. The impact 
  parameter of a track is marked with a blue dashed line and a symbol $d_{0}$.}
  \label{fig:SV}
\end{figure}

The algorithm used for this analysis is called \textit{Combined Secondary Vertex} (CSV) \cite{CMS-PAS-BTV-13-001}. It defines a likelihood-based discriminator which uses 
the lifetime information to distinguish $b$-jets, $c$-jets and light jets.

The minimum thresholds for the algorithm are defined as three \textit{working points}, loose (L), medium (M) and tight (T), as following \cite{CMS-PAS-BTV-13-001}:

\begin{itemize}
 \item [--] CSVL sets the threshold on the actual discriminator value as $\geq 0.244$, which has a $b$-tagging efficiency $\sim 80\%$ and a misidentification probability of
 light quark jets close to $10\%$;
 %
 \item [--] CSVM sets a harsher threshold on the discriminator value of $\geq 0.679$, which lowers the misidentification probability to $\sim 1\%$, but also
 reduces the $b$-tagging efficiency down to 65$\%$;
 %
 \item [--] CSVT has the hardest threshold on the discriminator value of $\geq 0.898$. This lowers the misidentification probability by other factor of 10 ($0.1\%$)
 and reduces the $b$-tagging efficiency down to 50$\%$;
\end{itemize}

In general the efficiency of the CSV algorithm was estimated in data and simulated QCD events \cite{CMS-PAS-BTV-13-001}. The resulting curve is presented in the figure \ref{fig:CSVeff}.

\begin{figure}[t]
  \centering
  \includegraphics[width=0.6\textwidth]{04_event_reconstruction/plots/Figure_012-b.pdf}
  \caption{The $b$-jet tagging efficiency as a function of the discriminator threshold for the CSV algorithm. The three CSV working points are marked with the red arrows on the $x$-axis.
  The upper panel shows the efficiency measured in data and predicted from the simulation.
  The lower panel shows the ratio between data and simulation efficiencies, where the blue line represents the combined statistical and systematical uncertainty. The plot is taken from \cite{CMS-PAS-BTV-13-001}.}
  \label{fig:CSVeff}
\end{figure}

%%%%%%%%%%%%%%%%%%%%%%%%%%%%%%%%%%%%%%%%%%%%%%%
\subsection{Missing Transverse Energy}

The colliding protons in the LHC have no transverse momentum component. Thus from the momentum conservation one expects to have a zero sum of the transverse momenta of the 
objects arising from the collisions. This can be expressed the following way:

\begin{equation}\label{eq:EnBal}
 \sum_{detected\; objects} \vec{p}_{T} + \sum_{undetected\; objects} \vec{p}_{T} = 0.
\end{equation}

The sum of the undetected objects transverse momentum is the missing transverse energy $E_{T}^{miss}$.
It can be expressed from eq. \ref{eq:EnBal} as the opposite vectorial sum of the transverse momenta over all reconstructed objects \cite{CMS-PAS-PFT-09-001}:

\begin{equation}\label{eq:PFMET}
 \vec{E}^{miss}_{T} = - \sum_{detected\; objects} \vec{p}_{T}.
\end{equation}

% \begin{align}
%  \vec{E}_{T}^{miss} & = (E_{T_{x}}^{miss}, \, E_{T_{y}}^{miss}) = - \sum_{leptons} \vec{p}_{T}^{lepton} - \sum_{jets} \vec{p}_{T}^{jet}, \\
%  E_{T}^{miss} & = \sqrt{E_{T_{x}}^{miss^{2}} + E_{T_{y}}^{miss^{2}}}.
% \end{align}

The missing transverse energy reconstruction is sensitive to the pile up objects. To correct for the 
pile up effects a special Boosted Decision Tree (BDT) was trained \cite{CMS-PAS-JME-12-002}.

An additional recoil correction\cite{TWikiMET} was applied. This was done with the help of a 
Multi Variate Analysis (MVA) and the resulting missing transverse energy is called MVA $E_{T}^{miss}$.
As shown in figure \ref{fig:METres} the MVA $E_{T}^{miss}$ exhibits a better resolution compared to the 
missing transverse energy calculated only from the PF (eq.(\ref{eq:PFMET})).

\begin{figure}[t]
  \centering
  \includegraphics[width=0.8\textwidth]{04_event_reconstruction/plots/MET_res.png}
  \caption{The resolution for the missing transverse energy defined with the PF algorithm only (PF $E_{T}^{miss}$)
  marked with a red line compared to the MVA $E_{T}^{miss}$ which is marked with a blue line.}
  \label{fig:METres}
\end{figure}

% The MVA is trained on the $Z \to e^{+}e^{-} / \mu^{+}\mu^{-}$ samples and attempts to distinguish between the events emerging from the hard interaction point and
% pile up vertices. The improvement in resolution which the exploiting of the MVA for pile up rejection gives is $\sim 6\%$. However, the differences between data
% and simulation are getting larger. To compensate for these discrepancies the \textit{recoil corrections} are applied \cite{CMS-PAS-JME-12-002}. 

% A recoil $\vec{u}$ is calculated using the well reconstructed $Z \to e^{+}e^{-} / \mu^{+}\mu^{-}$ samples and is defined as shown in the eq. \ref{eq:recoil}:
% 
% \begin{equation}\label{eq:recoil}
%  \vec{u} = - \vec{p}_{T}(Z) - \vec{E}_{T}^{miss}
% \end{equation}
% 
% The parallel ($u_{||}$) and perpendicular ($u_{\perp}$) components with respect to the $Z$ direction are simultaneously fitted in data and simulation for each bin
% of the jet multiplicity and $p_{T}(Z)$. The areas of the fit curves $f$ in data and MC simulations are required to be equal:
% 
% \begin{equation}
%  \int_{-\infty}^{u_{corr}} f_{data}\; du = \int_{-\infty}^{u_{uncorr}} f_{sim.}\; du.
% \end{equation}
% 
% The corrected recoil value $u_{corr}$ is used to reestimate the missing transverse energy as $\vec{E}_{T}^{miss} = | \vec{p}_{T}(Z) - \vec{u}_{corr} |$.

%%%%%%%%%%%%%%%%%%%%%%%%%%%%%%%%%%%%%%%%%%%%%%%
%%%%%%%%%%%%%%%%%%%%%%%%%%%%%%%%%%%%%%%%%%%%%%%
%%%%%%%%%%%%%%%%%%%%%%%%%%%%%%%%%%%%%%%%%%%%%%%
\section{Event Selection and Correction}\label{sec:sel}

The reconstructed objects in each event (which corresponds to one bunch crossing) have to fulfill certain criteria to be accepted for this analysis. These 
criteria are chosen taking to account the physical result this analysis is aiming for and the technical features of the CMS detector parts.

The imperfect correspondence of the simulation model to the real data has to be additionally corrected. The differences in the efficiencies 
of certain procedures in data and simulation are corrected by applying \textit{Scale Factors}, $SF = \frac{\epsilon_{data}}{\epsilon_{MC}}$, 
on the MC distributions. Here $\epsilon_{data}$ is the efficiency determined in the experimental data and $\epsilon_{MC}$ is the efficiency from
simulation. 
%The overall differences of the experimental data and MC distributions is corrected via reweighing the simulation rates towards the experimental once.

The results are presented in the control distributions. These plots show the comparison between the yields in the experimental data and the simulated MC samples.
The various background sources are also presented separately in the distributions. The simulated yields are additionally scaled to the data luminosity of 19.7\;$fb^{-1}$.
All the plots include the statistical uncertainties only.

All the selection requirements are summarized as follows:

\begin{itemize}
 \item [--] \textbf{Trigger selection}: The events have to be accepted by the HLT (see section \ref{sec:trig}) dilepton triggers, which require the presents of two leptons, electrons or muons, with
 minimum transverse momentum of 17 GeV and 8 GeV.
 
 The Scale Factors connected to the trigger selection were applied on the MC double-differentially in bins of the two leptons rapidities. 
 %%%%% 
%  \item [--] \textbf{Beam scrapping}: Accept only events with maximum 10 reconstructed tracks, or more then 10 in case at least quarter of them has high quality of reconstruction.
 %%%%%
%  \item [--] \textbf{Calorimeter noise removal}: Event with anomalous calorimeter noise are removed.
 %%%%%
 \item [--] \textbf{Vertex requirement}: Only the ``hardest vertex'' (with the highest sum of the $p_{T}^{2}$ of the assigned tracks) is taken for the analysis. 
 \\
 An additional weight correction depending on the event vertex multiplicity is applied on the MC. 
 The distributions presented in the figure \ref{fig:PUweight} show how the agreement
 between the experimental and simulated data improves after this reweighting is implemented.
 
 \begin{figure}[h]
 \centering
 \begin{subfigure}
   \centering
   \includegraphics[width=0.49\textwidth]{04_event_reconstruction/plots/vertex_mult_noPUw.png}
 \end{subfigure}
 \begin{subfigure}
   \centering
   \includegraphics[width=0.49\textwidth]{04_event_reconstruction/plots/vertex_mult_PUw.png}
 \end{subfigure}
 \caption{The vertex multiplicity control distribution before (left) and after (right) the vertex correction reweighing after the full event selection.
 Black dots represent the experimental data and the colored histograms are the MC simulation divided into signal and backgrounds contributions.
 The bottom plots represent the ratio between experimental and simulated data rates.}
 \label{fig:PUweight}
 \end{figure}
 
 %%%%%
 \item [--] \textbf{Lepton isolation}: All the leptons in the event have to be isolated with $I_{rel}\leq 0.15$ in a cone of $\Delta R = \sqrt{\Delta\eta^{2} + \Delta\phi^{2}} = 0.3$ 
 around the lepton track, where $I_{rel}$ is the relative isolation defined as:
  \begin{equation}\label{eq:Iso}
   I_{rel} = \frac{\sum E_{Tracker} + \sum E_{ECAL} + \sum E_{HCAL}}{p_{T}(l)}
  \end{equation}
 
 As it is shown in the figure \ref{fig:PFIso}, the lepton isolation requirement cuts on the events which dominated by the QCD background.

 \begin{figure}[h]
 \centering
 \begin{subfigure}
   \centering
   \includegraphics[width=0.49\textwidth]{04_event_reconstruction/plots/PF_e_Iso.png}
 \end{subfigure}
 \begin{subfigure}
   \centering
   \includegraphics[width=0.49\textwidth]{04_event_reconstruction/plots/PF_mu_Iso.png}
 \end{subfigure}
 \caption{The electron (left) and muon (right) relative isolation $I_{rel}$ (\ref{eq:Iso}) control distributions displaying the experimental data points
 and simulated distributions of signal and different background after the trigger selection.}
 \label{fig:PFIso}
 \end{figure}
 
 The efficiencies of the lepton isolation were determined using \textbf{tag and probe} method \cite{TWikiTP}. The corresponding scale factors are applied on the
 simulation level in bins of $p_{T}$ and $\eta$ of lepton separately for electrons and for muons.
 %%%%%
 \item [--] \textbf{Lepton pair selection}: An event has to contain at least two opposite signed leptons (electron-muon pair) with $p_{T} > 20 \; \textrm{GeV}$ and $|\eta| < 2.4$.
 The invariant mass of the system of the leading $p_{T}$ electron and muon has to be more than 20 GeV, otherwise the event is rejected. 
 %%%%%
 \item [--] \textbf{Jets selection}: Events which contain at least two jets with $p_{T} > 30$ GeV and $|\eta| \leq 2.4$ are accepted. It is natural to expect
 that events with less than two jets will be dominated by Drell-Yan background as the leading order Drell-Yan process does not contain jets in the final state. This is also 
 reflected by the jet multiplicity distribution (Fig. \ref{fig:jetMultiSel}). 
 
 \begin{figure}[h]
  \centering
  \includegraphics[width=0.6\textwidth]{04_event_reconstruction/plots/JetMulti.png}
  \caption{The control distribution of the jet multiplicity in the events after the trigger and lepton selection. The experimental data points
  and simulated distributions of signal and different backgrounds are plotted. The error bars of the data points
 correspond to the statistical uncertainties of the measurement. The bottom plots represent the data-to-MC yield ratio distributions.}
  \label{fig:jetMultiSel}
 \end{figure}
 
%  In the figure \ref{fig:mllJetSel} the dilepton mass before and after jet selection is shown, demonstrating a sizable background (especially Drell-Yan) fraction suppression power
%  of the cuts implemented in this selection step.
%  
%  \begin{figure}[h]
%  \centering
%  \begin{subfigure}
%    \centering
%    \includegraphics[width=0.49\textwidth]{04_event_reconstruction/plots/mll_step4.png}
%  \end{subfigure}
%  \begin{subfigure}
%    \centering
%    \includegraphics[width=0.49\textwidth]{04_event_reconstruction/plots/mll_step5.png}
%  \end{subfigure}
%  \caption{The control distribution of the dilepton mass in the events after the trigger and lepton selection (left) and after trigger, lepton and jet selection (right). 
%  The experimental data points and simulated distributions of signal and different background are plotted.}
%  \label{fig:mllJetSel}
%  \end{figure}
 %
 \item [--] \textbf{$b$-jets selection}: An event has to contain at least one jet, tagged as originating from the $b$-quark with the b-tagging probability according to the CSVL (sec. \ref{ssec:bTag}). The multiplicity of the $b$-tagged
 jets is presented in the figure \ref{fig:bjetMultiSel} which shows that cutting out the events with no $b$-tagged jets should remove a sizable background fraction. Indeed, the figure \ref{fig:mllbJetSel} presenting
 the dilepton mass before and after the $b$-jets selection shows this background reduction.
 
 \begin{figure}[h]
  \centering
  \includegraphics[width=0.6\textwidth]{04_event_reconstruction/plots/bJetMulti.png}
  \caption{The control distribution of the $b$-jet multiplicity in the events after the trigger, lepton and jet selection. The experimental data points
  and simulated distributions of signal and different background are plotted. The error bars of the data points
  correspond to the statistical uncertainties of the measurement. The bottom plots represent the data-to-MC yield ratio distributions.}
  \label{fig:bjetMultiSel}
  \end{figure}
  
 \begin{figure}[h]
 \centering
 \begin{subfigure}
   \centering
   \includegraphics[width=0.49\textwidth]{04_event_reconstruction/plots/mll_step5.png}
 \end{subfigure}
 \begin{subfigure}
   \centering
   \includegraphics[width=0.49\textwidth]{04_event_reconstruction/plots/mll_step6.png}
 \end{subfigure}
 \caption{The control distribution of the dilepton mass in the events after the trigger, lepton and jet selection (left) and after applying in addition the $b$-jet selection (right). 
 The experimental data points and simulated distributions of signal and different background are plotted. The error bars of the data points
 correspond to the statistical uncertainties of the measurement. The bottom plots represent the data-to-MC yield ratio distributions.}
 \label{fig:mllbJetSel}
 \end{figure}
 
 The scale factors corresponding to the $b$-tagging procedure were applied on the MC improving much the agreement between data and simulation. This effect can be seen in the 
 figure \ref{fig:bTagDiscr}, which presents the CSV discriminator distribution before and after the $b$-tagging SF reweighing. The data-to-MC ratio plots are getting closer to one with
 applying the SFs.
 
 \begin{figure}[h]
 \centering
 \begin{subfigure}
   \centering
   \includegraphics[width=0.49\textwidth]{04_event_reconstruction/plots/bTagDiscr_step6.png}
 \end{subfigure}
 \begin{subfigure}
   \centering
   \includegraphics[width=0.49\textwidth]{04_event_reconstruction/plots/bTagDiscr_step7.png}
 \end{subfigure}
 \caption{The control distribution of the $b$-tag discriminator in the events after the full even selection (\ref{sec:sel}) not applying the $b$-tag Scale Factors (left)
 and after applying the $b$-tag Scale Factors (right). The experimental data points and simulated distributions of signal and different background are plotted.
 The error bars of the data points correspond to the statistical uncertainties of the measurement.
 The bottom plots represent the data-to-MC yield ratio distributions.}
 \label{fig:bTagDiscr}
 \end{figure}
 %
 \end{itemize}

The applied selection criteria dramatically reduced the fraction of background events.

\section{Control Distributions}

The results of the reconstruction and selection described above can be illustrated by control distributions of the objects which are reconstructed for the $t\bar{t}$
final state definition.

The figure \ref{fig:CPetaptLep} shows the lepton $\eta$ and $p_{T}$. An overall reasonable agreement between the data and simulation shapes is observed in all $\eta$ regions
and $p_{T}$'s. The control distribution of the mass of the dilepton (electron-muon) system is shown in figure \ref{fig:CPmll}. The simulation and experimental data
agree well within the statistical uncertainties except for the first bin where data are higher than simulation.

 \begin{figure}[h]
 \centering
 \begin{subfigure}
   \centering
   \includegraphics[width=0.49\textwidth]{04_event_reconstruction/plots/basic_lepton_eta_step7.png}
 \end{subfigure}
 \begin{subfigure}
   \centering
   \includegraphics[width=0.49\textwidth]{04_event_reconstruction/plots/basic_lepton_pt_step7.png}
 \end{subfigure}
 \caption{The control distribution of lepton $\eta$ (left) and lepton $p_{T}$ (right) in the events after the whole event selection (\ref{sec:sel}). 
 The experimental data points and simulated distributions of signal and different background are plotted. The error bars of the data points
 correspond to the statistical uncertainties of the measurement. The bottom plots represent the data-to-MC yield ratio distributions. 
 Each distribution has two entries per event - for the electron and for the muon.}
 \label{fig:CPetaptLep}
 \end{figure}
 
 \begin{figure}[h]
  \centering
  \includegraphics[width=0.6\textwidth]{04_event_reconstruction/plots/basic_dilepton_mass_step7.png}
  \caption{The control distribution of the mass of the electron-muon system in the events after the whole \ref{sec:sel} selection. 
  The experimental data points and simulated distributions of signal and different background are plotted. The jets $p_{T}$ and multiplicity 
  distributions are presented in the logarithmic scale. The error bars of the data points
  correspond to the statistical uncertainties of the measurement. The bottom plots represent the data-to-MC yield ratio distributions.}
  \label{fig:CPmll}
 \end{figure}
 
The kinematics of the reconstructed and selected jets is presented in the figure \ref{fig:CPjetskin} which shows the control distributions for the
jets $\eta$, $p_{T}$ and the jet multiplicity in the events. The simulation describes better the central rapidity ranges. For the jet multiplicities
smaller than 5 a good agreement between experimental data and MC is observed.
 
 \begin{figure}[h]
 \centering
 \begin{subfigure}
   \centering
   \includegraphics[width=0.49\textwidth]{04_event_reconstruction/plots/basic_jet_eta_step7.png}
 \end{subfigure}
 \begin{subfigure}
   \centering
   \includegraphics[width=0.49\textwidth]{04_event_reconstruction/plots/basic_jet_pt_step7.png}
 \end{subfigure}
  \begin{subfigure}
   \centering
   \includegraphics[width=0.49\textwidth]{04_event_reconstruction/plots/basic_jet_multiplicity_step7.png}
 \end{subfigure}
 \caption{The control distribution of the jet $\eta$ (top left) and jet $p_{T}$ (top right) and jet multiplicity in the events 
 after the whole \ref{sec:sel} selection. The experimental data with the error bars corresponding to the statistical uncertainties
 and simulated distributions of signal and different background are plotted. The bottom plots represent the data-to-MC yield ratio distributions. 
 The top distributions have multiple entries per event, corresponding to all reconstructed jets.}
 \label{fig:CPjetskin}
 \end{figure}
 
The multiplicity for the $b$-tagged jets is presented in figure \ref{fig:CPbJetMult} which shows a good agreement for the multiplicities smaller than 4.

 \begin{figure}[h]
  \centering
  \includegraphics[width=0.6\textwidth]{04_event_reconstruction/plots/basic_bjet_multiplicity_step7.png}
  \caption{The control distribution of the $b$-tagged jets multiplicities in the events after the whole \ref{sec:sel} selection. 
  The experimental data points and simulated distributions of signal and different background are plotted in the logarithmic scales. 
  The error bars of the data points correspond to the statistical uncertainties of the measurement. 
  The bottom plots represent the data-to-MC yield ratio distributions.}
  \label{fig:CPbJetMult}
 \end{figure}
 
The control distributions are signal dominated which shows a good performance of the criteria for the $t\bar{t}$ final state selection.