\chapter{Event Selection}\label{chapt:event_selection}
 
This work was aiming to reconstruct the $t\bar{t}$ pair in the dilepton decay channel, or
$t\bar{t} \to W^{+}b\,W^{-}\bar{b} \to \bar{l}b\,l\bar{b}$. Thus the aim is to 
look for the event in the detector with two leptons of the different sign and two jets. The 
neutrino can't be measured directly, their presence is reflected in the non-zero transverse
missing energy $E_{T}^{miss}$. The lowest branching ratio of the dileptonic decay channel
($BR \simeq 4.8\%$) is countervailed by a a precise lepton reconstruction, which can reduce
the fraction of the background events to larger extend.

The dataset used for this measurement was collected over the full 2012 run period of the LHC
by the CMS detector with a collected integrated luminosity of $\mathcal{L} = 19.7 \pm 0.5 \textrm{pb}^{-1}$.

This chapter gives an overview of the $t\bar{t}$ final state objects reconstruction and selection. The 
procedures are based on the CMS Top-Quark-Physics-Analysis group recommendations \cite{TopPAGreco}.
The full chain of event selection are described and the details of each component reconstruction are given. 
Resulting event yields are represented in the control distributions, showing the data, simulated signal and backgrounds.

\section{Event cleaning and Pile-Up treatment}

In a very harsh and busy environment which the LHC collisions provide for the CMS detector, it is necessary to make an event 
cleaning. For this sake the \textit{filters} are used.

One of the filter removes \textit{beam scraping} events which have a large background from the beam remnants. This filter accepts 
the events with maximum 10 reconstructed tracks or those with more than 10 tracks, but at least quarter of them should have
high quality.

The phenomenon that one collision may contain several points of hadronic interaction, or \textit{primary verteces}, is called \textit{pile-up}. 
This effect challenges the data analysis thus needs a separate treatment. 

First all the verteces in the event are reconstructed.
Tracks which appear to originate from the same point in space are clustered and afterwards the position of the vertex if fitted using those tracks \cite{TrackPerf}.

After the vertex reconstruction the second filter used for this analysis selects the events with at least one "good" primary vertex.
This means that the number of associated tracks should be larger then 4 and a vertex should be positioned centrally in the detector
\footnote{Only the events with the vertex position $|z| < \textrm{24 cm}$ and $|\rho| < \textrm{2 cm}$ are accepted. All the coordinates
are given with respect to the CMS coordinate system (see sec.\label{sec:CMS} )}. To overcome the pile-up the primary vertex with the 
largest sum of transverse momenta squared ($p_{T}^{2}$) of the associated tracks is taken for further analysis.

The events with \textit{anomalous noise} from calorimeters are removed by the third filter.

\section{Trigger selection}

All the data flow in the CMS detector is reduced by triggers based on the physical requirements (see sec. \ref{sec:trig}).
The HLT triggers provide a selection which sorts the data by the processes which are taking place in the events. Thereby
a choice of a trigger which selected the events is already easing the task of selecting the correct physical process from
all the recorded data.

The events for this work were selected using several dileptonic HLTs which require at least two leptons (electrons or muons)
with the momenta larger than 15 GeV for one of them and 8 GeV for another. To identify a muon the triggers use the information
from the inner tracker and muon system. An electron is identified from the tracker and calorimeter information and additionally
has to fulfill a very loose isolation criterion. The details on the trigger paths used for this analysis are given in the
Appendix !!!.

As all of the data recorded is accepted by some HLT trigger, the determination of a particular trigger efficiency is determined with 
respect to another independent trigger. To calculate the dileptonic trigger efficiencies the triggers based on $E_{T}^{miss}$ selection
were taken as a reference.

\section{Particle Flow Concept}\label{sec:PF}

All the objects in $t\bar{t}$ dileptonic final state were reconstructed making use of \textit{Particle Flow} (PF)
algorithm \cite{Beaudette:2014cea}. In other words each particle or jet is identified exploiting the information from all parts
of the detector instead of using only one dedicated detector sector. The algorithm relies on an efficient track reconstruction,
clustering algorithm in wich is able to distinguish overlapping showers and on the efficient linking procedure to connect 
the signals from different sub-detectors.

A simplistic description of the reconstruction with the PF algorithm can be given as follows \cite{Beaudette:2014cea}.

\begin{itemize}
 \item [--] Muons are identified beforehand to exclude overlapping with the charged hadrons. Their tracks are extrapolated
 from tracker to calorimeter clusters and to the muon systems. An example of muon reconstruction using particle flow algorithm is 
 shown on the figure \ref{fig:PFmuons}. The clusters in calorimeter and tracks in the tracker which the muons are assigned to
 do not enter the further objects identification process.
 %
 \item [--] Charged hadrons are reconstructed from the tracks which after the extrapolation to the HCAL region fall within the boundaries
 of one or more calorimeter clusters. Analogically to the tracks used for muon identification, tracks associated to the charged 
 hadrons do not enter any further particle reconstruction.
 %
 \item [--] Electrons have to be reconstructed taking into account not only the tracks and ECAL deposits matching to them, but also
 adding the photons from the frequent Bremsstrahlung.
 %
 \item [--] The remaining ECAL clusters are assigned to photons, and the one from HCAL - to the neutral hadrons
\end{itemize}

\begin{figure}[t]
  \centering
  \includegraphics[width=1.0\textwidth]{04_event_reconstruction/plots/CMS_Slice.png}
  \caption{Muon track reconstructed using different sub-detector information in combination in Particle Flow algorithm. An actual
  muon track is shown with a curved blue line.}
  \label{fig:PFmuons}
\end{figure}

The resulting list of particles is used for further jet and missing transverse energy ($E_{T}^{miss}$) construction. 

The performance of the algorithm was studied with the simulated events \cite{CMS-PAS-PFT-09-001}. Particularly the jet energy
resolution gain reaches factor 3 for a low transverse momentum region. Furthermore, the angular resolution is improved by factor 2-3.

A more detailed description of each object reconstruction relevant for this analysis is given in the following sections of this chapter.
A complete overview of the PF algorithm can be found in \cite{CMS-PAS-PFT-09-001}. 

\section{Lepton reconstruction and selection}

The presence of two leptons, electrons or muons, is required for this analysis. Their reconstruction
differs a lot due to a huge mass distinction. Thus, the reconstruction of leptons and muons is discussed separately.

\subsection{Muons}

As it was discussed in section \ref{sec:CMS}, the CMS experiment has a well established setup for the muon reconstruction.
Muons are the only particles expected to appear in the muon sub-detector system, thus their identification is unambiguous.
Three techniques of muon reconstruction were implemented:

\begin{itemize}
 \item [--] \textit{Standalone}: muons are reconstructed using the information from the muon system only. As 
 this part of the detector provides a tracking information, a complete set of physical properties needed for the further
 analysis is recorded.
 
 \item [--] \textit{Tracker}: the tracks, reconstructed in the tracking detector only, are assigned as muons in case they 
 match at least one hit in the muon sub-detector.
 
 \item [--] \textit{Global}: making use of the PF algorithm (sec. \ref{sec:PF}), the global muons are reconstructed using 
 the combined fit of tracks from the muon system and from the inner tracker.
\end{itemize}

% \subsection{Electrons}
% \subsection{Lepton Isolation}
%
% \section{Jets}
% \subsection{Jet Finder Algorithm}
% \subsection{Jet selection}
% \subsection{b-Jets identification}
% 
% 
% \section{Missing Transverse Energy}
% 
% \section{Control Distributions}