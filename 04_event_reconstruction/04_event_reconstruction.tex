\chapter{Event Selection}\label{chapt:event_reconstruction}

\section{Particle Flow Concept}

\section{Jets}
\subsection{Jet Finder Algorithm}
\subsection{Jet selection}
\subsection{b-Jets identification}

\section{Lepton reconstruction}
\subsection{Muons}
\subsection{Electrons}
\subsection{Lepton Isolation}

\section{Missing Transverse Energy}

\section{Control Distributions}
% 
% After an event was accepted by the ZEUS trigger system, it was stored to tape for offline analysis.
% This chapter describes algorithms which were employed to reconstruct high level energy objects such
% as tracks, vertices or jets, based on the information provided by individual components of the ZEUS detector.
% First, the SINISTRA algorithm is discussed, which was used in the measurement presented in this thesis to identify the scattered electron.
% Track and vertex reconstruction algorithms are discussed afterwards.
% The concept of energy flow objects, which combine tracking and calorimeter information in order to obtain an optimal measurement of the hadronic final state, is then introduced.
% Finally, the jet reconstruction algorithm and three methods for the determination of the main event kinematic variables are presented.
% 
% \section{Scattered Electron Identification}\label{sect:sinistra}
% 
% A main signature of the Neutral Current Deep Inelastic Scattering (NC DIS) events, which are studied in this thesis, is the presence of the scattered electron in the detector active volume.
% In contrast, photoproduction leads to a small electron scattering angle, and the electron escapes through the beam pipe not leaving any signals in the detector. 
% In the Charged Current Deep Inelastic Scattering, the electron transforms to a neutrino which leaves the detector undetected. 
% Thus, in order to select NC DIS events it is necessary to identify the scattered electron.
% Furthermore, the electron information (scattering angle and energy) may be used for determination of the event kinematic variables (see Section~\ref{sect:kinematic_variables}).
% In this thesis, the SINISTRA~\cite{Abramowicz1995508} algorithm was employed to identify the scattered electron.
% 
% Electrons are detected primarily by the high resolution uranium calorimeter (see Section~\ref{sect:calorimeter}) where they leave energy deposits.
% The main task is to separate deposits of the electron from those of single hadrons or jets.
% This is done based on showering properties of the electromagnetic particles and hadrons.
% Electromagnetic showers in RCAL result in deposits in about three cells, with a tail to more cells due to interactions in the inactive material before the CAL.
% Hadronic showers are typically transversely much broader and longitudinally deeper. For example, a \SI{10}{\GeV} pion typically leads to deposits in about seven EMC cells and six HAC cells.
% 
% In clusters due to a single electromagnetic shower most of the energy is deposited in the EMC part;
% transversely the shower is typically contained within a window of $3\times3$ towers (i.e. $60\times\SI{60}{\cm\squared}$) centred on the tower with highest energy deposition.
% This leads to a maximum of 54 CAL variables which can describe the shower (9 HAC cells and 18 EMC cells, every cell is read out by two photomultipliers).
% In order to improve the separation power, a neural network (NN) approach is used in SINISTRA. All 54 variables are used by the NN, and additionaly the incidence angle of the particle causing
% the deposit, calculated from the coordinates of the cluster and from the primary vertex position.
% For every cluster the NN returns the probability $p$ for it to be an electromagnetic cluster ($p\sim0$ for hadronic clusters, $p\sim1$ for electromagnetic ones).
% The NN was trained on a Monte Carlo sample of low-$Q^2$ NC DIS events.
% The distribution of the probability for hadronic and electromagnetic clusters in this sample is shown in Fig.~\ref{fig:sinistra}.
% For electromagnetic clusters, the distribution is indeed peaking towards $p=1$.
% The electron energy is obtained by summing up the energy deposits in the cells associated with the cluster.
% It is corrected for the energy losses in the inactive material between the interaction point and the calorimeter.
% 
% \begin{figure}[t]
%   \centering
%   \includegraphics[width=0.7\textwidth]{/scratch/libov/writeups/05_PhD_thesis/03_event_reconstruction/plots/sinistra.png}
%   \caption{Example of the SINISTRA probability distribution for electromagnetic and hadronic clusters~\cite{Abramowicz1995508}.}
%   \label{fig:sinistra}
% \end{figure}
% 
% % separation between electrons and pions improves with energy
% % 
% % clustering: island algorithm - based on energy distributions; if several cells - one should be high energetic, adjacent - lower energy;
% % thus it is possible to separate two clusters even if cells are adjacent
% % seed: if energy deposit is higher than in each neighbouring cells; otherwise, a cell is assigned to a neighboring cell of highest energy
% % 
% % EMC energy/EMC+HAC energy > 0.8: 1\% loss of $e^-$, 84\% rejection of hadronic clusters.
% 
% \section {Track reconstruction}
% 
% A charged particle traversing the ZEUS tracking detector produces signals in the Microvertex Detector (MVD) and in the Central Tracking Detector (CTD).
% In the forward region, it may also produce signals in the Straw Tube Tracker (STT).
% Track reconstruction is the process of determination of the particle trajectory based on the information from these subdetectors.
% Since CTD and MVD are placed in the magnetic field provided by the solenoid, the curvature of the trajectory may be employed for the particle transverse momentum measurement.
% 
% The first step is the {\it pattern recognition}. This is the process of assigning particular hits from the subdetectors to a track.
% It is performed by the VCTRACK package~\cite{vctrack}.
% First, track {\it seeds} are identified.
% They are groups of hits in the outermost layers of the CTD or in the STT for the forward region.
% An STT seed is required to have at least eight hits,
% while for the CTD at least three hits (in the axial layers) are required.
% Afterwards, hits from the inner layers are added to the seeds to form a track candidate.
% The nominal primary vertex position is used to guide the hit search.
% 
% In the second step, the {\it track fit} is performed, based on the hit assignment from the previous step.
% This is done by the RTFIT program~\cite{rtfit}.
% It uses a Kalman Filter technique~\cite{kalman1960} that takes into account multiple scattering effects and the residual inhomogenities of the magnetic field.
% 
% The track fit employs a {\it helix} parametrisation specified by five parameters (see Fig.~\ref{fig:helix} for an illustration):
% \begin{itemize}
%  \item $\phi_H$ -- the angle tangent to the helix in the $XY$-plane;
%  \item $Q/R$ -- the ratio of the charge $Q$ to the radius in the $XY$-plane, $R$;
%  \item $QD_H$ -- the charge times the distance of closest approach in the $XY$-plane with respect to the reference point (0,0);
%  \item $Z_H$ -- the $Z$-coordinate of the track at $(X,Y)=(0,0)$;
%  \item $\cot\theta$ -- where $\theta$ is the polar angle of the track. 
% \end{itemize}
% The first three parameters specify a circle in the $XY$-plane (Fig.~\ref{fig:helix}, left) while the other two parameters define the location and the pitch of the track
% trajectory in $Z$ (Fig.~\ref{fig:helix}, right).
% The transverse momentum $p_T$ of the track is determined from its curvature $R$ according to $p_T=0.3\,Q\,B\,R$~\cite{Beringer:1900zz},
% where $B$ is the magnetic field in Tesla ($B=\SI{1.43}{\tesla}$ in the ZEUS tracking system), $R$ is measured meters, $Q$ is given in units of the positron charge, and $p_T$ in GeV/c.
% The momentum is calculated as $p=p_T/\sin\theta$.
% 
% \begin{figure}[h]
%   \centering
%   \includegraphics[width=0.8\textwidth]{/scratch/libov/writeups/05_PhD_thesis/03_event_reconstruction/plots/helix.png}
%   \caption{Illustration of the helix parametrisation in the $XY$-plane (left) and in the $YZ$-plane (right).}
%   \label{fig:helix}
% \end{figure}
% 
% \section {Vertex reconstruction}\label{sect:vertex}
% 
% The position of the $ep$-interaction point ({\it primary vertex}) is determined in two steps.
% In the first step, the {\it vertex finding} is performed, i.e. tracks are associated to the vertex.
% This is done by the VCTRACK package~\cite{vctrack}.
% Pairs of tracks that are loosely consistent with a common vertex close the $Z$-axis are searched for.
% Such pairs are combined and a $\chi^2$-fit is performed.
% A combination of tracks that gives the best overall $\chi^2$ in the vertex fit is considered as a set of primary tracks.
% In the second step, the vertex coordinates and track parameters at the vertex are refined by 
% usage of the  {\it Deterministic Annealing Filter} (DAF)~\cite{Fruhwirth1999197,RosePhysRevLett.65.945,Rose726788,Didierjean2010188}.
% The main idea behind is that instead of a fixed $\chi^2$ cut, a weight is applied to each track according to the Fermi function:
% $$w(\chi^2,T)=\frac{1}{1+\exp(\frac{\chi^2-\chi^2_\text{cut}}{2T})},$$
% where $T$ is called a temperature and $\chi^2_\text{cut}$ is a parameter.
% At zero temperature, the weight function $w(\chi^2,T=0)$ is a step function with $w=1$ for $\chi^2<\chi^2_\text{cut}$ and $w=0$ for $\chi^2>\chi^2_\text{cut}$, which corresponds to a fixed $\chi^2$ cut.
% An iterative fitting procedure is performed.
% In the first iteration, the values of $\chi^2$ for each track are taken from the VCTRACK vertex fit.
% A large value for the temperature $T$ is assigned and the weight for each track is calculated.
% The fit is then performed taking into account the weights, and the values of track $\chi^2$ are updated.
% Afterwards, the temperature $T$ is reduced and the procedure is repeated, with new values of $\chi^2$ as obtained in the previous iteration.
% This is done until $T$ reaches a small value.
% The described procedure leads to a better vertex determination, compared to the fixed $\chi^2$-cut approach, in particular it is well suited for fits with outliers.
% The knowledge of the primary vertex position in the $XY$-plane can be further improved by averaging it over a set of events.
% The result is referred to as the {\it beam spot}. At ZEUS it was calculated for sets of $\sim2000$ good events~\cite{miglioranzi}.
% The beam spot size for the HERA~II period is around \SI{90}{\um} (\SI{25}{\um}) in the $X$- ($Y$-) direction.
% Additionally, the {\it beam tilt} -- the dependence of the beam spot position on the $Z$-coordinate of the primary vertex, was determined.
% 
% The DAF algorithm is also used to reconstruct the secondary vertices due to decays of heavy flavour hadrons, in order to tag charm and beauty events.
% The track selection for these vertices is described in detail in Section~\ref{sect:secondary_vertex}.
% 
% \section {Energy Flow Objects}\label{sect:efo}
% 
% The ZEUS calorimeter provides energy measurements of charged and neutral hadrons as well as of electrons and photons.
% The tracking system supplies the momentum measurement of only charged particles.
% The energy resolution of the CAL improves with energy as $\propto1/\sqrt{E}$, while the transverse momentum
% resolution of the tracking system degrades as~$\propto p_T$ (see Sections~\ref{sect:mvd} and~\ref{sect:calorimeter}).
% Hence, at high energies the calorimeter information is more precise than that from tracking, and vice versa at low energies.
% Often, a particle is measured by both systems.
% Therefore, by combining the tracking and calorimeter information it is possible to improve the measurement of the hadronic system.
% This concept is known as {\it Energy Flow Objects} (EFO)~\cite{briskin, tuning}.
% The output of the algorithm are EFOs with the properties determined either from CAL or from the tracking system, or combining them.
% In this section, the procedure of EFO reconstruction is briefly described.
% 
% As a first step, the clustering of the energies in the calorimeter cells is performed.
% This procedure is required since a particle leads typically to energy deposits in several neighbouring cells (see also discussion in Section~\ref{fig:sinistra}).
% First, an {\it island clustering} is performed, separately for EMC, HAC1 and HAC2 sections (only for one HAC section in RCAL) in each part of the calorimeter (FCAL, BCAL, RCAL).
% If a cell has a larger energy deposit than all of its four direct neighbours (corner cells are not considered), it is called the {\it local maxima}.
% Otherwise, the cell is linked to the highest energy neighbouring cell.
% The procedure results in two dimensional clusters of cells, the {\it cell islands}. Figure~\ref{fig:efo}(a) illustrates this algorithm. 
% Then, a {\it cone clustering} is performed which is a linking of cell islands from EMC, HAC1 and HAC2 (only HAC1 for RCAL) into {\it cone islands}. 
% The association is based on the distance between cell islands in the $\theta-\phi$ space.
% It is converted to a probability according to a distribution obtained from a single pion Monte Carlo.
% The positions of the resulting cone islands are determined using the logarithmically weighted centre-of-gravity of the shower.
% Figure~\ref{fig:efo}~(b) illustrates the cone clustering algorithm.
% 
% After the clustering, tracks are matched to the cone islands. For this purpose tracks are extrapolated to the inner surface of CAL.
% Only tracks  with at least 4 superlayers in the CTD and with transverse momentum of $0.1<p_T<\SI{20}{\GeV}$ are considered.
% If a track passed more than 7 superlayers, the maximum $p_T$ is increased to \SI{25}{\GeV}. A track must be associated to the primary vertex.
% The track is considered as matched to an island, if the distance of closest approach between the track and the island position is less than \SI{20}{\cm} or it is less than
% the maximum island radius in the plane perpendicular to the ray joining the primary vertex and the island.
% 
% \begin{figure}[!t]
%   \centering
%     \begin{overpic}[width=0.3\textwidth,height=0.3\textheight]{/scratch/libov/writeups/05_PhD_thesis/03_event_reconstruction/plots/CELIslands.png}
%     \put(-15,89){(a)}
%     \end{overpic}\hspace{1cm}
%     \begin{overpic}[width=0.5\textwidth,height=0.3\textheight]{/scratch/libov/writeups/05_PhD_thesis/03_event_reconstruction/plots/efo.png}
%     \put(10,70){(b)}
%     \end{overpic}
%   \caption{(a) An illustration of the cell clustering algorithm~\cite{briskin}. (b) The cone clustering algorithm~\cite{tuning}. Initially there are four EMC cell islands and one HAC cell island. EMC islands 2 and 3 are  merged with HAC island 1 to form a single cone island. Matching of tracks to cone islands is also illustrated.}
%   \label{fig:efo}
% \end{figure}
% 
% The energy of EFO is determined in the following way:
% \begin{itemize}
%   \item if a track that passes the above criteria is not associated to any calorimeter island, the tracking information is used; the pion mass is assumed for the energy calculation; 
%   \item if a calorimeter object has no track counterpart, the calorimeter information (energy) is used; the object is assumed to be massless;
%   \item if a calorimeter object is associated with more than 3 tracks, the calorimeter information is used; the object is assumed to be massless;
%   \item if there was a match of a single track to a single cluster (1-to-1 match), the track information will be used if two conditions are fulfilled:
%       $E_\text{cal}/p < 1.0+1.2\,\sigma(E_\text{cal}/p)$ and $\sigma(p)/p<\sigma(E_\text{cal})/E_\text{cal}$, where $E_\text{cal}$ is the energy of the calorimeter cluster, $p$ is the track momentum (both are measured in GeV),
%       and $\sigma$ is the uncertainty on the quantity given in brackets. These requirements ensure that the calorimeter cluster is due to the track alone, and that the tracking
%       information is more precise. Otherwise, the CAL information is used.
% \end{itemize}
% More complicated cases as well as various corrections to the EFO energy are explained in detail in Ref.~\cite{tuning}.
% In this thesis, the EFOs were used to reconstruct jets (see the next section) as well as to calculate the event kinematic quantities (see Section~\ref{sect:kinematic_variables}) and the $E-p_Z$ variable. 
% 
% \section {Jet Reconstruction}\label{sect:jet_reconstruction}
% 
% A {\it jet} is a collimated flow of particles which arises due to the hadronisation of a quark or gluon emerging in a hard QCD process.
% From early days of collider physics, jets served as a tool to study properties of the underlying hard process.
% A {\it jet algorithm} is a procedure to reconstruct jets from individual particles observed in a detector (or those on the Monte Carlo true level).
% It defines the rules of assignment of particles to jets and the way the four-momentum of the jet is constructed from the four-momenta of particles constituting it ({\it recombination}).
% 
% In this analysis, the longitudinally invariant inclusive $k_T$-clustering algorithm~\cite{np:b406:187,pr:d48:3160,Catani1992291} with the $E$ recombination scheme was used.
% The procedure consists of the following steps:
% \begin{enumerate}
%   \item A list of all original particles, and a list of jets (empty at the beginning) are created;
%   \item For each entry $i$ from the particle list, the parameter $d_i$ is defined as: $$d_i=E_{T,i}^2,$$ where $E_{T,i}$ is the transverse energy of the particle; 
%   \item For each pair of particles $i$ and $j$, the distance $d_{i,j}$ is defined as: $$d_{i,j}=\min(E_{T,i}^2, E_{T,j}^2)[(\eta_i-\eta_j)^2 + (\phi_i-\phi_j)^2]/R^2,$$
%         where $\phi$ and $\theta$ are azimuthal and polar angles of the particles, respectively, and $R$ is a parameter (which is set to $R=1$ here);
%   \item The smallest number among all $d_i$ and $d_{i,j}$ is determined and labelled as $d_\text{min}$;
%   \item If $d_\text{min}$ is a $d_{i,j}$, the particles $i$ and $j$ are merged into a new (pseudo-) particle $k$ and are removed from the list.
%         The four-momentum of the new particle is given by the sum of the original particles four-momenta ($E$ recombination scheme):
%         \begin{equation}\label{eq:e_reco_scheme}
%          p_{k}=p_{i}+p_{j}. 
%         \end{equation}
%         This procedure of combining two particles leads to {\it massive} jets.
%   \item If $d_\text{min}$ is a $d_{i}$, it is removed from the list of particles and is added to the list of jets;
%   \item The procedure is repeated, starting from the step 2, until the list of particles is empty.
% \end{enumerate}
% The algorithm described is {\it infrared safe}, i.e. it is insensitive to long distance effects~\cite{pr:d48:3160}.
% In this thesis, the Energy Flow Objects (see Section~\ref{sect:efo}) served as the input to the jet algorithm; the jet selection criteria are described in Section~\ref{sect:jet_selection}.
% The same algorithm was also employed for theoretical calculations which were used for comparisons with the results of this measurement (see Section~\ref{sect:HVQDIS_settings} for details).
% 
% \section {Kinematic variables}\label{sect:kinematic_variables}
% 
% This section describes three methods for the reconstruction of main kinematic variables characterising the deep inelastic scattering process, namely the photon virtuality $Q^2$, the inelasticity $y$ and Bjorken~$x$.
% 
% \subsection{Electron Method}
% This method is based purely on the scattered electron information. The kinematic variables are given by:
% $$
% Q^2_\text{el}=2E_eE_e'(1+\cos\theta),
% $$
% $$
% y_\text{el}=1-\frac{E_e'}{2E_e}(1-\cos\theta),
% $$
% $$
% x_\text{el}=\frac{Q^2_\text{el}}{s\,y_\text{el}},
% $$
% where $\theta$ is the polar angle of the scattered electron (in the ZEUS coordinate system), $E_e$ ($E_e'$) is the energy of the initial (final) state electron.
% The method has a poor resolution for $x$ and $y$ at low values of the inelasticity $y$. 
% 
% \subsection{Jacquet-Blondel Method}
% 
% This method relies on the measurement of the hadronic system only~\cite{proc:epfacility:1979:391}.
% It is assumed that hadrons that are not detected escape through the beam pipe in the proton direction and their transverse momentum can be neglected.
% The formulae for the kinematic variables are as follows:
% $$
% y_\text{JB}=\frac{1}{2\,E_e}\sum_i\left(E_i-p_{Z,i}\right),
% $$
% $$
% Q^2_\text{JB}=\frac{(\sum_i p_{X,i})^2 + (\sum_i p_{Y,i})^2}{1-y_{JB}},
% $$
% $$
% x_\text{JB}=\frac{Q^2_\text{JB}}{s\,y_\text{JB}},
% $$
% where the sums run over all reconstructed energy flow objects (EFO); $p_{X,i}$, $p_{Y,i}$, $p_{Z,i}$ and $E_i$ are the components of the momentum and the energy of the $i^{\textrm{th}}$ EFO.
% The $Q^2$ resolution is somewhat poorer than for the electron method, while the $x$ and $y$ measurements are better, especially at low values of $y$.
% 
% \subsection{Double-Angle Method}
% 
% The Double-Angle Method~\cite{proc:HERA:1991:23} for reconstruction of the kinematic variables is intermediate between the electron method and the Jacquet-Blondel method described above,
% in the sense that it employs both electron and hadronic information.
% It uses the electron scattering angle and an angle $\gamma_\text{h}$, characterising the longitudinal and transverse momentum flow of the hadronic system~\cite{Wolf:1997fd}
% which is defined in the following way:
% $$
% \cos\gamma_\text{h}=\frac{\left(\sum_ip_{X,i}\right)^2 + \left(\sum_ip_{Y,i}\right)^2 - \left(\sum_i(E_i-p_{Z,i})\right)^2}{\left(\sum_ip_{X,i}\right)^2 + \left(\sum_ip_{Y,i}\right)^2 + \left(\sum_i(E_i-p_{Z,i})\right)^2}.
% $$
% The kinematic variables are then given by:
% $$
% Q_\text{da}^2=\frac{4E_e^2\sin\gamma_\text{h}(1+\cos\theta)}{\sin\gamma_\text{h}+\sin\theta-\sin(\gamma_\text{h}+\theta)},
% $$
% $$
% x_\text{da}=\frac{E_e}{E_p}\,\frac{\sin\gamma_\text{h}+\sin\theta+\sin(\gamma_\text{h}+\theta)}{\sin\gamma_\text{h}+\sin\theta-\sin(\gamma_\text{h}+\theta)},
% $$
% $$
% y_\text{da}=\frac{\sin\theta(1-\cos\gamma_\text{h})}{\sin\gamma_\text{h}+\sin\theta-\sin(\gamma_\text{h}+\theta)},
% $$
% where $E_p$ is the energy of the incoming proton and other variables were already defined.
% The advantage of this method is that it does not depend on the electromagnetic scale (as the electron method) and on the hadronic scale (as the Jacquet-Blondel method).
% Overall it gives the best performance compared to the Jacquet-Blondel and the electron methods.
% 
% \vspace{\baselineskip}
% In this thesis, all three approaches are employed.
% The double-angle method is used for $Q^2$ and $x$ determination in the whole phase space of the measurement.
% For the calculation of the inelasticity $y$, the electron and the Jacquet-Blondel methods are used at high and at low $y$, respectively.
