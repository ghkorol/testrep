\chapter{Results and Discussion}\label{chapt:results}

\section{Double Differential $t\bar{t}$ Production Cross Sections}\label{ssec:xsec_mes}

The $t\bar{t}$ production cross sections were measured double differentially in bins of top transverse momentum, rapidity and  $x_{1}$,
which corresponds to the parton momentum transfered to the $t$-quark,
$t\bar{t}$ mass, rapidity and transverse momentum and the angles $\phi$ and $\eta$ between $t$ and $\bar{t}$.
The binning was chosen to have enough entries in each bin at the detector level so that the error could be treated as Gaussian.
It was also checked if the purity, stability and efficiency in each bin are not too low (see the explanation below).
The regularization strength used to determine each set of the cross sections is listed in the Appendix \ref{appendix:tau}.

This sections is presenting only the normalized double differential $t\bar{t}$ production cross sections. The corresponding
unnormalized cross sections are shown in the Appendix \ref{appendix:unnorm_XSec}. All the numerical values for the cross sections
and their uncertainties are listed in Appendix \ref{appendix:xsec_table}.

%%%
\subsubsection{Cross sections in bins of $|y(t)|$ versus $p_{T}(t)$}

The production cross section of the $t\bar{t}$ in bins of the rapidity and the transverse momentum of the top quark was
measured in the bins presented in Fig. \ref{fig:CP_2D_y_pt}. This control distribution is also showing the agreement between 
the data and reconstructed MC $t\bar{t}$ signal. The MC slightly underestimates the data for the lower $p_{T}(t)$ bins and 
in the outer $y(t)$ bins for all the transverse momenta values.

The Fig. \ref{fig:XS_2D_y_pt} and fig. \ref{fig:XS_2D_y_pt1} represent the production cross sections of the $t\bar{t}$ pair in bins of top rapidity and top transverse momentum.
The experimentally measured cross sections are compared to the \MG+\PYTHIA, \Powheg+\PYTHIA, \Powheg+\HERWIG and \MCNLO+\HERWIG predictions.
The overall good agreement between theory predictions and experimental results is observed. \MG+\PYTHIA describes the higher $p_{T}$ bins
worth.

%%%

\subsubsection{Cross sections in bins of $p_{T}(t\bar{t})$ versus $|y(t\bar{t})|$}

Another pair of variables in bins of which the cross section was measured is the $p_{T}$ and $|y|$ of the $t\bar{t}$ system. 
The control distribution in bins of the is presented in the Fig. \ref{fig:CP_2D_pttt_ytt}. The agreement between data and MC is overall nice, 
except for the highest $p_{T}(t\bar{t})$ bin. The MC slightly underestimates the data for the first three $p_{T}(t\bar{t})$ bins, while
for the highest measured $p_{T}(t\bar{t})$ bin the MC overestimates data. 

The production cross sections in bins of $p_{T}(t\bar{t})$ and $|y(t\bar{t})|$ are shown in fig. \ref{fig:XS_2D_pttt_ytt} and fig. \ref{fig:XS_2D_pttt_ytt1}.
These plots show that the measured cross sections are more central in $|y(t\bar{t})|$ than the models. The models overestimate the cross
sections in high $p_{T}(t\bar{t})$.


%%%

\subsubsection{Cross sections in bins of $M(t\bar{t})$ versus $|y(t)|$}

Another measurement has been performed in bins of $M(t\bar{t})$ and $|y(t)|$. Fig. \ref{fig:CP_2D_Mtt_y} represents the control distribution in bins of these variables.
The agreement between the data and MC prediction is good in the lower $t\bar{t}$ mass bins. However, the MC starts to underestimate data for the highest $M(t\bar{t})$.
In general, the agreement in the outer rapidity bin is always slightly worse than for the central rapidity bins. 

The cross sections measured in bins of $M(t\bar{t})$ and $|y(t)|$ are presented in fig. \ref{fig:XS_2D_Mtt_yt}.

%%%

\subsubsection{Cross sections in bins of $p_{T}(t)$ versus $M(t\bar{t})$}

The other pair of variables in which the $t\bar{t}$ production cross section was measured double differentially is $p_{T}(t)$ and $M(t\bar{t})$.
The control plot, which shows the binning and the comparison between data and simulation, is presented in Fig. \ref{fig:CP_2D_Mtt_pt}. The MC underestimates 
the data for the lower $p_{T}(t)$ bins and this disagreement grows with the $M(t\bar{t})$. For the last two $p_{T}(t)$ bins, the MC overestimates data. A clear
linear trend is observed at the MC-to-data ratio plots.

The double differential production cross sections in bins of $M(t\bar{t})$ and $p_{T}(t)$ are presented in fig. \ref{fig:XS_2D_Mtt_pt} and in fig. \ref{fig:XS_2D_Mtt_pt1}. A slight underestimation is 
observed in the lowest $p_{T}(t)$ bin. The worst description is in the highest $M(t\bar{t})$ bin.

%%%
\subsubsection{Cross section in bins of $\Delta\eta(t\bar{t})$ versus $M(t\bar{t})$}

The azimuthal angle between the $t$ and $\bar{t}$ is represented with the pseudorapidity $\Delta\eta(t\bar{t})$. The cross section has been measured
double differentially in bins of $\Delta\eta(t\bar{t})$ and $M(t\bar{t})$.

The control distribution in Fig. \ref{fig:CP_2D_eta_Mtt} shows that the simulation slightly overestimates the experimental data in lowest $M(t\bar{t})$
bin. However, there is a strong disagreement in between MC and data in the $\Delta\eta(t\bar{t})$ spectra for the two higher $M(t\bar{t})$ bins. There is a 
clear negative slope in the MC-to-data ratio plots in these two bins.

The double differential production cross sections in bins of $\Delta\eta(t\bar{t})$ and $M(t\bar{t})$ is presented in fig. \ref{fig:XS_2D_eta_Mtt}. The \MG + \PYTHIA
prediction shows the worst agreement. There is a slight tendency that the higher the mass is, the more MC underestimates the data for the higher $\Delta\eta(t\bar{t})$.

%%%
\subsubsection{Cross section in bins of $\Delta\phi(t\bar{t})$ versus $M(t\bar{t})$}

The measurement of the cross section has been performed in bins of the polar angle between top and antitop, $\Delta\phi(t\bar{t})$, and the mass
of the $t\bar{t}$ system, $M(t\bar{t})$.

The control distribution in bins of these variable pair is presented in Fig. \ref{fig:CP_2D_phi_Mtt}. The simulation describes the data well in the lowest $M(t\bar{t})$ bin, 
however at the higher mass bins, MC starts to underestimate data in the $\Delta\phi(t\bar{t})$ spectra. There is a positive linear slope observed in the MC-to-data ratio plots
in these $M(t\bar{t})$ bins.

The Fig. \ref{fig:XS_2D_phi_Mtt} presents the double differential production cross sections of the $t\bar{t}$ pairs in bins of $\Delta\phi(t\bar{t})$ and $M(t\bar{t})$.
All the predictions provide a reasonable description of the measured cross sections. However, the agreement is getting slightly worse for the higher bins of the $M(t\bar{t})$.

%%%
\subsubsection{Cross section in bins of $|y(t\bar{t})|$ versus $M(t\bar{t})$}

The control distribution in bins of $|y(t\bar{t})|$ and $M(t\bar{t})$ is shown in fig. \ref{fig:CP_2D_ytt_Mtt}. The agreement between data and MC is overall nice.
However a slight linear slope in the MC-to-data plots can be observed. The higher the invariant mass of the top pair is, the stronger the slope gets.

The normalized double differential $t\bar{t}$ production cross sections in bins of $M(t\bar{t})$ and $|y(t\bar{t})|$ are presented in fig. \ref{fig:XS_2D_ytt_Mtt}.
The measured cross sections tend to be more central in $y(t\bar{t})$ then theoretical predictions. The description at the highest $M(t\bar{t})$ bins is the worst.

%%%
\subsubsection{Cross section in bins of $|p_{T}(t\bar{t})|$ versus $M(t\bar{t})$}

Another measurement of the cross sections was performed in bins of $|p_{T}(t\bar{t})|$ and $M(t\bar{t})$.
The control distribution in bins of $|p_{T}(t\bar{t})|$ and $M(t\bar{t})$ is presented in fig. \ref{fig:CP_2D_pttt_Mtt}. In all of the $M(t\bar{t})$
bins there is the same trend in the way how the simulation describes the experimentally measured $p_{T}(t\bar{t})$ spectrum. The MC slightly underestimates
the data for the lower $p_{T}(t\bar{t})$, while for the highest transverse momentum of the top-pair the significant overestimation is observed.

Fig. \ref{fig:XS_2D_Mtt_pttt} and fig. \ref{fig:XS_2D_Mtt_pttt1} show the double differential $t\bar{t}$ production cross section in bins of $|p_{T}(t\bar{t})|$ and $M(t\bar{t})$.
All the theoretical predictions describe the measured cross sections well, except for the highest $t\bar{t}$ masses and transverse momenta, where an
overestimation of measurement by theory is present.

%%%
\subsubsection{Cross section in bins of $p_{T}(t\bar{t})$ versus  $x_{1}$}

The $t\bar{t}$ cross section was measured in bins of $p_{T}(t\bar{t})$ and  $x_{1}$.
The corresponding control plot is shown in fig. \ref{fig:CP_2D_pttt_x1}. The agreement between simulation and data is overall nice, except for the highest
$x_{1}$ bin for all of the $p_{T}(t\bar{t})$, where MC overestimates data.

Fig. \ref{fig:XS_2D_x1_pttt} and fig. \ref{fig:XS_2D_x1_pttt1} show the double differential $t\bar{t}$ production cross sections in bins of $p_{T}(t\bar{t})$ and $x_{1}$.
The predictions overshoot the data in the highest bin of $p_{T}(t\bar{t})$ and the highest bin in $x_{1}$ has the worst agreement between measured and theoretical
cross sections.

%%%
\subsubsection{Cross section in bins of $M(t\bar{t})$ versus  $x_{1}$}

The control distributions of $x_{1}$ in bins of $M(t\bar{t})$ are shown in fig. \ref{fig:CP_2D_Mtt_x1}. The agreement between simulation and data is overall nice.
A slight underestimation of the data by simulation is observed for the highest $M(t\bar{t})$ bin.

The double differential $t\bar{t}$ production cross sections are shown in fig. \ref{fig:XS_2D_x1_Mtt} and fig. \ref{fig:XS_2D_x1_Mtt1}. The models do not describe the high $x_{1}$ region. The 
high $M(t\bar{t})$ bins are also poorly described. For the lower $x_{1}$ bin, \MG shows the best agreement with the measured cross sections.


\section{Summary}

After presenting the normalized\footnote{As the analysis presented in this work was tuned to measure the normalized cross sections, the unnormalized cross sections are not
discussed in this section. The results with unnormalized double differential top-quark-pair production cross sections are presented in Appendices \ref{appendix:unnorm_XSec}
and \ref{appendix:xsec_table}} double differential top-quark-pair production cross sections in bins of ten different combinations of variables,
one can summarize that overall the theoretical description of the measured data points is good. The precision of the measurements is allows a visual 
comparison of the experimental results to different theoretical models in LO and NLO.

All the double differential distributions and normalized cross sections, shown in previous section, can be divided into three groups:

\begin{itemize}
 \item [--] results, which describe the $t$ dynamics;
 \item [--] results, which describe the $t\bar{t}$ system dynamics;
 \item [--] results on combined $t$  and $t\bar{t}$ dynamics.
\end{itemize}

The results presented in bins of $p_{T}(t)$ and $|y(t)|$ (fig. \ref{fig:XS_2D_y_pt} and fig. \ref{fig:XS_2D_y_pt1}) are describing the $t$
dynamics. The observation, made previously in the analysis on measurement of top-quark-pair differential cross sections in dileptonic channel at 
$\sqrt{s}$ = 8 TeV \cite{Asin2014Auth}, was that the top-quark transverse momentum spectrum was softer in data than most of the predictions. The best
description was provided by $\Powheg+\HERWIG$ predictions. Looking at the $p_{T}(t)$ distributions in different $y(t)$ bins (fig. \ref{fig:XS_2D_y_pt1}), as presented in this work,
one can confirm this effect. However, this effect is very weak in the highest $|y(t)|$ bin. The trend towards softer $p_{T}(t)$ in data is even better
observed in fig. \ref{fig:XS_2D_y_pt1}, which shows the $|y(t)|$ distribution in different bins of $p_{T}(t)$. One can also conclude that the $\Powheg+\HERWIG$
predictions provide the best description of the data point, except for the highest $|y(t)|$ regions.

If talking about the $t\bar{t}$ system dynamics, the results in bins of $p_{T}(t\bar{t})$ and $|y(t\bar{t})|$ (fig. \ref{fig:XS_2D_pttt_ytt} and fig \ref{fig:XS_2D_pttt_ytt1})
show overall good agreement with the predictions. $\MG+\PYTHIA$ describe experimental points the best way. However, the slight tendency to the softer $p_{T}(t\bar{t})$
is observed. The same tendency is observed for the distributions in bins of $p_{T}(t\bar{t})$ and $M(t\bar{t})$ (fig. \ref{fig:XS_2D_Mtt_pttt} and fig. \ref{fig:XS_2D_Mtt_pttt1}).

The distributions, which represent the combined dynamics of $t$ and $t\bar{t}$, are the distributions of $M(t\bar{t})$ in different bins of $p_{T}(t)$, $|y(t)|$,
$\Delta\eta(t\bar{t})$ and $\Delta\phi(t\bar{t})$ (fig. \ref{fig:XS_2D_Mtt_pt}, fig. \ref{fig:XS_2D_Mtt_pt1}, fig. \ref{fig:XS_2D_Mtt_yt}, fig. \ref{fig:XS_2D_eta_Mtt}
and fig. \ref{fig:XS_2D_phi_Mtt}). The general observation is that the highest $M(t\bar{t})$ bin has the worst description of data by the predictions. The 
trend in the distributions in fig. \ref{fig:XS_2D_eta_Mtt} is discussed in sec. \ref{sec:deta_discuss}.

\section{Discussion}
In general, the cross sections measured and presented in this work are in general in good argeement with LO and NLO predictions
implemented in differential MC event generators (LO $\MG+\PYTHIA$ and NLO $\Powheg+\HERWIG$, $\Powheg+\PYTHIA$ and $\MCNLO+\HERWIG$).
However, there are some disagreements and trends observed in particular cross sections bins and control distributions.
Those should be discussed in more detail.

\subsection{Trend in the $\Delta\eta$ Between Top and Anti-top} \label{sec:deta_discuss}

If one throws a look onto the $t\bar{t}$ cross sections in bins of $\Delta\eta{t\bar{t}}$ in different bins of $t\bar{t}$ mass
(see fig.\ref{fig:XS_2D_eta_Mtt}), one observes the tendency that the $\Delta\eta{t\bar{t}}$ is not described in the higher mass bins.
The control distribution (see fig. \ref{fig:CP_2D_eta_Mtt}) shows the trend, that the higher the $M(t\bar{t})$ is, the smaller $\Delta\eta(t\bar{t})$
is modeled in $\MG+\PYTHIA$. That means that the top quarks, reconstructed from the experimental data, are more often back-to-back 
than for the MC.

The reason for the discrepancies in the pseudorapidities between two top-quarks may originate because of two reasons:

\begin{itemize}
 \item \textit{Wrong PDFs in the MC}: If the $x_{1}$ and $x_{2}$ transmitted from partons to first and second top quark are very different, than these
 top quarks will fly not back-to-back. This scenario, however, is unlikely, as the effect of underestimation of $\Delta\eta(t\bar{t})$
 by MC is observed stronger if the mass of the $t\bar{t}$ system is larger. For large $M(t\bar{t})$ the transmitted momenta from the partons
 to the top quarks are equivalently large.
 
 \item \textit{Additional radiation}: The presence of radiation may deviate the direction of the top quarks so that they will no longer be back-to-back.
\end{itemize}

One might conclude that MC models more radiation on higher $M(t\bar{t})$. This can be checked by looking at the $\Delta\phi(t\bar{t})$ in this region.
$\Delta\phi(t\bar{t})$ is not sensitive to the differences in $x_{1}$ and $x_{2}$ as there is no transverse component in proton momentum.
This control distribution is presented in fig. \ref{fig:CP_phi_eta}. It shows that there is a trend that there are more MC the smaller $\Delta\eta(t\bar{t})$.
In addition, there is a trend that there are more MC for the smaller $\Delta\phi(t\bar{t})$. This means, that there is more radiation modeled
in MC out of which there is more hard radiation (where $\Delta\phi(t\bar{t})$ is smaller).

\begin{figure}[t]
  \centering
  \includegraphics[width=0.8\textwidth]{09_conclusions/plots/CP_phi_eta.png}
  \caption{Control distribution of $\Delta\phi{t\bar{t}}$ in bins of $\Delta\eta(t\bar{t})$ for $M(t\bar{t})$ > 600 GeV.}
  \label{fig:CP_phi_eta}
\end{figure}

After comparing the results of the model predictions available in this analysis (see fig.\ref{fig:XS_2D_eta_Mtt}), one can conclude that this effect is the
strongest for $\MG+\PYTHIA$.



%%%%%%%%%%%%%%%%%%%%%%%%%%%
%%%%%%%Plots%%%%%%%%%%%%%%%
%%%%%%%%%%%%%%%%%%%%%%%%%%%

%%%%%Control Plots%%%%%%%%%

\begin{figure}[p]
  \centering
  \includegraphics[width=1.0\textwidth]{/home/dolinska/Dropbox/desy_plots/Thesis/Jenya/xSec/CP/CP_AllBins_top_arapidity_vs_top_pt.pdf}
  \caption{Control distribution of the $y$ of the top quark in bins of the $p_{T}$ of the top quark. The $|y|$ bins are shown on the top 
  of the plot. The experimental data are marked with the black dots and the reconstructed MC signal is marked with the red area. The error
  bars on the data points represent the statistical uncertainty only. The 
  different background contributions are also shown. On the bottom part of the plot the ratio between MC and data statistics in each bin
  is presented.}
  \label{fig:CP_2D_y_pt}
\end{figure}

\begin{figure}[p]
  \centering
  \includegraphics[width=1.0\textwidth]{/home/dolinska/Dropbox/desy_plots/Thesis/Jenya/xSec/CP/CP_AllBins_ttbar_arapidity_vs_ttbar_pt.pdf}
  \caption{Control distribution of the $p_{T}(t\bar{t})$ in bins of the $|y(t\bar{t})|$. The $|y(t\bar{t})|$ bins are shown on the top 
  of the plot. The experimental data are marked with the black dots and the reconstructed MC signal is marked with the red area. The error
  bars on the data points represent the statistical uncertainty only. The 
  different background contributions are also shown. On the bottom part of the plot the ratio between MC and data statistics in each bin
  is presented.}
  \label{fig:CP_2D_pttt_ytt}
\end{figure}

\begin{figure}[p]
  \centering
  \includegraphics[width=1.0\textwidth]{/home/dolinska/Dropbox/desy_plots/Thesis/Jenya/xSec/CP/CP_AllBins_top_arapidity_vs_ttbar_mass.pdf}
  \caption{Control distribution of the $M(t\bar{t})$ in bins of the $|y(t)|$. The $|y(t)|$ bins are shown on the top 
  of the plot. The experimental data are marked with the black dots and the reconstructed MC signal is marked with the red area. The error
  bars on the data points represent the statistical uncertainty only. The 
  different background contributions are also shown. On the bottom part of the plot the ratio between MC and data statistics in each bin
  is presented.}
  \label{fig:CP_2D_Mtt_y}
\end{figure}

\begin{figure}[p]
  \centering
  \includegraphics[width=1.0\textwidth]{/home/dolinska/Dropbox/desy_plots/Thesis/Jenya/xSec/CP/CP_AllBins_top_pt_vs_ttbar_mass.pdf}
  \caption{Control distribution of the $M(t\bar{t})$ in bins of the $p_{T}$ of the top quark. The $p_{T}(t)$ bins are shown on the top 
  of the plot. The experimental data are marked with the black dots and the reconstructed MC signal is marked with the red area. The error
  bars on the data points represent the statistical uncertainty only. The 
  different background contributions are also shown. On the bottom part of the plot the ratio between MC and data statistics in each bin
  is presented.}
  \label{fig:CP_2D_Mtt_pt}
\end{figure}

\begin{figure}[p]
  \centering
  \includegraphics[width=1.0\textwidth]{/home/dolinska/Dropbox/desy_plots/Thesis/Jenya/xSec/CP/CP_AllBins_ttbar_delta_eta_vs_ttbar_mass.pdf}
  \caption{Control distribution of the $\Delta\eta$ between the $t$ and $\bar{t}$ in bins of the $M(t\bar{t})$. The $\Delta\eta$ bins are shown on the top 
  of the plot. The experimental data are marked with the black dots and the reconstructed MC signal is marked with the red area. The error
  bars on the data points represent the statistical uncertainty only. The 
  different background contributions are also shown. On the bottom part of the plot the ratio between MC and data statistics in each bin
  is presented.}
  \label{fig:CP_2D_eta_Mtt}
\end{figure}

\begin{figure}[p]
  \centering
  \includegraphics[width=1.0\textwidth]{/home/dolinska/Dropbox/desy_plots/Thesis/Jenya/xSec/CP/CP_AllBins_ttbar_delta_phi_vs_ttbar_mass.pdf}
  \caption{Control distribution of the $\Delta\phi$ between the $t$ and $\bar{t}$ in bins of the $M(t\bar{t})$. The $\Delta\phi$ bins are shown on the top 
  of the plot. The experimental data are marked with the black dots and the reconstructed MC signal is marked with the red area. The error
  bars on the data points represent the statistical uncertainty only. The 
  different background contributions are also shown. On the bottom part of the plot the ratio between MC and data statistics in each bin
  is presented.}
  \label{fig:CP_2D_phi_Mtt}
\end{figure}

\begin{figure}[p]
  \centering
  \includegraphics[width=1.0\textwidth]{/home/dolinska/Dropbox/desy_plots/Thesis/Jenya/xSec/CP/CP_AllBins_ttbar_arapidity_vs_ttbar_mass.pdf}
  \caption{Control distribution of the $|y(t\bar{t})|$ in bins of the $M(t\bar{t})$. The $|y(t\bar{t})|$ bins are shown on the top 
  of the plot. The experimental data are marked with the black dots and the reconstructed MC signal is marked with the red area. The error
  bars on the data points represent the statistical uncertainty only. The 
  different background contributions are also shown. On the bottom part of the plot the ratio between MC and data statistics in each bin
  is presented.}
  \label{fig:CP_2D_ytt_Mtt}
\end{figure}

\begin{figure}[p]
  \centering
  \includegraphics[width=1.0\textwidth]{/home/dolinska/Dropbox/desy_plots/Thesis/Jenya/xSec/CP/CP_AllBins_ttbar_pt_vs_ttbar_mass.pdf}
  \caption{Control distribution of the $|p_{T}(t\bar{t})|$ in bins of the $M(t\bar{t})$. The $|p_{T}(t\bar{t})|$ bins are shown on the top 
  of the plot. The experimental data are marked with the black dots and the reconstructed MC signal is marked with the red area. The error
  bars on the data points represent the statistical uncertainty only. The 
  different background contributions are also shown. On the bottom part of the plot the ratio between MC and data statistics in each bin
  is presented.}
  \label{fig:CP_2D_pttt_Mtt}
\end{figure}

\begin{figure}[p]
  \centering
  \includegraphics[width=1.0\textwidth]{/home/dolinska/Dropbox/desy_plots/Thesis/Jenya/xSec/CP/CP_AllBins_ttbar_pt_vs_x1.pdf}
  \caption{Control distribution of the $p_{T}(t\bar{t})$ in bins of $x_{1}$. The $p_{T}(t\bar{t})$ bins are shown on the top 
  of the plot. The experimental data are marked with the black dots and the reconstructed MC signal is marked with the red area. The error
  bars on the data points represent the statistical uncertainty only. The 
  different background contributions are also shown. On the bottom part of the plot the ratio between MC and data statistics in each bin
  is presented.}
  \label{fig:CP_2D_pttt_x1}
\end{figure}

\begin{figure}[p]
  \centering
  \includegraphics[width=1.0\textwidth]{/home/dolinska/Dropbox/desy_plots/Thesis/Jenya/xSec/CP/CP_AllBins_x1_vs_ttbar_mass.pdf}
  \caption{Control distribution of $x_{1}$ in bins of $M(t\bar{t})$. The $x_{1}$ bins are shown on the top 
  of the plot. The experimental data are marked with the black dots and the reconstructed MC signal is marked with the red area. The error
  bars on the data points represent the statistical uncertainty only. The 
  different background contributions are also shown. On the bottom part of the plot the ratio between MC and data statistics in each bin
  is presented.}
  \label{fig:CP_2D_Mtt_x1}
\end{figure}

%%%%%%%%%%%%%%%%%%%%%%%%%%%
%%%%%%%XSec%%%%%%%%%%%%%%%%

\begin{figure}[p]
\centering
\begin{subfigure}
  \centering
  \includegraphics[width=0.49\textwidth]{/home/dolinska/Dropbox/desy_plots/Thesis/Jenya/xSec/xsecNorm/xSec_top_pt_IN_top_arapidity_0.pdf}
\end{subfigure}
\begin{subfigure}
  \centering
  \includegraphics[width=0.49\textwidth]{/home/dolinska/Dropbox/desy_plots/Thesis/Jenya/xSec/xsecNorm/xSec_top_pt_IN_top_arapidity_1.pdf}
\end{subfigure}
\begin{subfigure}
  \centering
  \includegraphics[width=0.49\textwidth]{/home/dolinska/Dropbox/desy_plots/Thesis/Jenya/xSec/xsecNorm/xSec_top_pt_IN_top_arapidity_2.pdf}
\end{subfigure}
\caption{Normalized differential cross sections in bins of top absolute rapidity and transverse momentum. The inner error bands are the statistical uncertainties from the data.
         The outer error bars are the combines statistical and systematical uncertainties on the data. The cross sections predicted different models are also presented:
         \MG + \PYTHIA (red line), \Powheg + \PYTHIA (blue line), \Powheg + \HERWIG (orange line) and \MCNLO + \HERWIG (green line).}
\label{fig:XS_2D_y_pt}
\end{figure}
\begin{sidewaysfigure}[p]
\centering
\begin{subfigure}
  \centering
  \includegraphics[width=0.325\textwidth]{/home/dolinska/Dropbox/desy_plots/Thesis/Jenya/xSec/xsecNorm/xSec_top_arapidity_IN_top_pt_0.pdf}
\end{subfigure}
\begin{subfigure}
  \centering
  \includegraphics[width=0.325\textwidth]{/home/dolinska/Dropbox/desy_plots/Thesis/Jenya/xSec/xsecNorm/xSec_top_arapidity_IN_top_pt_1.pdf}
\end{subfigure}
\begin{subfigure}
  \centering
  \includegraphics[width=0.325\textwidth]{/home/dolinska/Dropbox/desy_plots/Thesis/Jenya/xSec/xsecNorm/xSec_top_arapidity_IN_top_pt_2.pdf}
\end{subfigure}
\begin{subfigure}
  \centering
  \includegraphics[width=0.325\textwidth]{/home/dolinska/Dropbox/desy_plots/Thesis/Jenya/xSec/xsecNorm/xSec_top_arapidity_IN_top_pt_3.pdf}
\end{subfigure}
\begin{subfigure}
  \centering
  \includegraphics[width=0.325\textwidth]{/home/dolinska/Dropbox/desy_plots/Thesis/Jenya/xSec/xsecNorm/xSec_top_arapidity_IN_top_pt_4.pdf}
\end{subfigure}
\caption{Normalized differential cross sections in bins of top transverse momentum and absolute rapidity. The inner error bands are the statistical uncertainties from the data.
         The outer error bars are the combines statistical and systematical uncertainties on the data. The cross sections predicted different models are also presented:
         \MG + \PYTHIA (red line), \Powheg + \PYTHIA (blue line), \Powheg + \HERWIG (orange line) and \MCNLO + \HERWIG (green line).}
\label{fig:XS_2D_y_pt1}
\end{sidewaysfigure}

\begin{figure}[p]
\centering
\begin{subfigure}
  \centering
  \includegraphics[width=0.49\textwidth]{/home/dolinska/Dropbox/desy_plots/Thesis/Jenya/xSec/xsecNorm/xSec_ttbar_pt_IN_ttbar_arapidity_0.pdf}
\end{subfigure}
\begin{subfigure}
  \centering
  \includegraphics[width=0.49\textwidth]{/home/dolinska/Dropbox/desy_plots/Thesis/Jenya/xSec/xsecNorm/xSec_ttbar_pt_IN_ttbar_arapidity_1.pdf}
\end{subfigure}
\begin{subfigure}
  \centering
  \includegraphics[width=0.49\textwidth]{/home/dolinska/Dropbox/desy_plots/Thesis/Jenya/xSec/xsecNorm/xSec_ttbar_pt_IN_ttbar_arapidity_2.pdf}
\end{subfigure}
\caption{Normalized differential cross sections in bins of top pair absolute rapidity and transverse momentum. The inner error bands are the statistical uncertainties from the data.
         The outer error bars are the combines statistical and systematical uncertainties on the data. The cross sections predicted different models are also presented:
         \MG + \PYTHIA (red line), \Powheg + \PYTHIA (blue line), \Powheg + \HERWIG (orange line) and \MCNLO + \HERWIG (green line).}
\label{fig:XS_2D_pttt_ytt}
\end{figure}
\begin{figure}[p]
\centering
\begin{subfigure}
  \centering
  \includegraphics[width=0.49\textwidth]{/home/dolinska/Dropbox/desy_plots/Thesis/Jenya/xSec/xsecNorm/xSec_ttbar_arapidity_IN_ttbar_pt_0.pdf}
\end{subfigure}
\begin{subfigure}
  \centering
  \includegraphics[width=0.49\textwidth]{/home/dolinska/Dropbox/desy_plots/Thesis/Jenya/xSec/xsecNorm/xSec_ttbar_arapidity_IN_ttbar_pt_1.pdf}
\end{subfigure}
\begin{subfigure}
  \centering
  \includegraphics[width=0.49\textwidth]{/home/dolinska/Dropbox/desy_plots/Thesis/Jenya/xSec/xsecNorm/xSec_ttbar_arapidity_IN_ttbar_pt_2.pdf}
\end{subfigure}
\begin{subfigure}
  \centering
  \includegraphics[width=0.49\textwidth]{/home/dolinska/Dropbox/desy_plots/Thesis/Jenya/xSec/xsecNorm/xSec_ttbar_arapidity_IN_ttbar_pt_3.pdf}
\end{subfigure}
\caption{Normalized differential cross sections in bins of top pair transverse momentum and absolute rapidity. The inner error bands are the statistical uncertainties from the data.
         The outer error bars are the combines statistical and systematical uncertainties on the data. The cross sections predicted different models are also presented:
         \MG + \PYTHIA (red line), \Powheg + \PYTHIA (blue line), \Powheg + \HERWIG (orange line) and \MCNLO + \HERWIG (green line).}
\label{fig:XS_2D_pttt_ytt1}
\end{figure}

\begin{sidewaysfigure}[p]
\centering
\begin{subfigure}
  \centering
  \includegraphics[width=0.32\textwidth]{/home/dolinska/Dropbox/desy_plots/Thesis/Jenya/xSec/xsecNorm/xSec_ttbar_mass_IN_top_arapidity_0.pdf}
\end{subfigure}
\begin{subfigure}
  \centering
  \includegraphics[width=0.32\textwidth]{/home/dolinska/Dropbox/desy_plots/Thesis/Jenya/xSec/xsecNorm/xSec_ttbar_mass_IN_top_arapidity_1.pdf}
\end{subfigure}
\begin{subfigure}
  \centering
  \includegraphics[width=0.32\textwidth]{/home/dolinska/Dropbox/desy_plots/Thesis/Jenya/xSec/xsecNorm/xSec_ttbar_mass_IN_top_arapidity_2.pdf}
\end{subfigure}
\begin{subfigure}
  \centering
  \includegraphics[width=0.32\textwidth]{/home/dolinska/Dropbox/desy_plots/Thesis/Jenya/xSec/xsecNorm/xSec_top_arapidity_IN_ttbar_mass_0.pdf}
\end{subfigure}
\begin{subfigure}
  \centering
  \includegraphics[width=0.32\textwidth]{/home/dolinska/Dropbox/desy_plots/Thesis/Jenya/xSec/xsecNorm/xSec_top_arapidity_IN_ttbar_mass_1.pdf}
\end{subfigure}
\begin{subfigure}
  \centering
  \includegraphics[width=0.32\textwidth]{/home/dolinska/Dropbox/desy_plots/Thesis/Jenya/xSec/xsecNorm/xSec_top_arapidity_IN_ttbar_mass_2.pdf}
\end{subfigure}
\caption{Normalized differential cross sections in bins of $M(t\bar{t})$ and $|y(t)|$. The inner error bands are the statistical uncertainties from the data.
         The outer error bars are the combines statistical and systematical uncertainties on the data. The cross sections predicted different models are also presented:
         \MG + \PYTHIA (red line), \Powheg + \PYTHIA (blue line), \Powheg + \HERWIG (orange line) and \MCNLO + \HERWIG (green line).}
\label{fig:XS_2D_Mtt_yt}
\end{sidewaysfigure}

\begin{figure}[p]
\centering
\begin{subfigure}
  \centering
  \includegraphics[width=0.49\textwidth]{/home/dolinska/Dropbox/desy_plots/Thesis/Jenya/xSec/xsecNorm/xSec_top_pt_IN_ttbar_mass_0.pdf}
\end{subfigure}
\begin{subfigure}
  \centering
  \includegraphics[width=0.49\textwidth]{/home/dolinska/Dropbox/desy_plots/Thesis/Jenya/xSec/xsecNorm/xSec_top_pt_IN_ttbar_mass_1.pdf}
\end{subfigure}
\begin{subfigure}
  \centering
  \includegraphics[width=0.49\textwidth]{/home/dolinska/Dropbox/desy_plots/Thesis/Jenya/xSec/xsecNorm/xSec_top_pt_IN_ttbar_mass_2.pdf}
\end{subfigure}
\caption{Normalized differential cross sections in bins of $M(t\bar{t})$ and $p_{T}(t)$. The inner error bands are the statistical uncertainties from the data.
         The outer error bars are the combines statistical and systematical uncertainties on the data. The cross sections predicted different models are also presented:
         \MG + \PYTHIA (red line), \Powheg + \PYTHIA (blue line), \Powheg + \HERWIG (orange line) and \MCNLO + \HERWIG (green line).}
\label{fig:XS_2D_Mtt_pt}
\end{figure}
\begin{sidewaysfigure}[p]
\centering
\begin{subfigure}
  \centering
  \includegraphics[width=0.32\textwidth]{/home/dolinska/Dropbox/desy_plots/Thesis/Jenya/xSec/xsecNorm/xSec_ttbar_mass_IN_top_pt_0.pdf}
\end{subfigure}
\begin{subfigure}
  \centering
  \includegraphics[width=0.32\textwidth]{/home/dolinska/Dropbox/desy_plots/Thesis/Jenya/xSec/xsecNorm/xSec_ttbar_mass_IN_top_pt_1.pdf}
\end{subfigure}
\begin{subfigure}
  \centering
  \includegraphics[width=0.32\textwidth]{/home/dolinska/Dropbox/desy_plots/Thesis/Jenya/xSec/xsecNorm/xSec_ttbar_mass_IN_top_pt_2.pdf}
\end{subfigure}
\begin{subfigure}
  \centering
  \includegraphics[width=0.32\textwidth]{/home/dolinska/Dropbox/desy_plots/Thesis/Jenya/xSec/xsecNorm/xSec_ttbar_mass_IN_top_pt_3.pdf}
\end{subfigure}
\begin{subfigure}
  \centering
  \includegraphics[width=0.32\textwidth]{/home/dolinska/Dropbox/desy_plots/Thesis/Jenya/xSec/xsecNorm/xSec_ttbar_mass_IN_top_pt_4.pdf}
\end{subfigure}
\caption{Normalized differential cross sections in bins of $p_{T}(t)$ and $M(t\bar{t})$. The inner error bands are the statistical uncertainties from the data.
         The outer error bars are the combines statistical and systematical uncertainties on the data. The cross sections predicted different models are also presented:
         \MG + \PYTHIA (red line), \Powheg + \PYTHIA (blue line), \Powheg + \HERWIG (orange line) and \MCNLO + \HERWIG (green line).}
\label{fig:XS_2D_Mtt_pt1}
\end{sidewaysfigure}

\begin{sidewaysfigure}[p]
\centering
\begin{subfigure}
  \centering
  \includegraphics[width=0.32\textwidth]{/home/dolinska/Dropbox/desy_plots/Thesis/Jenya/xSec/xsecNorm/xSec_ttbar_delta_eta_IN_ttbar_mass_0.pdf}
\end{subfigure}
\begin{subfigure}
  \centering
  \includegraphics[width=0.32\textwidth]{/home/dolinska/Dropbox/desy_plots/Thesis/Jenya/xSec/xsecNorm/xSec_ttbar_delta_eta_IN_ttbar_mass_1.pdf}
\end{subfigure}
\begin{subfigure}
  \centering
  \includegraphics[width=0.32\textwidth]{/home/dolinska/Dropbox/desy_plots/Thesis/Jenya/xSec/xsecNorm/xSec_ttbar_delta_eta_IN_ttbar_mass_2.pdf}
\end{subfigure}
\begin{subfigure}
  \centering
  \includegraphics[width=0.32\textwidth]{/home/dolinska/Dropbox/desy_plots/Thesis/Jenya/xSec/xsecNorm/xSec_ttbar_mass_IN_ttbar_delta_eta_0.pdf}
\end{subfigure}
\begin{subfigure}
  \centering
  \includegraphics[width=0.32\textwidth]{/home/dolinska/Dropbox/desy_plots/Thesis/Jenya/xSec/xsecNorm/xSec_ttbar_mass_IN_ttbar_delta_eta_1.pdf}
\end{subfigure}
\begin{subfigure}
  \centering
  \includegraphics[width=0.32\textwidth]{/home/dolinska/Dropbox/desy_plots/Thesis/Jenya/xSec/xsecNorm/xSec_ttbar_mass_IN_ttbar_delta_eta_2.pdf}
\end{subfigure}
\caption{Normalized differential cross sections in bins of $M(t\bar{t})$ and $\Delta\eta(t\bar{t})$. The inner error bands are the statistical uncertainties from the data.
         The outer error bars are the combines statistical and systematical uncertainties on the data. The cross sections predicted different models are also presented:
         \MG + \PYTHIA (red line), \Powheg + \PYTHIA (blue line), \Powheg + \HERWIG (orange line) and \MCNLO + \HERWIG (green line).}
\label{fig:XS_2D_eta_Mtt}
\end{sidewaysfigure}

\begin{sidewaysfigure}[p]
\centering
\begin{subfigure}
  \centering
  \includegraphics[width=0.32\textwidth]{/home/dolinska/Dropbox/desy_plots/Thesis/Jenya/xSec/xsecNorm/xSec_ttbar_delta_phi_IN_ttbar_mass_0.pdf}
\end{subfigure}
\begin{subfigure}
  \centering
  \includegraphics[width=0.32\textwidth]{/home/dolinska/Dropbox/desy_plots/Thesis/Jenya/xSec/xsecNorm/xSec_ttbar_delta_phi_IN_ttbar_mass_1.pdf}
\end{subfigure}
\begin{subfigure}
  \centering
  \includegraphics[width=0.32\textwidth]{/home/dolinska/Dropbox/desy_plots/Thesis/Jenya/xSec/xsecNorm/xSec_ttbar_delta_phi_IN_ttbar_mass_2.pdf}
\end{subfigure}
\begin{subfigure}
  \centering
  \includegraphics[width=0.32\textwidth]{/home/dolinska/Dropbox/desy_plots/Thesis/Jenya/xSec/xsecNorm/xSec_ttbar_mass_IN_ttbar_delta_phi_0.pdf}
\end{subfigure}
\begin{subfigure}
  \centering
  \includegraphics[width=0.32\textwidth]{/home/dolinska/Dropbox/desy_plots/Thesis/Jenya/xSec/xsecNorm/xSec_ttbar_mass_IN_ttbar_delta_phi_1.pdf}
\end{subfigure}
\begin{subfigure}
  \centering
  \includegraphics[width=0.32\textwidth]{/home/dolinska/Dropbox/desy_plots/Thesis/Jenya/xSec/xsecNorm/xSec_ttbar_mass_IN_ttbar_delta_phi_2.pdf}
\end{subfigure}
\caption{Normalized differential cross sections in bins of $M(t\bar{t})$ and $\Delta\phi(t\bar{t})$. The inner error bands are the statistical uncertainties from the data.
         The outer error bars are the combines statistical and systematical uncertainties on the data. The cross sections predicted different models are also presented:
         \MG + \PYTHIA (red line), \Powheg + \PYTHIA (blue line), \Powheg + \HERWIG (orange line) and \MCNLO + \HERWIG (green line).}
\label{fig:XS_2D_phi_Mtt}
\end{sidewaysfigure}

\begin{sidewaysfigure}[p]
\centering
\begin{subfigure}
  \centering
  \includegraphics[width=0.32\textwidth]{/home/dolinska/Dropbox/desy_plots/Thesis/Jenya/xSec/xsecNorm/xSec_ttbar_arapidity_IN_ttbar_mass_0.pdf}
\end{subfigure}
\begin{subfigure}
  \centering
  \includegraphics[width=0.32\textwidth]{/home/dolinska/Dropbox/desy_plots/Thesis/Jenya/xSec/xsecNorm/xSec_ttbar_arapidity_IN_ttbar_mass_1.pdf}
\end{subfigure}
\begin{subfigure}
  \centering
  \includegraphics[width=0.32\textwidth]{/home/dolinska/Dropbox/desy_plots/Thesis/Jenya/xSec/xsecNorm/xSec_ttbar_arapidity_IN_ttbar_mass_2.pdf}
\end{subfigure}
\begin{subfigure}
  \centering
  \includegraphics[width=0.32\textwidth]{/home/dolinska/Dropbox/desy_plots/Thesis/Jenya/xSec/xsecNorm/xSec_ttbar_mass_IN_ttbar_arapidity_0.pdf}
\end{subfigure}
\begin{subfigure}
  \centering
  \includegraphics[width=0.32\textwidth]{/home/dolinska/Dropbox/desy_plots/Thesis/Jenya/xSec/xsecNorm/xSec_ttbar_mass_IN_ttbar_arapidity_1.pdf}
\end{subfigure}
\begin{subfigure}
  \centering
  \includegraphics[width=0.32\textwidth]{/home/dolinska/Dropbox/desy_plots/Thesis/Jenya/xSec/xsecNorm/xSec_ttbar_mass_IN_ttbar_arapidity_2.pdf}
\end{subfigure}
\caption{Normalized differential cross sections in bins of $M(t\bar{t})$ and $|y(t\bar{t})|$. The inner error bands are the statistical uncertainties from the data.
         The outer error bars are the combines statistical and systematical uncertainties on the data. The cross sections predicted different models are also presented:
         \MG + \PYTHIA (red line), \Powheg + \PYTHIA (blue line), \Powheg + \HERWIG (orange line) and \MCNLO + \HERWIG (green line).}
\label{fig:XS_2D_ytt_Mtt}
\end{sidewaysfigure}

\begin{figure}[p]
\centering
\begin{subfigure}
  \centering
  \includegraphics[width=0.49\textwidth]{/home/dolinska/Dropbox/desy_plots/Thesis/Jenya/xSec/xsecNorm/xSec_ttbar_pt_IN_ttbar_mass_0.pdf}
\end{subfigure}
\begin{subfigure}
  \centering
  \includegraphics[width=0.49\textwidth]{/home/dolinska/Dropbox/desy_plots/Thesis/Jenya/xSec/xsecNorm/xSec_ttbar_pt_IN_ttbar_mass_1.pdf}
\end{subfigure}
\begin{subfigure}
  \centering
  \includegraphics[width=0.49\textwidth]{/home/dolinska/Dropbox/desy_plots/Thesis/Jenya/xSec/xsecNorm/xSec_ttbar_pt_IN_ttbar_mass_2.pdf}
\end{subfigure}
\caption{Normalized differential cross sections in bins of $M(t\bar{t})$ and $p_{T}(t\bar{t})$. The inner error bands are the statistical uncertainties from the data.
         The outer error bars are the combines statistical and systematical uncertainties on the data. The cross sections predicted different models are also presented:
         \MG + \PYTHIA (red line), \Powheg + \PYTHIA (blue line), \Powheg + \HERWIG (orange line) and \MCNLO + \HERWIG (green line).}
\label{fig:XS_2D_Mtt_pttt}
\end{figure}
\begin{figure}[p]
\centering
\begin{subfigure}
  \centering
  \includegraphics[width=0.49\textwidth]{/home/dolinska/Dropbox/desy_plots/Thesis/Jenya/xSec/xsecNorm/xSec_ttbar_mass_IN_ttbar_pt_0.pdf}
\end{subfigure}
\begin{subfigure}
  \centering
  \includegraphics[width=0.49\textwidth]{/home/dolinska/Dropbox/desy_plots/Thesis/Jenya/xSec/xsecNorm/xSec_ttbar_mass_IN_ttbar_pt_1.pdf}
\end{subfigure}
\begin{subfigure}
  \centering
  \includegraphics[width=0.49\textwidth]{/home/dolinska/Dropbox/desy_plots/Thesis/Jenya/xSec/xsecNorm/xSec_ttbar_mass_IN_ttbar_pt_2.pdf}
\end{subfigure}
\begin{subfigure}
  \centering
  \includegraphics[width=0.49\textwidth]{/home/dolinska/Dropbox/desy_plots/Thesis/Jenya/xSec/xsecNorm/xSec_ttbar_mass_IN_ttbar_pt_3.pdf}
\end{subfigure}
\caption{Normalized differential cross sections in bins of $p_{T}(t\bar{t})$ and $M(t\bar{t})$. The inner error bands are the statistical uncertainties from the data.
         The outer error bars are the combines statistical and systematical uncertainties on the data. The cross sections predicted different models are also presented:
         \MG + \PYTHIA (red line), \Powheg + \PYTHIA (blue line), \Powheg + \HERWIG (orange line) and \MCNLO + \HERWIG (green line).}
\label{fig:XS_2D_Mtt_pttt1}
\end{figure}

\begin{figure}[p]
\centering
\begin{subfigure}
  \centering
  \includegraphics[width=0.49\textwidth]{/home/dolinska/Dropbox/desy_plots/Thesis/Jenya/xSec/xsecNorm/xSec_x1_IN_ttbar_mass_0.pdf}
\end{subfigure}
\begin{subfigure}
  \centering
  \includegraphics[width=0.49\textwidth]{/home/dolinska/Dropbox/desy_plots/Thesis/Jenya/xSec/xsecNorm/xSec_x1_IN_ttbar_mass_1.pdf}
\end{subfigure}
\begin{subfigure}
  \centering
  \includegraphics[width=0.49\textwidth]{/home/dolinska/Dropbox/desy_plots/Thesis/Jenya/xSec/xsecNorm/xSec_x1_IN_ttbar_mass_2.pdf}
\end{subfigure}
\caption{Normalized differential cross sections in bins of  $x_{1}$ and $M(t\bar{t})$. The inner error bands are the statistical uncertainties from the data.
         The outer error bars are the combines statistical and systematical uncertainties on the data. The cross sections predicted different models are also presented:
         \MG + \PYTHIA (red line), \Powheg + \PYTHIA (blue line), \Powheg + \HERWIG (orange line) and \MCNLO + \HERWIG (green line).}
\label{fig:XS_2D_x1_Mtt}
\end{figure}
\begin{sidewaysfigure}[p]
\centering
\begin{subfigure}
  \centering
  \includegraphics[width=0.32\textwidth]{/home/dolinska/Dropbox/desy_plots/Thesis/Jenya/xSec/xsecNorm/xSec_ttbar_mass_IN_x1_0.pdf}
\end{subfigure}
\begin{subfigure}
  \centering
  \includegraphics[width=0.32\textwidth]{/home/dolinska/Dropbox/desy_plots/Thesis/Jenya/xSec/xsecNorm/xSec_ttbar_mass_IN_x1_1.pdf}
\end{subfigure}
\begin{subfigure}
  \centering
  \includegraphics[width=0.32\textwidth]{/home/dolinska/Dropbox/desy_plots/Thesis/Jenya/xSec/xsecNorm/xSec_ttbar_mass_IN_x1_2.pdf}
\end{subfigure}
\begin{subfigure}
  \centering
  \includegraphics[width=0.32\textwidth]{/home/dolinska/Dropbox/desy_plots/Thesis/Jenya/xSec/xsecNorm/xSec_ttbar_mass_IN_x1_3.pdf}
\end{subfigure}
\begin{subfigure}
  \centering
  \includegraphics[width=0.32\textwidth]{/home/dolinska/Dropbox/desy_plots/Thesis/Jenya/xSec/xsecNorm/xSec_ttbar_mass_IN_x1_4.pdf}
\end{subfigure}
\caption{Normalized differential cross sections in bins of $M(t\bar{t})$ and  $x_{1}$. The inner error bands are the statistical uncertainties from the data.
         The outer error bars are the combines statistical and systematical uncertainties on the data. The cross sections predicted different models are also presented:
         \MG + \PYTHIA (red line), \Powheg + \PYTHIA (blue line), \Powheg + \HERWIG (orange line) and \MCNLO + \HERWIG (green line).}
\label{fig:XS_2D_x1_Mtt1}
\end{sidewaysfigure}

\begin{sidewaysfigure}[p]
\centering
\begin{subfigure}
  \centering
  \includegraphics[width=0.32\textwidth]{/home/dolinska/Dropbox/desy_plots/Thesis/Jenya/xSec/xsecNorm/xSec_ttbar_pt_IN_x1_0.pdf}
\end{subfigure}
\begin{subfigure}
  \centering
  \includegraphics[width=0.32\textwidth]{/home/dolinska/Dropbox/desy_plots/Thesis/Jenya/xSec/xsecNorm/xSec_ttbar_pt_IN_x1_1.pdf}
\end{subfigure}
\begin{subfigure}
  \centering
  \includegraphics[width=0.32\textwidth]{/home/dolinska/Dropbox/desy_plots/Thesis/Jenya/xSec/xsecNorm/xSec_ttbar_pt_IN_x1_2.pdf}
\end{subfigure}
\begin{subfigure}
  \centering
  \includegraphics[width=0.32\textwidth]{/home/dolinska/Dropbox/desy_plots/Thesis/Jenya/xSec/xsecNorm/xSec_ttbar_pt_IN_x1_3.pdf}
\end{subfigure}
\begin{subfigure}
  \centering
  \includegraphics[width=0.32\textwidth]{/home/dolinska/Dropbox/desy_plots/Thesis/Jenya/xSec/xsecNorm/xSec_ttbar_pt_IN_x1_4.pdf}
\end{subfigure}
\caption{Normalized differential cross sections in bins of  $x_{1}$ and $p_{T}(t\bar{t})$. The inner error bands are the statistical uncertainties from the data.
         The outer error bars are the combines statistical and systematical uncertainties on the data. The cross sections predicted different models are also presented:
         \MG + \PYTHIA (red line), \Powheg + \PYTHIA (blue line), \Powheg + \HERWIG (orange line) and \MCNLO + \HERWIG (green line).}
\label{fig:XS_2D_x1_pttt}
\end{sidewaysfigure}
\begin{figure}[p]
\centering
\begin{subfigure}
  \centering
  \includegraphics[width=0.49\textwidth]{/home/dolinska/Dropbox/desy_plots/Thesis/Jenya/xSec/xsecNorm/xSec_x1_IN_ttbar_pt_0.pdf}
\end{subfigure}
\begin{subfigure}
  \centering
  \includegraphics[width=0.49\textwidth]{/home/dolinska/Dropbox/desy_plots/Thesis/Jenya/xSec/xsecNorm/xSec_x1_IN_ttbar_pt_1.pdf}
\end{subfigure}
\begin{subfigure}
  \centering
  \includegraphics[width=0.49\textwidth]{/home/dolinska/Dropbox/desy_plots/Thesis/Jenya/xSec/xsecNorm/xSec_x1_IN_ttbar_pt_2.pdf}
\end{subfigure}
\begin{subfigure}
  \centering
  \includegraphics[width=0.49\textwidth]{/home/dolinska/Dropbox/desy_plots/Thesis/Jenya/xSec/xsecNorm/xSec_x1_IN_ttbar_pt_3.pdf}
\end{subfigure}
\caption{Normalized differential cross sections in bins of $p_{T}(t\bar{t})$ and  $x_{1}$. The inner error bands are the statistical uncertainties from the data.
         The outer error bars are the combines statistical and systematical uncertainties on the data. The cross sections predicted different models are also presented:
         \MG + \PYTHIA (red line), \Powheg + \PYTHIA (blue line), \Powheg + \HERWIG (orange line) and \MCNLO + \HERWIG (green line).}
\label{fig:XS_2D_x1_pttt1}
\end{figure}

%%%%%%%%%%%%%%%%%%%%%%%%%%%

%%%%%%%%%%%%%%%%%%%%%%%%%%%
%%%%%%%%%%%%%%%%%%%%%%%%%%%
%%%%%%%%%%%%%%%%%%%%%%%%%%%