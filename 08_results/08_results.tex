\chapter{Discussions}

In general, the cross sections measured and presented in this work are in general in good argeement with LO and NLO predictions
implemented in differential MC event generators (LO $\MG+\PYTHIA$ and NLO $\Powheg+\HERWIG$, $\Powheg+\PYTHIA$ and $\MCNLO+\HERWIG$).
However, there are some disagreements and trends observed in particular cross sections bins and control distributions.
Those should be discussed in more detail.

\section{Trend in the $\Delta\eta$ Between Top and Anti-top} \label{sec:deta_discuss}

If one throws a look onto the $t\bar{t}$ cross sections in bins of $\Delta\eta{t\bar{t}}$ in different bins of $t\bar{t}$ mass
(see fig.\ref{fig:XS_2D_eta_Mtt}), one observes the tendency that the $\Delta\eta{t\bar{t}}$ is not described in the higher mass bins.
The control distribution (see fig. \ref{fig:CP_2D_eta_Mtt}) shows the trend, that the higher the $M(t\bar{t})$ is, the smaller $\Delta\eta(t\bar{t})$
is modeled in $\MG+\PYTHIA$. That means that the top quarks, reconstructed from the experimental data, are more often back-to-back 
than for the MC.

The reason for the discrepancies in the pseudorapidities between two top-quarks may originate because of two reasons:

\begin{itemize}
 \item \textit{Wrong PDFs in the MC}: If the $x_{1}$ and $x_{2}$ transmitted from partons to first and second top quark are very different, than these
 top quarks will fly not back-to-back. This scenario, however, is unlikely, as the effect of underestimation of $\Delta\eta(t\bar{t})$
 by MC is observed stronger if the mass of the $t\bar{t}$ system is larger. For large $M(t\bar{t})$ the transmitted momenta from the partons
 to the top quarks are equivalently large.
 
 \item \textit{Additional radiation}: The presence of radiation may deviate the direction of the top quarks so that they will no longer be back-to-back.
\end{itemize}

One might conclude that MC models more radiation on higher $M(t\bar{t})$. This can be checked by looking at the $\Delta\phi(t\bar{t})$ in this region.
$\Delta\phi(t\bar{t})$ is not sensitive to the differences in $x_{1}$ and $x_{2}$ as there is no transverse component in proton momentum.
This control distribution is presented in fig. \ref{fig:CP_phi_eta}. It shows that there is a trend that there are more MC the smaller $\Delta\eta(t\bar{t})$.
In addition, there is a trend that there are more MC for the smaller $\Delta\phi(t\bar{t})$. This means, that there is more radiation modeled
in MC out of which there is more hard radiation (where $\Delta\phi(t\bar{t})$ is smaller).

\begin{figure}[t]
  \centering
  \includegraphics[width=0.8\textwidth]{09_conclusions/plots/CP_phi_eta.png}
  \caption{Control distribution of $\Delta\phi{t\bar{t}}$ in bins of $\Delta\eta(t\bar{t})$ for $M(t\bar{t})$ > 600 GeV.}
  \label{fig:CP_phi_eta}
\end{figure}

After comparing the results of the model predictions available in this analysis (see fig.\ref{fig:XS_2D_eta_Mtt}), one can conclude that this effect is the
strongest for $\MG+\PYTHIA$.