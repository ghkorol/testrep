\chapter{Results and Discussion}\label{chapt:results}

\section{Normalized Double Differential $t\bar{t}$ Production Cross Sections}\label{ssec:xsec_mes}

The normalized $t\bar{t}$ production cross sections were measured double differentially in bins of top transverse momentum, rapidity and  $x_{1}$,
which corresponds to the parton momentum transfered to the $t$-quark,
$t\bar{t}$ mass, rapidity and transverse momentum and the azimuthal and pseudorapidity difference between the $t$ and the $\bar{t}$.
The binning for the detector level distributions was chosen to have enough statistics in each bin so that the statistics could be treated as Gaussian.
It was also checked if the purity, stability and efficiency in each bin are not too low (see the explanation below).
The regularization strength used to determine each set of the cross sections is listed in Appendix \ref{appendix:tau}.

This sections is presenting only the normalized double differential $t\bar{t}$ production cross sections. The corresponding
unnormalized cross sections are shown in Appendix \ref{appendix:unnorm_XSec}. All numerical values for the cross sections
and their uncertainties are listed in Appendix \ref{appendix:xsec_table}.

%%%
\subsubsection{Cross sections in bins of $|y(t)|$ versus $p_{T}(t)$}

The $t\bar{t}$ production cross sections in bins of the rapidity and the transverse momentum of the top quark was
measured in the bins presented in Fig. \ref{fig:CP_2D_y_pt}. The shown control distribution is also demonstrating the agreement between 
the data and the MC estimated $t\bar{t}$ signal and background contributions. The MC slightly underestimates the data for 
the lower $p_{T}(t)$ bins and in the outer $y(t)$ bins for all the transverse momenta values. There is a trend that the $p_{T}$
spectrum for data is softer than for the MC $t\bar{t}$ simulation.

Fig. \ref{fig:XS_2D_y_pt} represents the production cross sections of the $t\bar{t}$ pair in bins of top rapidity and top transverse momentum.
The experimentally measured cross sections are compared to the \MG+\PYTHIA, \Powheg+\PYTHIA, \Powheg+\HERWIG and \MCNLO+\HERWIG predictions.
An overall good agreement between theory predictions and experimental results is observed. However, all the simulation models tend to have 
harder transverse momenta than data. This disagreement is the strongest between the data and \MG+\PYTHIA model and is more pronounced in the 
central rapidity bins. The best description is provided by $\Powheg+\HERWIG$.

%%%

\subsubsection{Cross sections in bins of $p_{T}(t\bar{t})$ versus $|y(t\bar{t})|$}

Another pair of variables in bins of which the cross section was measured is the $p_{T}$ and $|y|$ of the $t\bar{t}$ system. 
The control distribution in bins of the $p_{T}(t\bar{t})$ and $|y(t\bar{t})|$ is presented in the Fig. \ref{fig:CP_2D_pttt_ytt}. 
The agreement between data and MC is overall nice, except for the highest $p_{T}(t\bar{t})$ bin. The MC slightly underestimates 
the data for the first three $p_{T}(t\bar{t})$ bins, while for the highest measured $p_{T}(t\bar{t})$ bin the MC overestimates the data. 

The production cross sections in bins of $p_{T}(t\bar{t})$ and $|y(t\bar{t})|$ are shown in Fig. \ref{fig:XS_2D_pttt_ytt}.
These plots show that the models cross sections are a bit less central in $|y(t\bar{t})|$ than the data except in the 
high $p_{T}(t\bar{t})$ region when the models generally overestimate the cross sections. Also, in the high $p_{T}(t\bar{t})$
region the difference between different models is the largest. One can observe that the $p_{T}(t\bar{t})$ spectrum simulated
in $\Powheg+\PYTHIA$ is a bit steeper than in the other models.

%%%

\subsubsection{Cross sections in bins of $M(t\bar{t})$ versus $|y(t)|$}

Another measurement has been performed in bins of $M(t\bar{t})$ and $|y(t)|$. Fig. \ref{fig:CP_2D_Mtt_y} represents the control distribution in bins of these variables.
The agreement between data and MC prediction is good in the lower bins of the invariant mass of the $t\bar{t}$ pair. 
However, the MC starts to underestimate data for the highest $M(t\bar{t})$. In general, the MC is lower for the outer rapidity 
bin. 

The cross sections measured in bins of $M(t\bar{t})$ and $|y(t)|$ are presented in fig. \ref{fig:XS_2D_Mtt_yt}. The $\MCNLO+\HERWIG$ 
predictions have the worst agreement with data in the smallest rapidity bin, where the model exhibits a too a hard $M(t\bar{t})$ spectrum.
In general, the other MC models (except $\MG + \PYTHIA$) show a bit harder $M(t\bar{t})$ spectrum than the data.

%%%

\subsubsection{Cross sections in bins of $M(t\bar{t})$ versus $p_{T}(t)$}

The $t\bar{t}$ production cross section was also measured double differentially in bins of $p_{T}(t)$ and $M(t\bar{t})$.
The control plot, which shows the binning and the comparison between data and simulation, is presented in Fig. \ref{fig:CP_2D_Mtt_pt}. 
The MC shows harder $p_{T}(t)$ spectra than data and this discrepancy increases with $M(t\bar{t})$.

The cross sections in bins of $M(t\bar{t})$ and $p_{T}(t)$ are presented in Fig. \ref{fig:XS_2D_Mtt_pt}. $\MG+\PYTHIA$ predicts harder
$p_{T}(t)$ spectra and this effect is much enhanced at high $M(t\bar{t})$. All the other models describe the data a bit better, but 
also predict a too hard $p_{T}(t)$ spectrum, in particular at the highest $M(t\bar{t})$.

%%%
\subsubsection{Cross section in bins of $M(t\bar{t})$ versus $\Delta\eta(t\bar{t})$}

The cross section has been measured double differentially in bins of $\Delta\eta(t\bar{t})$ and $M(t\bar{t})$, where $\Delta\eta(t\bar{t}) = \eta(t) - \eta(\bar{t})$
denotes the difference in pseudorapidity between the top and the antitop.

The control distribution in Fig. \ref{fig:CP_2D_eta_Mtt} shows that the simulation slightly overestimates the experimental data in the lowest $M(t\bar{t})$
bin. However, there is a strong disagreement between MC and data in the $\Delta\eta(t\bar{t})$ spectra for the two higher $M(t\bar{t})$ bins. The MC predicts 
a too small pseudorapidity separation between $t$ and $\bar{t}$ for high $M(t\bar{t})$.

The double differential production cross sections in bins of $\Delta\eta(t\bar{t})$ and $M(t\bar{t})$ is presented in fig. \ref{fig:XS_2D_eta_Mtt}. The \MG + \PYTHIA
prediction shows the worst agreement. There is a tendency that the higher the $M(t\bar{t})$ is, the more too small $\Delta\eta(t\bar{t})$ values are predicted by
the models.

%%%
\subsubsection{Cross section in bins of $M(t\bar{t})$ versus $\Delta\phi(t\bar{t})$}

The measurement of the cross section has been also performed in bins of the azimuthal angle between the top and the antitop, $\Delta\phi(t\bar{t})$, and the mass
of the $t\bar{t}$ system, $M(t\bar{t})$.

The control distribution in bins of these variable pair is presented in Fig. \ref{fig:CP_2D_phi_Mtt}. The MC is a bit more back-to-back than the data for the
two highest $M(t\bar{t})$ bins, while the lowest $M(t\bar{t})$ bin has a good description of the data by the MC model.

The Fig. \ref{fig:XS_2D_phi_Mtt} presents the double differential production cross sections of the $t\bar{t}$ pairs in bins of $\Delta\phi(t\bar{t})$ and $M(t\bar{t})$.
All the predictions provide a reasonable description of the measured cross sections. However, the agreement is getting slightly worse for the higher bins of the $M(t\bar{t})$.
$\MG+\PYTHIA$ provides the worst description of the data.

%%%
\subsubsection{Cross section in bins of $|y(t\bar{t})|$ versus $M(t\bar{t})$}

The control distribution in bins of $|y(t\bar{t})|$ and $M(t\bar{t})$ is shown in fig. \ref{fig:CP_2D_ytt_Mtt}. The agreement between MC and data is overall nice.
For higher masses the MC tends to a bit less central rapidity.

The normalized double differential $t\bar{t}$ production cross sections in bins of $M(t\bar{t})$ and $|y(t\bar{t})|$ are presented in fig. \ref{fig:XS_2D_ytt_Mtt}.
$\MG+\PYTHIA$ provides the best agreement with the data. 
The other theoretical predictions tend to be less central in $y(t\bar{t})$ then the data. 
The description at the highest $M(t\bar{t})$ bins is the worst.

%%%
\subsubsection{Cross section in bins of $|p_{T}(t\bar{t})|$ versus $M(t\bar{t})$}

Another measurement of the cross sections was performed in bins of $|p_{T}(t\bar{t})|$ and $M(t\bar{t})$.
The control distribution in bins of $|p_{T}(t\bar{t})|$ and $M(t\bar{t})$ is presented in fig. \ref{fig:CP_2D_pttt_Mtt}. In all of the $M(t\bar{t})$
bins there is the same trend in the way how the simulation describes the experimentally measured $p_{T}(t\bar{t})$ spectrum. The MC slightly underestimates
the data for the lower $p_{T}(t\bar{t})$, while for the highest transverse momentum of the top-pair a significant overestimation is observed.

Fig. \ref{fig:XS_2D_Mtt_pttt} shows the double differential $t\bar{t}$ production cross section in bins of $|p_{T}(t\bar{t})|$ and $M(t\bar{t})$.
All the theoretical predictions describe the measured cross sections well. However, the predictions are higher than data at higher $p_{T}(t\bar{t})$.

%%%
% \subsubsection{Cross section in bins of $p_{T}(t\bar{t})$ versus $x_{1}$}
% 
% The $t\bar{t}$ cross section was measured in bins of $p_{T}(t\bar{t})$ and  $x_{1}$.
% The corresponding control plot is shown in fig. \ref{fig:CP_2D_pttt_x1}. The agreement between simulation and data is overall nice, except for the highest
% $x_{1}$ bin for all of the $p_{T}(t\bar{t})$, where MC overestimates data.
% 
% Fig. \ref{fig:XS_2D_x1_pttt} and fig. \ref{fig:XS_2D_x1_pttt1} show the double differential $t\bar{t}$ production cross sections in bins of $p_{T}(t\bar{t})$ and $x_{1}$.
% The predictions overshoot the data in the highest bin of $p_{T}(t\bar{t})$ and the highest bin in $x_{1}$ has the worst agreement between measured and theoretical
% cross sections.

%%%
\subsubsection{Cross section in bins of $M(t\bar{t})$ versus $x_{1}$}

The control distributions of $x_{1}$ in bins of $M(t\bar{t})$ are shown in Fig. \ref{fig:CP_2D_Mtt_x1}. A discrepancy in the medium $x_{1}$ bins for the 
high $M(t\bar{t})$ are observed. The MC is a bit lower than data in this region.

The double differential $t\bar{t}$ production cross sections are shown in Fig. \ref{fig:XS_2D_x1_Mtt}. In general a reasonable description of data by MC
models is observed. In the two highest $M(t\bar{t})$ bins the predictions tend to be a bit higher than data at the lowest $x_{1}$ values.


%%%%%%%%%%%%%%%%%%%%%%%%%%%
%%%%%%%Plots%%%%%%%%%%%%%%%
%%%%%%%%%%%%%%%%%%%%%%%%%%%

%%%%%Control Plots%%%%%%%%%

\begin{figure}[H]
  \centering
  \includegraphics[width=1.0\textwidth]{/home/dolinska/Dropbox/desy_plots/Thesis/Jenya/xSec/CP/CP_AllBins_top_arapidity_vs_top_pt.pdf}
  \caption{Control distribution of the $y$ of the top quark in bins of the $p_{T}$ of the top quark. The $|y|$ bins are shown on the top 
  of the plot. The experimental data are marked with the black dots and the reconstructed MC signal is marked with the red area. The error
  bars on the data points represent the statistical uncertainty only. The 
  different background contributions are also shown. On the bottom part of the plot the ratio between MC and data statistics in each bin
  is presented. The error bars represent the data statistical uncertainties only.}
  \label{fig:CP_2D_y_pt}
\end{figure}

\begin{figure}[H]
  \centering
  \includegraphics[width=1.0\textwidth]{/home/dolinska/Dropbox/desy_plots/Thesis/Jenya/xSec/CP/CP_AllBins_ttbar_arapidity_vs_ttbar_pt.pdf}
  \caption{Control distribution of the $p_{T}(t\bar{t})$ in bins of the $|y(t\bar{t})|$. The $|y(t\bar{t})|$ bins are shown on the top 
  of the plot. Other details as in Fig. \ref{fig:CP_2D_y_pt}.}
  \label{fig:CP_2D_pttt_ytt}
\end{figure}

\begin{figure}[H]
  \centering
  \includegraphics[width=1.0\textwidth]{/home/dolinska/Dropbox/desy_plots/Thesis/Jenya/xSec/CP/CP_AllBins_top_arapidity_vs_ttbar_mass.pdf}
  \caption{Control distribution of the $M(t\bar{t})$ in bins of the $|y(t)|$. The $|y(t)|$ bins are shown on the top 
  of the plot. Other details as in Fig. \ref{fig:CP_2D_y_pt}.}
  \label{fig:CP_2D_Mtt_y}
\end{figure}

\begin{figure}[H]
  \centering
  \includegraphics[width=1.0\textwidth]{/home/dolinska/Dropbox/desy_plots/Thesis/Jenya/xSec/CP/CP_AllBins_top_pt_vs_ttbar_mass.pdf}
  \caption{Control distribution of the $M(t\bar{t})$ in bins of the $p_{T}$ of the top quark. The $p_{T}(t)$ bins are shown on the top 
  of the plot. Other details as in Fig. \ref{fig:CP_2D_y_pt}.}
  \label{fig:CP_2D_Mtt_pt}
\end{figure}

\begin{figure}[H]
  \centering
  \includegraphics[width=1.0\textwidth]{/home/dolinska/Dropbox/desy_plots/Thesis/Jenya/xSec/CP/CP_AllBins_ttbar_delta_eta_vs_ttbar_mass.pdf}
  \caption{Control distribution of the $\Delta\eta$ between the $t$ and $\bar{t}$ in bins of the $M(t\bar{t})$. The $\Delta\eta$ bins are shown on the top 
  of the plot. Other details as in Fig. \ref{fig:CP_2D_y_pt}.}
  \label{fig:CP_2D_eta_Mtt}
\end{figure}

\begin{figure}[H]
  \centering
  \includegraphics[width=1.0\textwidth]{/home/dolinska/Dropbox/desy_plots/Thesis/Jenya/xSec/CP/CP_AllBins_ttbar_delta_phi_vs_ttbar_mass.pdf}
  \caption{Control distribution of the $\Delta\phi$ between the $t$ and $\bar{t}$ in bins of the $M(t\bar{t})$. The $\Delta\phi$ bins are shown on the top 
  of the plot. Other details as in Fig. \ref{fig:CP_2D_y_pt}.}
  \label{fig:CP_2D_phi_Mtt}
\end{figure}

\begin{figure}[H]
  \centering
  \includegraphics[width=1.0\textwidth]{/home/dolinska/Dropbox/desy_plots/Thesis/Jenya/xSec/CP/CP_AllBins_ttbar_arapidity_vs_ttbar_mass.pdf}
  \caption{Control distribution of the $|y(t\bar{t})|$ in bins of the $M(t\bar{t})$. The $|y(t\bar{t})|$ bins are shown on the top 
  of the plot. Other details as in Fig. \ref{fig:CP_2D_y_pt}.}
  \label{fig:CP_2D_ytt_Mtt}
\end{figure}

\begin{figure}[H]
  \centering
  \includegraphics[width=1.0\textwidth]{/home/dolinska/Dropbox/desy_plots/Thesis/Jenya/xSec/CP/CP_AllBins_ttbar_pt_vs_ttbar_mass.pdf}
  \caption{Control distribution of the $|p_{T}(t\bar{t})|$ in bins of the $M(t\bar{t})$. The $|p_{T}(t\bar{t})|$ bins are shown on the top 
  of the plot. Other details as in Fig. \ref{fig:CP_2D_y_pt}.}
  \label{fig:CP_2D_pttt_Mtt}
\end{figure}

% \begin{figure}[H]
%   \centering
%   \includegraphics[width=1.0\textwidth]{/home/dolinska/Dropbox/desy_plots/Thesis/Jenya/xSec/CP/CP_AllBins_ttbar_pt_vs_x1.pdf}
%   \caption{Control distribution of the $p_{T}(t\bar{t})$ in bins of $x_{1}$. The $p_{T}(t\bar{t})$ bins are shown on the top 
%   of the plot. The experimental data are marked with the black dots and the reconstructed MC signal is marked with the red area. The error
%   bars on the data points represent the statistical uncertainty only. The 
%   different background contributions are also shown. On the bottom part of the plot the ratio between MC and data statistics in each bin
%   is presented.}
%   \label{fig:CP_2D_pttt_x1}
% \end{figure}

\begin{figure}[H]
  \centering
  \includegraphics[width=1.0\textwidth]{/home/dolinska/Dropbox/desy_plots/Thesis/Jenya/xSec/CP/CP_AllBins_x1_vs_ttbar_mass.pdf}
  \caption{Control distribution of $x_{1}$ in bins of $M(t\bar{t})$. The $x_{1}$ bins are shown on the top 
  of the plot. Other details as in Fig. \ref{fig:CP_2D_y_pt}.}
  \label{fig:CP_2D_Mtt_x1}
\end{figure}

%%%%%%%%%%%%%%%%%%%%%%%%%%%
%%%%%%%XSec%%%%%%%%%%%%%%%%

\begin{figure}[H]
\centering
\begin{subfigure}
  \centering
  \includegraphics[width=0.49\textwidth]{/home/dolinska/Dropbox/desy_plots/Thesis/Jenya/xSec/xsecNorm/xSec_top_pt_IN_top_arapidity_0.pdf}
\end{subfigure}
\begin{subfigure}
  \centering
  \includegraphics[width=0.49\textwidth]{/home/dolinska/Dropbox/desy_plots/Thesis/Jenya/xSec/xsecNorm/xSec_top_pt_IN_top_arapidity_1.pdf}
\end{subfigure}
\begin{subfigure}
  \centering
  \includegraphics[width=0.49\textwidth]{/home/dolinska/Dropbox/desy_plots/Thesis/Jenya/xSec/xsecNorm/xSec_top_pt_IN_top_arapidity_2.pdf}
\end{subfigure}
\caption{Normalized differential cross sections in bins of top absolute rapidity and transverse momentum. The inner error bands are the statistical uncertainties from the data.
         The outer error bars are the combines statistical and systematical uncertainties on the data. The cross sections predicted different models are also presented:
         \MG + \PYTHIA (red line), \Powheg + \PYTHIA (blue line), \Powheg + \HERWIG (orange line) and \MCNLO + \HERWIG (green line).}
\label{fig:XS_2D_y_pt}
\end{figure}
% \begin{sidewaysfigure}[H]
% \centering
% \begin{subfigure}
%   \centering
%   \includegraphics[width=0.325\textwidth]{/home/dolinska/Dropbox/desy_plots/Thesis/Jenya/xSec/xsecNorm/xSec_top_arapidity_IN_top_pt_0.pdf}
% \end{subfigure}
% \begin{subfigure}
%   \centering
%   \includegraphics[width=0.325\textwidth]{/home/dolinska/Dropbox/desy_plots/Thesis/Jenya/xSec/xsecNorm/xSec_top_arapidity_IN_top_pt_1.pdf}
% \end{subfigure}
% \begin{subfigure}
%   \centering
%   \includegraphics[width=0.325\textwidth]{/home/dolinska/Dropbox/desy_plots/Thesis/Jenya/xSec/xsecNorm/xSec_top_arapidity_IN_top_pt_2.pdf}
% \end{subfigure}
% \begin{subfigure}
%   \centering
%   \includegraphics[width=0.325\textwidth]{/home/dolinska/Dropbox/desy_plots/Thesis/Jenya/xSec/xsecNorm/xSec_top_arapidity_IN_top_pt_3.pdf}
% \end{subfigure}
% \begin{subfigure}
%   \centering
%   \includegraphics[width=0.325\textwidth]{/home/dolinska/Dropbox/desy_plots/Thesis/Jenya/xSec/xsecNorm/xSec_top_arapidity_IN_top_pt_4.pdf}
% \end{subfigure}
% \caption{Normalized differential cross sections in bins of top transverse momentum and absolute rapidity. The inner error bands are the statistical uncertainties from the data.
%          The outer error bars are the combines statistical and systematical uncertainties on the data. The cross sections predicted different models are also presented:
%          \MG + \PYTHIA (red line), \Powheg + \PYTHIA (blue line), \Powheg + \HERWIG (orange line) and \MCNLO + \HERWIG (green line).}
% \label{fig:XS_2D_y_pt1}
% \end{sidewaysfigure}

\begin{figure}[H]
\centering
\begin{subfigure}
  \centering
  \includegraphics[width=0.49\textwidth]{/home/dolinska/Dropbox/desy_plots/Thesis/Jenya/xSec/xsecNorm/xSec_ttbar_pt_IN_ttbar_arapidity_0.pdf}
\end{subfigure}
\begin{subfigure}
  \centering
  \includegraphics[width=0.49\textwidth]{/home/dolinska/Dropbox/desy_plots/Thesis/Jenya/xSec/xsecNorm/xSec_ttbar_pt_IN_ttbar_arapidity_1.pdf}
\end{subfigure}
\begin{subfigure}
  \centering
  \includegraphics[width=0.49\textwidth]{/home/dolinska/Dropbox/desy_plots/Thesis/Jenya/xSec/xsecNorm/xSec_ttbar_pt_IN_ttbar_arapidity_2.pdf}
\end{subfigure}
\caption{Normalized differential cross sections in bins of top pair absolute rapidity and transverse momentum. Other details as in Fig. \ref{fig:XS_2D_y_pt}.}
\label{fig:XS_2D_pttt_ytt}
\end{figure}
% \begin{figure}[H]
% \centering
% \begin{subfigure}
%   \centering
%   \includegraphics[width=0.49\textwidth]{/home/dolinska/Dropbox/desy_plots/Thesis/Jenya/xSec/xsecNorm/xSec_ttbar_arapidity_IN_ttbar_pt_0.pdf}
% \end{subfigure}
% \begin{subfigure}
%   \centering
%   \includegraphics[width=0.49\textwidth]{/home/dolinska/Dropbox/desy_plots/Thesis/Jenya/xSec/xsecNorm/xSec_ttbar_arapidity_IN_ttbar_pt_1.pdf}
% \end{subfigure}
% \begin{subfigure}
%   \centering
%   \includegraphics[width=0.49\textwidth]{/home/dolinska/Dropbox/desy_plots/Thesis/Jenya/xSec/xsecNorm/xSec_ttbar_arapidity_IN_ttbar_pt_2.pdf}
% \end{subfigure}
% \begin{subfigure}
%   \centering
%   \includegraphics[width=0.49\textwidth]{/home/dolinska/Dropbox/desy_plots/Thesis/Jenya/xSec/xsecNorm/xSec_ttbar_arapidity_IN_ttbar_pt_3.pdf}
% \end{subfigure}
% \caption{Normalized differential cross sections in bins of top pair transverse momentum and absolute rapidity. The inner error bands are the statistical uncertainties from the data.
%          The outer error bars are the combines statistical and systematical uncertainties on the data. The cross sections predicted different models are also presented:
%          \MG + \PYTHIA (red line), \Powheg + \PYTHIA (blue line), \Powheg + \HERWIG (orange line) and \MCNLO + \HERWIG (green line).}
% \label{fig:XS_2D_pttt_ytt1}
% \end{figure}

\begin{figure}[H]
\centering
\begin{subfigure}
  \centering
  \includegraphics[width=0.49\textwidth]{/home/dolinska/Dropbox/desy_plots/Thesis/Jenya/xSec/xsecNorm/xSec_ttbar_mass_IN_top_arapidity_0.pdf}
\end{subfigure}
\begin{subfigure}
  \centering
  \includegraphics[width=0.49\textwidth]{/home/dolinska/Dropbox/desy_plots/Thesis/Jenya/xSec/xsecNorm/xSec_ttbar_mass_IN_top_arapidity_1.pdf}
\end{subfigure}
\begin{subfigure}
  \centering
  \includegraphics[width=0.49\textwidth]{/home/dolinska/Dropbox/desy_plots/Thesis/Jenya/xSec/xsecNorm/xSec_ttbar_mass_IN_top_arapidity_2.pdf}
\end{subfigure}
% \begin{subfigure}
%   \centering
%   \includegraphics[width=0.32\textwidth]{/home/dolinska/Dropbox/desy_plots/Thesis/Jenya/xSec/xsecNorm/xSec_top_arapidity_IN_ttbar_mass_0.pdf}
% \end{subfigure}
% \begin{subfigure}
%   \centering
%   \includegraphics[width=0.32\textwidth]{/home/dolinska/Dropbox/desy_plots/Thesis/Jenya/xSec/xsecNorm/xSec_top_arapidity_IN_ttbar_mass_1.pdf}
% \end{subfigure}
% \begin{subfigure}
%   \centering
%   \includegraphics[width=0.32\textwidth]{/home/dolinska/Dropbox/desy_plots/Thesis/Jenya/xSec/xsecNorm/xSec_top_arapidity_IN_ttbar_mass_2.pdf}
% \end{subfigure}
\caption{Normalized differential cross sections in bins of $M(t\bar{t})$ and $|y(t)|$. Other details as in Fig. \ref{fig:XS_2D_y_pt}.}
\label{fig:XS_2D_Mtt_yt}
\end{figure}

\begin{figure}[H]
\centering
\begin{subfigure}
  \centering
  \includegraphics[width=0.49\textwidth]{/home/dolinska/Dropbox/desy_plots/Thesis/Jenya/xSec/xsecNorm/xSec_top_pt_IN_ttbar_mass_0.pdf}
\end{subfigure}
\begin{subfigure}
  \centering
  \includegraphics[width=0.49\textwidth]{/home/dolinska/Dropbox/desy_plots/Thesis/Jenya/xSec/xsecNorm/xSec_top_pt_IN_ttbar_mass_1.pdf}
\end{subfigure}
\begin{subfigure}
  \centering
  \includegraphics[width=0.49\textwidth]{/home/dolinska/Dropbox/desy_plots/Thesis/Jenya/xSec/xsecNorm/xSec_top_pt_IN_ttbar_mass_2.pdf}
\end{subfigure}
\caption{Normalized differential cross sections in bins of $M(t\bar{t})$ and $p_{T}(t)$. Other details as in Fig. \ref{fig:XS_2D_y_pt}.}
\label{fig:XS_2D_Mtt_pt}
\end{figure}
% \begin{sidewaysfigure}[H]
% \centering
% \begin{subfigure}
%   \centering
%   \includegraphics[width=0.32\textwidth]{/home/dolinska/Dropbox/desy_plots/Thesis/Jenya/xSec/xsecNorm/xSec_ttbar_mass_IN_top_pt_0.pdf}
% \end{subfigure}
% \begin{subfigure}
%   \centering
%   \includegraphics[width=0.32\textwidth]{/home/dolinska/Dropbox/desy_plots/Thesis/Jenya/xSec/xsecNorm/xSec_ttbar_mass_IN_top_pt_1.pdf}
% \end{subfigure}
% \begin{subfigure}
%   \centering
%   \includegraphics[width=0.32\textwidth]{/home/dolinska/Dropbox/desy_plots/Thesis/Jenya/xSec/xsecNorm/xSec_ttbar_mass_IN_top_pt_2.pdf}
% \end{subfigure}
% \begin{subfigure}
%   \centering
%   \includegraphics[width=0.32\textwidth]{/home/dolinska/Dropbox/desy_plots/Thesis/Jenya/xSec/xsecNorm/xSec_ttbar_mass_IN_top_pt_3.pdf}
% \end{subfigure}
% \begin{subfigure}
%   \centering
%   \includegraphics[width=0.32\textwidth]{/home/dolinska/Dropbox/desy_plots/Thesis/Jenya/xSec/xsecNorm/xSec_ttbar_mass_IN_top_pt_4.pdf}
% \end{subfigure}
% \caption{Normalized differential cross sections in bins of $p_{T}(t)$ and $M(t\bar{t})$. The inner error bands are the statistical uncertainties from the data.
%          The outer error bars are the combines statistical and systematical uncertainties on the data. The cross sections predicted different models are also presented:
%          \MG + \PYTHIA (red line), \Powheg + \PYTHIA (blue line), \Powheg + \HERWIG (orange line) and \MCNLO + \HERWIG (green line).}
% \label{fig:XS_2D_Mtt_pt1}
% \end{sidewaysfigure}

\begin{figure}[H]
\centering
\begin{subfigure}
  \centering
  \includegraphics[width=0.49\textwidth]{/home/dolinska/Dropbox/desy_plots/Thesis/Jenya/xSec/xsecNorm/xSec_ttbar_delta_eta_IN_ttbar_mass_0.pdf}
\end{subfigure}
\begin{subfigure}
  \centering
  \includegraphics[width=0.49\textwidth]{/home/dolinska/Dropbox/desy_plots/Thesis/Jenya/xSec/xsecNorm/xSec_ttbar_delta_eta_IN_ttbar_mass_1.pdf}
\end{subfigure}
\begin{subfigure}
  \centering
  \includegraphics[width=0.49\textwidth]{/home/dolinska/Dropbox/desy_plots/Thesis/Jenya/xSec/xsecNorm/xSec_ttbar_delta_eta_IN_ttbar_mass_2.pdf}
\end{subfigure}
% \begin{subfigure}
%   \centering
%   \includegraphics[width=0.32\textwidth]{/home/dolinska/Dropbox/desy_plots/Thesis/Jenya/xSec/xsecNorm/xSec_ttbar_mass_IN_ttbar_delta_eta_0.pdf}
% \end{subfigure}
% \begin{subfigure}
%   \centering
%   \includegraphics[width=0.32\textwidth]{/home/dolinska/Dropbox/desy_plots/Thesis/Jenya/xSec/xsecNorm/xSec_ttbar_mass_IN_ttbar_delta_eta_1.pdf}
% \end{subfigure}
% \begin{subfigure}
%   \centering
%   \includegraphics[width=0.32\textwidth]{/home/dolinska/Dropbox/desy_plots/Thesis/Jenya/xSec/xsecNorm/xSec_ttbar_mass_IN_ttbar_delta_eta_2.pdf}
% \end{subfigure}
\caption{Normalized differential cross sections in bins of $M(t\bar{t})$ and $\Delta\eta(t\bar{t})$. Other details as in Fig. \ref{fig:XS_2D_y_pt}.}
\label{fig:XS_2D_eta_Mtt}
\end{figure}

\begin{figure}[H]
\centering
\begin{subfigure}
  \centering
  \includegraphics[width=0.49\textwidth]{/home/dolinska/Dropbox/desy_plots/Thesis/Jenya/xSec/xsecNorm/xSec_ttbar_delta_phi_IN_ttbar_mass_0.pdf}
\end{subfigure}
\begin{subfigure}
  \centering
  \includegraphics[width=0.49\textwidth]{/home/dolinska/Dropbox/desy_plots/Thesis/Jenya/xSec/xsecNorm/xSec_ttbar_delta_phi_IN_ttbar_mass_1.pdf}
\end{subfigure}
\begin{subfigure}
  \centering
  \includegraphics[width=0.49\textwidth]{/home/dolinska/Dropbox/desy_plots/Thesis/Jenya/xSec/xsecNorm/xSec_ttbar_delta_phi_IN_ttbar_mass_2.pdf}
\end{subfigure}
% \begin{subfigure}
%   \centering
%   \includegraphics[width=0.32\textwidth]{/home/dolinska/Dropbox/desy_plots/Thesis/Jenya/xSec/xsecNorm/xSec_ttbar_mass_IN_ttbar_delta_phi_0.pdf}
% \end{subfigure}
% \begin{subfigure}
%   \centering
%   \includegraphics[width=0.32\textwidth]{/home/dolinska/Dropbox/desy_plots/Thesis/Jenya/xSec/xsecNorm/xSec_ttbar_mass_IN_ttbar_delta_phi_1.pdf}
% \end{subfigure}
% \begin{subfigure}
%   \centering
%   \includegraphics[width=0.32\textwidth]{/home/dolinska/Dropbox/desy_plots/Thesis/Jenya/xSec/xsecNorm/xSec_ttbar_mass_IN_ttbar_delta_phi_2.pdf}
% \end{subfigure}
\caption{Normalized differential cross sections in bins of $M(t\bar{t})$ and $\Delta\phi(t\bar{t})$. Other details as in Fig. \ref{fig:XS_2D_y_pt}.}
\label{fig:XS_2D_phi_Mtt}
\end{figure}

\begin{figure}[H]
\centering
\begin{subfigure}
  \centering
  \includegraphics[width=0.49\textwidth]{/home/dolinska/Dropbox/desy_plots/Thesis/Jenya/xSec/xsecNorm/xSec_ttbar_arapidity_IN_ttbar_mass_0.pdf}
\end{subfigure}
\begin{subfigure}
  \centering
  \includegraphics[width=0.49\textwidth]{/home/dolinska/Dropbox/desy_plots/Thesis/Jenya/xSec/xsecNorm/xSec_ttbar_arapidity_IN_ttbar_mass_1.pdf}
\end{subfigure}
\begin{subfigure}
  \centering
  \includegraphics[width=0.49\textwidth]{/home/dolinska/Dropbox/desy_plots/Thesis/Jenya/xSec/xsecNorm/xSec_ttbar_arapidity_IN_ttbar_mass_2.pdf}
\end{subfigure}
% \begin{subfigure}
%   \centering
%   \includegraphics[width=0.32\textwidth]{/home/dolinska/Dropbox/desy_plots/Thesis/Jenya/xSec/xsecNorm/xSec_ttbar_mass_IN_ttbar_arapidity_0.pdf}
% \end{subfigure}
% \begin{subfigure}
%   \centering
%   \includegraphics[width=0.32\textwidth]{/home/dolinska/Dropbox/desy_plots/Thesis/Jenya/xSec/xsecNorm/xSec_ttbar_mass_IN_ttbar_arapidity_1.pdf}
% \end{subfigure}
% \begin{subfigure}
%   \centering
%   \includegraphics[width=0.32\textwidth]{/home/dolinska/Dropbox/desy_plots/Thesis/Jenya/xSec/xsecNorm/xSec_ttbar_mass_IN_ttbar_arapidity_2.pdf}
% \end{subfigure}
\caption{Normalized differential cross sections in bins of $M(t\bar{t})$ and $|y(t\bar{t})|$. Other details as in Fig. \ref{fig:XS_2D_y_pt}.}
\label{fig:XS_2D_ytt_Mtt}
\end{figure}

\begin{figure}[H]
\centering
\begin{subfigure}
  \centering
  \includegraphics[width=0.49\textwidth]{/home/dolinska/Dropbox/desy_plots/Thesis/Jenya/xSec/xsecNorm/xSec_ttbar_pt_IN_ttbar_mass_0.pdf}
\end{subfigure}
\begin{subfigure}
  \centering
  \includegraphics[width=0.49\textwidth]{/home/dolinska/Dropbox/desy_plots/Thesis/Jenya/xSec/xsecNorm/xSec_ttbar_pt_IN_ttbar_mass_1.pdf}
\end{subfigure}
\begin{subfigure}
  \centering
  \includegraphics[width=0.49\textwidth]{/home/dolinska/Dropbox/desy_plots/Thesis/Jenya/xSec/xsecNorm/xSec_ttbar_pt_IN_ttbar_mass_2.pdf}
\end{subfigure}
\caption{Normalized differential cross sections in bins of $M(t\bar{t})$ and $p_{T}(t\bar{t})$. Other details as in Fig. \ref{fig:XS_2D_y_pt}.}
\label{fig:XS_2D_Mtt_pttt}
\end{figure}
% \begin{figure}[H]
% \centering
% \begin{subfigure}
%   \centering
%   \includegraphics[width=0.49\textwidth]{/home/dolinska/Dropbox/desy_plots/Thesis/Jenya/xSec/xsecNorm/xSec_ttbar_mass_IN_ttbar_pt_0.pdf}
% \end{subfigure}
% \begin{subfigure}
%   \centering
%   \includegraphics[width=0.49\textwidth]{/home/dolinska/Dropbox/desy_plots/Thesis/Jenya/xSec/xsecNorm/xSec_ttbar_mass_IN_ttbar_pt_1.pdf}
% \end{subfigure}
% \begin{subfigure}
%   \centering
%   \includegraphics[width=0.49\textwidth]{/home/dolinska/Dropbox/desy_plots/Thesis/Jenya/xSec/xsecNorm/xSec_ttbar_mass_IN_ttbar_pt_2.pdf}
% \end{subfigure}
% \begin{subfigure}
%   \centering
%   \includegraphics[width=0.49\textwidth]{/home/dolinska/Dropbox/desy_plots/Thesis/Jenya/xSec/xsecNorm/xSec_ttbar_mass_IN_ttbar_pt_3.pdf}
% \end{subfigure}
% \caption{Normalized differential cross sections in bins of $p_{T}(t\bar{t})$ and $M(t\bar{t})$. The inner error bands are the statistical uncertainties from the data.
%          The outer error bars are the combines statistical and systematical uncertainties on the data. The cross sections predicted different models are also presented:
%          \MG + \PYTHIA (red line), \Powheg + \PYTHIA (blue line), \Powheg + \HERWIG (orange line) and \MCNLO + \HERWIG (green line).}
% \label{fig:XS_2D_Mtt_pttt1}
% \end{figure}

\begin{figure}[H]
\centering
\begin{subfigure}
  \centering
  \includegraphics[width=0.49\textwidth]{/home/dolinska/Dropbox/desy_plots/Thesis/Jenya/xSec/xsecNorm/xSec_x1_IN_ttbar_mass_0.pdf}
\end{subfigure}
\begin{subfigure}
  \centering
  \includegraphics[width=0.49\textwidth]{/home/dolinska/Dropbox/desy_plots/Thesis/Jenya/xSec/xsecNorm/xSec_x1_IN_ttbar_mass_1.pdf}
\end{subfigure}
\begin{subfigure}
  \centering
  \includegraphics[width=0.49\textwidth]{/home/dolinska/Dropbox/desy_plots/Thesis/Jenya/xSec/xsecNorm/xSec_x1_IN_ttbar_mass_2.pdf}
\end{subfigure}
\caption{Normalized differential cross sections in bins of  $x_{1}$ and $M(t\bar{t})$. Other details as in Fig. \ref{fig:XS_2D_y_pt}.}
\label{fig:XS_2D_x1_Mtt}
\end{figure}
% \begin{sidewaysfigure}[H]
% \centering
% \begin{subfigure}
%   \centering
%   \includegraphics[width=0.32\textwidth]{/home/dolinska/Dropbox/desy_plots/Thesis/Jenya/xSec/xsecNorm/xSec_ttbar_mass_IN_x1_0.pdf}
% \end{subfigure}
% \begin{subfigure}
%   \centering
%   \includegraphics[width=0.32\textwidth]{/home/dolinska/Dropbox/desy_plots/Thesis/Jenya/xSec/xsecNorm/xSec_ttbar_mass_IN_x1_1.pdf}
% \end{subfigure}
% \begin{subfigure}
%   \centering
%   \includegraphics[width=0.32\textwidth]{/home/dolinska/Dropbox/desy_plots/Thesis/Jenya/xSec/xsecNorm/xSec_ttbar_mass_IN_x1_2.pdf}
% \end{subfigure}
% \begin{subfigure}
%   \centering
%   \includegraphics[width=0.32\textwidth]{/home/dolinska/Dropbox/desy_plots/Thesis/Jenya/xSec/xsecNorm/xSec_ttbar_mass_IN_x1_3.pdf}
% \end{subfigure}
% \begin{subfigure}
%   \centering
%   \includegraphics[width=0.32\textwidth]{/home/dolinska/Dropbox/desy_plots/Thesis/Jenya/xSec/xsecNorm/xSec_ttbar_mass_IN_x1_4.pdf}
% \end{subfigure}
% \caption{Normalized differential cross sections in bins of $M(t\bar{t})$ and  $x_{1}$. The inner error bands are the statistical uncertainties from the data.
%          The outer error bars are the combines statistical and systematical uncertainties on the data. The cross sections predicted different models are also presented:
%          \MG + \PYTHIA (red line), \Powheg + \PYTHIA (blue line), \Powheg + \HERWIG (orange line) and \MCNLO + \HERWIG (green line).}
% \label{fig:XS_2D_x1_Mtt1}
% \end{sidewaysfigure}

% \begin{sidewaysfigure}[H]
% \centering
% \begin{subfigure}
%   \centering
%   \includegraphics[width=0.32\textwidth]{/home/dolinska/Dropbox/desy_plots/Thesis/Jenya/xSec/xsecNorm/xSec_ttbar_pt_IN_x1_0.pdf}
% \end{subfigure}
% \begin{subfigure}
%   \centering
%   \includegraphics[width=0.32\textwidth]{/home/dolinska/Dropbox/desy_plots/Thesis/Jenya/xSec/xsecNorm/xSec_ttbar_pt_IN_x1_1.pdf}
% \end{subfigure}
% \begin{subfigure}
%   \centering
%   \includegraphics[width=0.32\textwidth]{/home/dolinska/Dropbox/desy_plots/Thesis/Jenya/xSec/xsecNorm/xSec_ttbar_pt_IN_x1_2.pdf}
% \end{subfigure}
% \begin{subfigure}
%   \centering
%   \includegraphics[width=0.32\textwidth]{/home/dolinska/Dropbox/desy_plots/Thesis/Jenya/xSec/xsecNorm/xSec_ttbar_pt_IN_x1_3.pdf}
% \end{subfigure}
% \begin{subfigure}
%   \centering
%   \includegraphics[width=0.32\textwidth]{/home/dolinska/Dropbox/desy_plots/Thesis/Jenya/xSec/xsecNorm/xSec_ttbar_pt_IN_x1_4.pdf}
% \end{subfigure}
% \caption{Normalized differential cross sections in bins of  $x_{1}$ and $p_{T}(t\bar{t})$. The inner error bands are the statistical uncertainties from the data.
%          The outer error bars are the combines statistical and systematical uncertainties on the data. The cross sections predicted different models are also presented:
%          \MG + \PYTHIA (red line), \Powheg + \PYTHIA (blue line), \Powheg + \HERWIG (orange line) and \MCNLO + \HERWIG (green line).}
% \label{fig:XS_2D_x1_pttt}
% \end{sidewaysfigure}
% \begin{figure}[H]
% \centering
% \begin{subfigure}
%   \centering
%   \includegraphics[width=0.49\textwidth]{/home/dolinska/Dropbox/desy_plots/Thesis/Jenya/xSec/xsecNorm/xSec_x1_IN_ttbar_pt_0.pdf}
% \end{subfigure}
% \begin{subfigure}
%   \centering
%   \includegraphics[width=0.49\textwidth]{/home/dolinska/Dropbox/desy_plots/Thesis/Jenya/xSec/xsecNorm/xSec_x1_IN_ttbar_pt_1.pdf}
% \end{subfigure}
% \begin{subfigure}
%   \centering
%   \includegraphics[width=0.49\textwidth]{/home/dolinska/Dropbox/desy_plots/Thesis/Jenya/xSec/xsecNorm/xSec_x1_IN_ttbar_pt_2.pdf}
% \end{subfigure}
% \begin{subfigure}
%   \centering
%   \includegraphics[width=0.49\textwidth]{/home/dolinska/Dropbox/desy_plots/Thesis/Jenya/xSec/xsecNorm/xSec_x1_IN_ttbar_pt_3.pdf}
% \end{subfigure}
% \caption{Normalized differential cross sections in bins of $p_{T}(t\bar{t})$ and  $x_{1}$. The inner error bands are the statistical uncertainties from the data.
%          The outer error bars are the combines statistical and systematical uncertainties on the data. The cross sections predicted different models are also presented:
%          \MG + \PYTHIA (red line), \Powheg + \PYTHIA (blue line), \Powheg + \HERWIG (orange line) and \MCNLO + \HERWIG (green line).}
% \label{fig:XS_2D_x1_pttt1}
% \end{figure}

%%%%%%%%%%%%%%%%%%%%%%%%%%%

%%%%%%%%%%%%%%%%%%%%%%%%%%%
%%%%%%%%%%%%%%%%%%%%%%%%%%%
%%%%%%%%%%%%%%%%%%%%%%%%%%%

\section{Discussion}
In general, the cross sections measured and presented in this work are in general in reasonable agreement with LO and NLO predictions
implemented in differential MC event generators (LO $\MG+\PYTHIA$ and NLO $\Powheg+\HERWIG$, $\Powheg+\PYTHIA$ and $\MCNLO+\HERWIG$).
However, there are some disagreements and trends observed in particular cross sections bins and control distributions.
These should be discussed in more detail.

Additionally, the double differential distributions provide a detailed check of the tendencies observed in the single differential
cross sections, measured in the dilepton channel at a center-of-mass energy $\sqrt{s} = 8\;\text{TeV}$ (TOP-12-028) \cite{Khachatryan:2015oqa}.

\subsection{Comparison to the Single Differential Cross Sections}

The results of the measurement of single differential $t\bar{t}$ production cross sections \cite{Khachatryan:2015oqa, Asin2014Auth} showed some 
tendencies in the way how the theoretical predictions describe the measurements. The following table summarizes these observations:

\begin{center}
 \begin{tabular}{| c | c |}
  \hline
  \textbf{Cross Sections} 		& \textbf{$\frac{\mathbf{\text{Theory}}}{\mathbf{\text{Measurement}}}$ tendency} \\ \hline\hline
  
  In bins of $p_{T}(t)$			& Theoretical predictions are harder than measurement \\ \hline 
  In bins of $y(t)$			& Theoretical predictions are more central than measurement \\ \hline 
  In bins of $p_{T}(t\bar{t})$		& Theory predicts more $t\bar{t}$ pairs with high $p_{T}$ \\ \hline 
  In bins of $y(t\bar{t})$		& Theoretical predictions are more central in $y(t\bar{t})$ \\ \hline
  In bins of $M(t\bar{t})$		& Theory predicts a harder $M(t\bar{t})$ spectrum \\ \hline
 \end{tabular}
\end{center}

The corresponding single differential $t\bar{t}$ cross section plots are shown in Fig. \ref{fig:XSec_1d}

\begin{sidewaysfigure}[p]
\centering
\begin{subfigure}
  \centering
  \includegraphics[width=0.325\textwidth]{08_results/plots/CMS-TOP-12-028_Figure_012-a.pdf}
\end{subfigure}
\begin{subfigure}
  \centering
  \includegraphics[width=0.325\textwidth]{08_results/plots/CMS-TOP-12-028_Figure_012-c.pdf}
\end{subfigure}
\begin{subfigure}
  \centering
  \includegraphics[width=0.325\textwidth]{08_results/plots/CMS-TOP-12-028_Figure_014-a.pdf}
\end{subfigure}
\begin{subfigure}
  \centering
  \includegraphics[width=0.325\textwidth]{08_results/plots/CMS-TOP-12-028_Figure_014-b.pdf}
\end{subfigure}
\begin{subfigure}
  \centering
  \includegraphics[width=0.325\textwidth]{08_results/plots/CMS-TOP-12-028_Figure_014-c.pdf}
\end{subfigure}
\caption{Normalized differential $t\bar{t}$ production cross section in the dilepton channels as a function of the $p_{T}(t)$ (upper left), $y(t)$ (upper middle), 
         $p_{T}(t\bar{t})$ (upper right), $y(t\bar{t})$ (lower left) and $M(t\bar{t})$ (lower right). The data points are placed at the midpoint of the bins. The 
         inner (outer) error bars indicate the statistical (combined statistical and systematic) uncertainties. The measurements are compared to predictions from 
         $\MG+\PYTHIA$, $\Powheg+\PYTHIA$, $\Powheg+\HERWIG$ and $\MCNLO+\HERWIG$, and to NLO+NNLL \cite{Ferroglia:2013zwa, Li:2013mia} calculations, when available. 
         The lower part of each plot shows the ratio of the predictions to data. Plost taken from \cite{Khachatryan:2015oqa}.}
\label{fig:XSec_1d}
\end{sidewaysfigure}

It is interesting to see how the double differential cross sections measured in this analysis shed more light on these effects.

The double differential cross sections in bins of $p_{T}(t)$ and $|y(t)|$ (see Fig. \ref{fig:XS_2D_y_pt}) show that the effect observed in
the single differential cross sections in bins of $y(t)$ is stronger for the middle and high $p_{T}(t)$. That means, that there is a slight 
tendency that the top quarks with higher $p_{T}$ tend to be more central in the theoretical predictions compared to the measurements.

The effect observed for the single differential $p_{T}(t\bar{t})$ spectrum may be explained by an overestimation of the amount of hard radiation
in the predictions since such a radiation is usually accompanied by a recoiling $t\bar{t}$ system with high $p_{T}(t\bar{t})$. This effect is the weakest 
for $\MG+\PYTHIA$, is equal for $\MCNLO+\HERWIG$ and $\Powheg+\HERWIG$ and is the strongest for
$\Powheg+\PYTHIA$. The tendency that the measurements are more central in $y(t\bar{t})$ can originate from the PDF effects: it shows that the 
measurements have an overall less high $x$, which results in an overall smaller total rapidity. 
% It means that there are less possibilities to get different momenta from gluons for the top and antitop 
% quarks, thus their total rapidity will be smaller. 
This effect is the smallest for the $\MG+\PYTHIA$ model, and similarly high for $\Powheg+\PYTHIA$ and
$\Powheg+\HERWIG$ predictions. The double differential cross sections in bins of $p_{T}(t\bar{t})$ and $|y(t\bar{t})|$ (see Fig. \ref{fig:XS_2D_pttt_ytt}) 
show the same tendencies. The effect for the $y(t\bar{t})$ in the high $p_{T}(t\bar{t})$ is not pronounced, probably due to the statistical fluctuations.

\subsection{Observations on the Double Differential Measurements}\label{sec:ddxsec_discuss}

The region with the high invariant masses of the $t\bar{t}$ system shows disagreement between data and predictions in many different observables.

The double differential cross sections in bins of $M(t\bar{t})$ and $p_{T}$ (see Fig. \ref{fig:XS_2D_Mtt_pt}) shows that in the region with 
$M(t\bar{t}) > 600\;\text{GeV}$, predictions have a harder $p_{T}$ spectrum than the measurements. This may point to the fact that there are 
more soft radiation in MC models compared to the data. The PDF effects can be excluded due to the high value of the $M(t\bar{t})$ in which
the effect is observed. If a high mass has been produced in the hard interaction, the dominant part of the parton momenta were transfered to
create this mass and there is not much energy left to produce a shift towards the beam direction. For the same reason (lack of energy) it is also 
unlikely to find much hard radiation in the high $M(t\bar{t})$ region. 

A similar conclusion can be made looking at the double differential cross sections in bins of $M(t\bar{t})$ and $\Delta\eta(t\bar{t})$ (see Fig. \ref{fig:XS_2D_eta_Mtt}),
where in the high $M(t\bar{t})$ bin the $\Delta\eta(t\bar{t})$ for the predictions tends to smaller values compared to the measurements.
This effect also points to a slight excess of the soft radiation in the models.

For both cases the largest discrepancies between the measurements and the predicted cross section values are observed for the $\MG+\PYTHIA$
model.

There are no strong effects seen in the cross sections in bins of $M(t\bar{t})$ and $\Delta\phi(t\bar{t})$ (see Fig. \ref{fig:XS_2D_phi_Mtt}).

To investigate the guesses about the nature of the smaller $\Delta\eta(t\bar{t})$ (see Fig. \ref{fig:XS_2D_eta_Mtt}) discussed above,
a distribution of $\Delta\eta(t\bar{t})$ in bins of hard jet (with the $p_{T} > 30\;\text{GeV}$) multiplicity should be discussed
(see Fig. \ref{fig:eta_jetMult}). As one observes from the plot, the ratio between theory predictions and measurements doesn't depend
on the jet multiplicity (not taking to account the high multiplicity bin where the statistics is low). The conclusion is that it
is not the hard radiation in the predictions, which is causing the effect. This statement confirms the guess which was made before, that
the reason for the effect is the soft radiation.

\begin{figure}[t]
  \centering
  \includegraphics[width=0.8\textwidth]{/home/dolinska/Dropbox/desy_plots/Thesis/Jenya/Discussions/CP_AllBins_ttbar_delta_eta_vs_jet_multiplicity.pdf}
  \caption{Control distribution of $\Delta\eta({t\bar{t}})$ in bins of jet multiplicity for $\mathbf{M(t\bar{t})\: >}$ \textbf{600 GeV}. The $|\Delta\eta(t\bar{t})|$ 
  bins are shown on the top of the plot. The experimental data are marked with the black dots and the reconstructed MC signal is marked with the red area. The error
  bars on the data points represent the statistical uncertainty only. The 
  different background contributions are also shown. On the bottom part of the plot the ratio between MC and data statistics in each bin
  is presented.}
  \label{fig:eta_jetMult}
\end{figure}

The effect on the $\Delta\eta(t\bar{t})$ in the highest $M(t\bar{t})$ bin is also checked in the systematic variations of matching and hard scales in the nominal $\MG+\PYTHIA$ sample 
(see sec. \ref{ssec:matchS_sys} and sec. \ref{ssec:hardS_sys}). The control distributions of $\Delta\eta(t\bar{t})$ in the finer binning are shown 
in Fig. \ref{fig:CP_eta_match} for the matching scale variations and in Fig. \ref{fig:CP_eta_hard} for the hard scale variations. All the varied 
distributions are compared to the nominal unvaried spectrum. This comparison shows that the effect of the MC tending towards smaller $\Delta\eta(t\bar{t})$
compared to the measurements is getting smaller for the variation of the matching scale up by factor 2 and for the variation of the hard scale down 
by factor 0.5.

\begin{figure}[t]
\centering
\begin{subfigure}
  \centering
  \includegraphics[width=0.49\textwidth]{/home/dolinska/Dropbox/desy_plots/Thesis/Jenya/Discussions/CP_ttbar_delta_eta-MATCH_UP.pdf}
\end{subfigure}
\begin{subfigure}
  \centering
  \includegraphics[width=0.49\textwidth]{/home/dolinska/Dropbox/desy_plots/Thesis/Jenya/Discussions/CP_ttbar_delta_eta-MATCH_DOWN.pdf}
\end{subfigure}
\begin{subfigure}
  \centering
  \includegraphics[width=0.49\textwidth]{/home/dolinska/Dropbox/desy_plots/Thesis/Jenya/Discussions/CP_ttbar_delta_eta.pdf}
\end{subfigure}
\caption{Control distribution of $\Delta\eta({t\bar{t}})$ compared to the MC with matching scale variation by factor 2 (top left), to the MC with matching scale variation 
         by factor 0.5 (top right) and to the nominal unvaried MC (bottom) for $\mathbf{M(t\bar{t})\: >}$ \textbf{600 GeV}. The experimental data are marked with the black 
         dots and the reconstructed MC signal is marked with the red area. The error
         bars on the data points represent the statistical uncertainty only. The 
         different background contributions are also shown. On the bottom part of the plot the ratio between MC and data statistics in each bin
         is presented.}
\label{fig:CP_eta_match}
\end{figure}

\begin{figure}[t]
\centering
\begin{subfigure}
  \centering
  \includegraphics[width=0.49\textwidth]{/home/dolinska/Dropbox/desy_plots/Thesis/Jenya/Discussions/CP_ttbar_delta_eta-SCALE_UP.pdf}
\end{subfigure}
\begin{subfigure}
  \centering
  \includegraphics[width=0.49\textwidth]{/home/dolinska/Dropbox/desy_plots/Thesis/Jenya/Discussions/CP_ttbar_delta_eta-SCALE_DOWN.pdf}
\end{subfigure}
\begin{subfigure}
  \centering
  \includegraphics[width=0.49\textwidth]{/home/dolinska/Dropbox/desy_plots/Thesis/Jenya/Discussions/CP_ttbar_delta_eta.pdf}
\end{subfigure}
\caption{Control distribution of $\Delta\eta({t\bar{t}})$ compared to the MC with hard scale variation by factor 2 (top left), to the MC with hard scale variation 
         by factor 0.5 (top right) and to the nominal unvaried MC (bottom) for $\mathbf{M(t\bar{t})\: >}$ \textbf{600 GeV}. The experimental data are marked with the black 
         dots and the reconstructed MC signal is marked with the red area. The error
         bars on the data points represent the statistical uncertainty only. The 
         different background contributions are also shown. On the bottom part of the plot the ratio between MC and data statistics in each bin
         is presented.}
\label{fig:CP_eta_hard}
\end{figure}
% \subsection{Trend in the $\Delta\eta$ Between Top and Anti-top} \label{sec:deta_discuss}
% 
% If one throws a look onto the $t\bar{t}$ cross sections in bins of $\Delta\eta{t\bar{t}}$ in different bins of $t\bar{t}$ mass
% (see fig.\ref{fig:XS_2D_eta_Mtt}), one observes the tendency that the $\Delta\eta{t\bar{t}}$ is not described in the higher mass bins.
% The control distribution (see fig. \ref{fig:CP_2D_eta_Mtt}) shows the trend, that the higher the $M(t\bar{t})$ is, the smaller $\Delta\eta(t\bar{t})$
% is modeled in $\MG+\PYTHIA$. That means that the top quarks, reconstructed from the experimental data, are more often back-to-back 
% than for the MC.
% 
% The reason for the discrepancies in the pseudorapidities between two top-quarks may originate because of two reasons:
% 
% \begin{itemize}
%  \item \textit{Wrong PDFs in the MC}: If the $x_{1}$ and $x_{2}$ transmitted from partons to first and second top quark are very different, than these
%  top quarks will fly not back-to-back. This scenario, however, is unlikely, as the effect of underestimation of $\Delta\eta(t\bar{t})$
%  by MC is observed stronger if the mass of the $t\bar{t}$ system is larger. For large $M(t\bar{t})$ the transmitted momenta from the partons
%  to the top quarks are equivalently large.
%  
%  \item \textit{Additional radiation}: The presence of radiation may deviate the direction of the top quarks so that they will no longer be back-to-back.
% \end{itemize}
% 
% One might conclude that MC models more radiation on higher $M(t\bar{t})$. This can be checked by looking at the $\Delta\phi(t\bar{t})$ in this region.
% $\Delta\phi(t\bar{t})$ is not sensitive to the differences in $x_{1}$ and $x_{2}$ as there is no transverse component in proton momentum.
% This control distribution is presented in fig. \ref{fig:CP_phi_eta}. It shows that there is a trend that there are more MC the smaller $\Delta\eta(t\bar{t})$.
% In addition, there is a trend that there are more MC for the smaller $\Delta\phi(t\bar{t})$. This means, that there is more radiation modeled
% in MC out of which there is more hard radiation (where $\Delta\phi(t\bar{t})$ is smaller).
% 
% \begin{figure}[t]
%   \centering
%   \includegraphics[width=0.8\textwidth]{09_conclusions/plots/CP_phi_eta.png}
%   \caption{Control distribution of $\Delta\phi({t\bar{t}})$ in bins of $\Delta\eta(t\bar{t})$ for $M(t\bar{t})\: >$ 600 GeV.}
%   \label{fig:CP_phi_eta}
% \end{figure}
% 
% After comparing the results of the model predictions available in this analysis (see fig.\ref{fig:XS_2D_eta_Mtt}), one can conclude that this effect is the
% strongest for $\MG+\PYTHIA$.

\section{Summary on the Results}

After presenting the normalized\footnote{As the analysis presented in this work was tuned to measure the normalized cross sections, the unnormalized cross sections are not
discussed in this section. The results with unnormalized double differential top-quark-pair production cross sections are presented in Appendices \ref{appendix:unnorm_XSec}
and \ref{appendix:xsec_table}} double differential top-quark-pair production cross sections in bins of nine different combinations of variables,
one can summarize that overall the theoretical description of the measured data points is good. The precision of the measurements is allows a visual 
comparison of the experimental results to different theoretical models in LO and NLO.

All the double differential distributions and normalized cross sections, shown in previous section, can be divided into three groups:

\begin{itemize}
 \item [--] results, which describe the $t$ dynamics;
 \item [--] results, which describe the $t\bar{t}$ system dynamics;
 \item [--] results on combined $t$  and $t\bar{t}$ dynamics.
\end{itemize}

The results presented in bins of $p_{T}(t)$ and $|y(t)|$ (fig. \ref{fig:XS_2D_y_pt}) are describing the $t$
dynamics. The observation, made previously in the analysis on measurement of top-quark-pair differential cross sections in dileptonic channel at 
$\sqrt{s}$ = 8 TeV \cite{Asin2014Auth}, was that the top-quark transverse momentum spectrum was softer in data than most of the predictions. The best
description was provided by $\Powheg+\HERWIG$ predictions. However, this effect is very weak in the highest $|y(t)|$ bin. One can also conclude that the $\Powheg+\HERWIG$
predictions provide the best description of the data point, except for the highest $|y(t)|$ regions.

If talking about the $t\bar{t}$ system dynamics, the results in bins of $p_{T}(t\bar{t})$ and $|y(t\bar{t})|$ (fig. \ref{fig:XS_2D_pttt_ytt})
show overall good agreement with the predictions. $\MG+\PYTHIA$ describe experimental points the best way. However, the slight tendencies to the softer $p_{T}(t\bar{t})$
and central $y(t\bar{t})$ in the measured data are observed. The same tendency towards softer $p_{T}(t\bar{t})$ in the measurements is observed for the distributions 
in bins of $p_{T}(t\bar{t})$ and $M(t\bar{t})$ (fig. \ref{fig:XS_2D_Mtt_pttt}).

The distributions, which represent the combined dynamics of $t$ and $t\bar{t}$, are the distributions of $M(t\bar{t})$ in different bins of $p_{T}(t)$, $|y(t)|$,
$\Delta\eta(t\bar{t})$ and $\Delta\phi(t\bar{t})$ (fig. \ref{fig:XS_2D_Mtt_pt}, fig. \ref{fig:XS_2D_Mtt_yt}, fig. \ref{fig:XS_2D_eta_Mtt}
and fig. \ref{fig:XS_2D_phi_Mtt}). The general observation is that the highest $M(t\bar{t})$ bin has the worst description of data by the predictions. This 
trend in the distributions may be caused by the larger amount of soft radiation modeled in the MC predictions (discussion in sec. \ref{sec:ddxsec_discuss}).