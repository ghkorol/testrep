\chapter{Measurement without Regularization}\label{appendix:tau_zero}

To check the how much the regularization (see sec. \ref{ssec:TUmini}) influences the results, a measurement of the 
normalized double differential cross sections in bins of $p_{T}(t)$ and $|y(t)|$ was performed with manually setting the regularization
strength $\tau$ to zero.

Fig. \ref{fig:corr_dtau} shows the correlation matrices for the measurement in bins of $p_{T}(t)$ and $|y(t)|$ for the cases when the 
regularization strength is optimal (see Appendix \ref{appendix:tau}) and when the regularization strength is zero. One can conclude 
that without the regularization one also obtains a meaningful result. The comparison of the two correlation matrices shows that 
for the case with a non-zero optimal $\tau$ the anticorrelations are moved a bit from the directly neighboring bins to more 
distant bins.

\begin{figure}[h]
\centering
\begin{subfigure}
  \centering
  \includegraphics[width=0.49\textwidth]{/home/dolinska/Dropbox/desy_plots/Thesis/Jenya/xSec/covX/histCovXnorm_top_arapidity_vs_top_pt.pdf}
\end{subfigure}
\begin{subfigure}
  \centering
  \includegraphics[width=0.49\textwidth]{/home/dolinska/Dropbox/desy_plots/Thesis/Jenya/tauZERO/histCovXnorm_top_arapidity_vs_top_pt.pdf}
\end{subfigure}
\caption{Correlation matrix $\mathbf{V}_{xx}$ for the bins of $p_{T}(t)$ and $|y(t)|$ with regularization (left) and without regularization 
        (right). The binning is the following: the five sequences of three bins (1-3, 4-6, 7-9, 10-12, 13-15) correspond to the five $p_{T}(t)$ 
        bins $[0,\:\:65,\:\:130,\:\:200,\:\:300,\:\:500]$ GeV. There are three $|y(t)|$ bins -- $[0,\:\:0.6,\:\:1.2,\:\:2.5]$ -- 
        in each $p_{T}(t)$ bin.}
\label{fig:corr_dtau}
\end{figure}

The cross section measured without the regularization in bins of $p_{T}(t)$ and $|y(t)|$ are shown in Fig. \ref{fig:XSec_zerotau}.
These cross sections can be compared to the corresponding nominal cross sections which were measured applying regularization
(see Fig. \ref{fig:XS_2D_y_pt}). In general, the observed differences in the cross section values are within $\pm 1\sigma$ of
the cross sections with regularization. Both statistical and systematic uncertainties increase for the unregularized cross sections 
by factor $\sim 1.4$, however, this is accompanied by increasing anticorrelations between neighboring bins.

\begin{figure}[h]
\centering
\begin{subfigure}
  \centering
  \includegraphics[width=0.49\textwidth]{/home/dolinska/Dropbox/desy_plots/Thesis/Jenya/tauZERO/xSec_top_pt_IN_top_arapidity_0.pdf}
\end{subfigure}
\begin{subfigure}
  \centering
  \includegraphics[width=0.49\textwidth]{/home/dolinska/Dropbox/desy_plots/Thesis/Jenya/tauZERO/xSec_top_pt_IN_top_arapidity_1.pdf}
\end{subfigure}
\begin{subfigure}
  \centering
  \includegraphics[width=0.49\textwidth]{/home/dolinska/Dropbox/desy_plots/Thesis/Jenya/tauZERO/xSec_top_pt_IN_top_arapidity_2.pdf}
\end{subfigure}
\caption{Normalized differential cross sections in bins of $|y(t)|$ and $p_{T}(t)$ with no regularization applied. The inner error bars show the statistical 
         uncertainties from the data. The outer error bars are the combined statistical and systematical uncertainties on the data. The predicted cross 
         sections from four different models are also presented: \MG + \PYTHIA (red line), \Powheg + \PYTHIA (blue line), \Powheg + \HERWIG (orange line) and 
         \MCNLO + \HERWIG (green line). The ratio of the predictions over measured cross sections are shown in the bottom panels with the error bars 
         corresponding to the measurement uncertainties.}
\label{fig:XSec_zerotau}
\end{figure}