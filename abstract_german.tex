\thispagestyle{empty}
\vspace{-3cm}
\section*{\centering Kurzfassung}

\vspace{\baselineskip}

% In dieser Arbeit wird eine Messung von Charm und Beauty Produktion in tief-unelastischer $ep$-Streuung bei HERA vorgestellt.
% Die Analyse basiert auf Daten, die im Zeitraum von 2003 bis 2007 mit dem ZEUS-Detektor aufgezeichnet wurden, und einer integrierten Luminosit\"at von \SI{354}{\per\pico\barn} entsprechen.
% Der kinematische Bereich der Messung wird durch $\num{5}<\Qsq<\SI{1000}{\GeV\squared}$ und $0.02<y<0.7$ definiert, wobei $Q^2$ die Virtualit\"at und $y$ die Inelastizit\"at ist.
% Um die Ereignisse mit Charm- und Beauty-Quarks zu identifizieren wurde die endliche Lebensdauer der durch die schwache Kraft zerfallenden Grundzust\"ande der Charm- und Beauty-Hadronen ausgenutzt.
% Sekund\"are Zerfallspunkte von Charm- und Beauty-Hadronen, die mit Jets assoziiert sind, werden rekonstruiert.
% Die Kinematik der Jets wird auf den folgenden Bereich eingeschr\"ankt: $\ETjet>\SI{4.2(5)}{\GeV}$ f\"ur Charm (Beauty) und $-1.6<\etajet<2.2$ f\"ur Charm und Beauty,
% wobei \ETjet und \etajet die transversale Energie bzw. Pseudorapidit\"at ist.
% Die Signifikanz der Zerfallsl\"ange sowie die Masse der geladenen Spuren vom Zerfallsvertex werden als diskriminierende Variablen benutzt, um zwischen dem Charm-Signal, dem Beauty-Signal und dem Untergrund zu unterscheiden.
% Differenzielle Wirkungquerschnitte von Jet-Produktion in Charm- und Beauty-Ereignissen werden als Funktion von \Qsq, $y$, \ETjet and \etajet gemessen.
% Die Ergebnisse werden mit Next-to-Leading Order (NLO) Vorhersagen der Quantenchromodynamik (QCD) im ``Fixed Flavour Number Scheme'' verglichen.
% Die Theorie stimmt gut mit den Daten \"uberein.
% Die Beitr\"age von Charm- und Beauty-Produktion zur inklusiven Protonstrukturfunktion, \Ftwoc and \Ftwob,
% werden durch Extrapolation von doppelt differenziellen Wirkungsquerschnitten mit Hilfe der NLO QCD Vorhersagen bestimmt.
% 
% \noindent
% Beitr\"age zur Teststrahlmessung-Program von Insertable B-Layer Upgrade-Projekt f\"ur {ATLAS} Pixeldetektor werden vorgestellt.
% Die Teststrahlmessung-Analysesoftware EUTelescope wurde erweitert, was eine effiziente Analyse von ATLAS-Pixelsensoren erlaubt hat.
% Das USBPix DAQ-System wurde in das EUDET-Teleskop integriert, was Teststrahlmessungen mit dem Front-End-Chip FE-I4 erm\"oglicht.
% Planar- und 3D- ATLAS Pixelsensoren wurden w\"ahrend der ersten IBL-Teststrahlmessung am CERN SPS studiert.

\newpage
\thispagestyle{empty}
\mbox{}