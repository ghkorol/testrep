\thispagestyle{empty}
\vspace{-3cm}
\section*{\centering Kurzfassung}

\vspace{\baselineskip}

Die hohe Energieskala der $pp$ Kollisionen am Large Hadron Collider (LHC) am CERN macht
diese Beschleunigeranlage zu einer wahren $t\bar{t}$-Paar Produktionsfabrik. 
%
Dies erlaubt es die Eigenschaften und
Produktions- und Zerfallmechanismen von top-Quarks mit beispielloser Pr{\"a}zision zu erforschen.
%
In dieser Analyse wurden top-Quarkpaare im dileptonischen 
Zerfallskanal untersucht bei dem beide W-Bosonen aus den top-Quarkzerf\"allen
in ein Lepton und Neutrino zerfallen.
%
Die Beschr\"ankung in dieser Arbeit auf Endzust\"ande mit einem Elektron und einem Muon
unterdr\"uckt sehr stark m\"ogliche Untergrundprozesse und f\"uhrt zu einer hohen Signalreinheit.
%
Etwa 40k Ereignisse mit einem top-Quarkpaar sind aus den
Daten, die mit dem CMS Detektor im Jahr 2012 bei einer Schwerpunktsenergie
von  $\sqrt{s}$ = 8 TeV genommen wurden, ausgew{\"a}hlt worden. 
%
Diese gro{\ss}e Datenmenge erm\"oglicht es zum ersten Mal 
doppeldifferenzielle Wirkungsquerschnitte der top-Quarkpaarproduktion
zu messen.
%
Die Wirkungsquerschnitte sind als Funktion von verschiedenen Variablen, die die Kinematik des
top-Quarks und des top-Quarkpaares beschreiben, untersucht worden.
%
Um die volle Kinematik des $t\bar{t}$ Endzustandes mit zwei nicht detektierbaren Neutrinos 
zu erhalten, wurde ein neues Verfahren der kinematischen Ereignisrekonstruktion im Rahmen dieser Arbeit entwickelt und benutzt. 
%
Die beobachteten Verteilungen der Daten als Funktion zweier rekonstruierter kinematischer
Variablen wurden dann mit einer zweidimensionalen Entfaltungsmethode, basierend
auf einer $\chi^2$-Minimisierung,  auf Detektoreffekte
korrigiert um so die doppeltdifferenziellen Wirkungsquerschnitte zu erlangen.
%
Die in dieser Arbeit gewonnenen Wirkungsquerschnitte erlauben 
es das  Standardmodell der Teilchenphysik in grosser Detailtiefe 
zu {\"u}berpr{\"u}fen und vorher beobachtete Diskrepanzen zwischen gemessenen und vorhergesagten 
einfach-differenziellen Wirkungsquerschnitten genauer anzuschauen. 
%
Die Resultate dieser Arbeit werden mit vier verschiedenen 
Standardmodellvorhersagen (bis hin zur n\"achstf\"uhrenden Ordnung der QCD St\"orungsreihe),
repr\"asentiert durch vier Monte Carlo Ereignisgeneratoren, verglichen.

Im Allgemeinen stimmen die Resultate und Vorhersagen {\"u}berein, au{\ss}er in einigen kinematischen
Bereichen, was ausf{\"u}hrlich diskutiert wird.

\newpage
\thispagestyle{empty}
\mbox{}