\thispagestyle{empty}
\vspace{-3cm}
\section*{\centering Kurzfassung}

\vspace{\baselineskip}

Die hohe Energieskala der $pp$ Kollisionen auf dem Large Hadron Collider (LHC) im CERN verwendelt 
die Anlage zur $t\bar{t}$-Paar Produktionsfabrik. Das erlaubt die Eigenschaften und
Produktuions- und Zerfallmechanismen von top-Quark mit der beispielloser Pr{\"a}zision zu erforschen.
Etwa 40k Ereignisse mit top-Quarkpaar und hohem Purity sind von der $\sqrt{s}$ = 8 TeV
Daten, die mit dem CMS Detektor im Jahr 2012 aufgezeichnet wurden, ausgew{\"a}hlt worden. 
Anhand diesen gro{\ss}en Sample sind die doppeldifferenziale Querschnitte der top-Quarkpaarproduktion
gemessen. Die Querschnitte sind als Funktionen der verschiedenen Variablen, die die Kinematik von
top-Quark und top-Quarkpaar beschreiben, untersucht.

Um die volle Kinematik des $t\bar{t}$ Endzustandes mit zwei nicht detektierbaren Neutrinos 
zu erhalten, war das Verfahren der kinematischer Rekonstruktion in Rahmen dieser Arbeit entwickelt und benutzt. 
Die rekonstruierten Destributionen sind f{\"u}r die Detektorbewirkungen von der Nutzung des doppeldifferenzialen
Unfolding, das sich auf der $\chi^{2}$ Minimisierung basiert, korregiert.

Die in dieser Arbeit vorgestellte doppeldifferenziale Querschnitte erlauben die Standart Modell ausf{\"u}hrlich 
zu {\"u}berpr{\"u}fen und die vorcher beobachtete Uneinigkeiten zwischen gemessenen und vorhergesagten einzeldifferenziallen
Querschnitte zu untersuchen. Die Resultaten dieser Arbeit sind mit vier unterschiedlichen Vorhersagen von
Standart Modell (bis aufs NLO von dem Stark Wechselwirkung Bezeichnung), die in verschiedenen Monte Carlo 
Ereignissgeneratoren implementiert sind, vergleicht.

Im Allgemeinen stimmen die Resultaten und die Vorhersagen {\"u}berein, au{\ss}er einigen F{\"a}llen, die ausf{\"u}hrlich
diskutiert sind.

% In dieser Arbeit wird eine Messung von Charm und Beauty Produktion in tief-unelastischer $ep$-Streuung bei HERA vorgestellt.
% Die Analyse basiert auf Daten, die im Zeitraum von 2003 bis 2007 mit dem ZEUS-Detektor aufgezeichnet wurden, und einer integrierten Luminosit\"at von \SI{354}{\per\pico\barn} entsprechen.
% Der kinematische Bereich der Messung wird durch $\num{5}<\Qsq<\SI{1000}{\GeV\squared}$ und $0.02<y<0.7$ definiert, wobei $Q^2$ die Virtualit\"at und $y$ die Inelastizit\"at ist.
% Um die Ereignisse mit Charm- und Beauty-Quarks zu identifizieren wurde die endliche Lebensdauer der durch die schwache Kraft zerfallenden Grundzust\"ande der Charm- und Beauty-Hadronen ausgenutzt.
% Sekund\"are Zerfallspunkte von Charm- und Beauty-Hadronen, die mit Jets assoziiert sind, werden rekonstruiert.
% Die Kinematik der Jets wird auf den folgenden Bereich eingeschr\"ankt: $\ETjet>\SI{4.2(5)}{\GeV}$ f\"ur Charm (Beauty) und $-1.6<\etajet<2.2$ f\"ur Charm und Beauty,
% wobei \ETjet und \etajet die transversale Energie bzw. Pseudorapidit\"at ist.
% Die Signifikanz der Zerfallsl\"ange sowie die Masse der geladenen Spuren vom Zerfallsvertex werden als diskriminierende Variablen benutzt, um zwischen dem Charm-Signal, dem Beauty-Signal und dem Untergrund zu unterscheiden.
% Differenzielle Wirkungquerschnitte von Jet-Produktion in Charm- und Beauty-Ereignissen werden als Funktion von \Qsq, $y$, \ETjet and \etajet gemessen.
% Die Ergebnisse werden mit Next-to-Leading Order (NLO) Vorhersagen der Quantenchromodynamik (QCD) im ``Fixed Flavour Number Scheme'' verglichen.
% Die Theorie stimmt gut mit den Daten \"uberein.
% Die Beitr\"age von Charm- und Beauty-Produktion zur inklusiven Protonstrukturfunktion, \Ftwoc and \Ftwob,
% werden durch Extrapolation von doppelt differenziellen Wirkungsquerschnitten mit Hilfe der NLO QCD Vorhersagen bestimmt.
% 
% \noindent
% Beitr\"age zur Teststrahlmessung-Program von Insertable B-Layer Upgrade-Projekt f\"ur {ATLAS} Pixeldetektor werden vorgestellt.
% Die Teststrahlmessung-Analysesoftware EUTelescope wurde erweitert, was eine effiziente Analyse von ATLAS-Pixelsensoren erlaubt hat.
% Das USBPix DAQ-System wurde in das EUDET-Teleskop integriert, was Teststrahlmessungen mit dem Front-End-Chip FE-I4 erm\"oglicht.
% Planar- und 3D- ATLAS Pixelsensoren wurden w\"ahrend der ersten IBL-Teststrahlmessung am CERN SPS studiert.

\newpage
\thispagestyle{empty}
\mbox{}