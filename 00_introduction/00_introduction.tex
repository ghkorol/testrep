\chapter{Introduction}

Nowadays the heaviest known elementary elementary particle is the top quark. 
Having a much smaller size then a proton, it is as heavy as the atom of gold. 
The search for the heaviest particle lasted more then two decades and successfully 
ended in March 1995 at Fermi National Accelerator Laboratory, Fermilab, where 
the discovery of the top quark was announced [!!!]. 

The third generation of quarks, which included top and bottom quarks, 
was predicted in 1973 by Kobayashi and Maskawa [!!!]. But a huge mass of a 
top quark was not expected, so it couldn't be discovered so long. While it's partner, 
the bottom quark, was experimentally found already in four years after the theoretical prediction. 

Thus, to produce a top quark one needs to concentrate immense amount of energy 
into a small region of space. This is done on the accelerator experiments. 
And the most powerful accelerator nowadays is the Large Hadron Collider (LHC), CERN. 
Delivering proton-proton collisions which had the center of mass energy of 8 TeV in 2012, 
it became a real top quark factory. The large number of the events with top quark, 
which were produced on the LHC, gives us a unique possibility to study precisely the properties of the heaviest quark. 

The top quarks are dominantly produced in pairs, decaying before they could hadronize. 
Each top quark of the pair decays to a W boson and a b quark. 
And a W boson has several decay channels. In this work only the channel 
where W bosons from top quarks decay to two leptons is studied. 
For this purpose the $19 fb^{-1}$ data sample with 8 TeV center-of-mass energy taken in 2012 is used.

This work represents the first measurement of the normalized double differential top pair production
cross section at the LHC. This measurement is done in totally XXX diffferent variables
connected to the top kinematics. The comparison to the theory predictions is also performed.

In order to contribute to the future measurements with the higher statistics during
the next years runs of the LHC, a part of my work was connected to the studies of irradiated
prototype silicon sensors and readout chips for the Phase-I upgrade of the CMS pixel detector.

This work is structured the following way.