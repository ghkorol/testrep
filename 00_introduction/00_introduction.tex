\chapter{Introduction}

Nowadays the heaviest known elementary elementary particle is the top quark which, is as heavy as the atom of gold. 
The search for this particle lasted more then two decades and ended successfully 
in March 1995 at the Fermi National Accelerator Laboratory, Fermilab, where 
the discovery of the top quark was announced \cite{PhysRevLett.74.2626}. 

The third generation of quarks, which includes top and bottom quarks, 
was predicted in 1973 by Kobayashi and Maskawa \cite{Kobayashi:1973fv}. However, the huge mass of the 
top quark was not expected, so it couldn't be discovered for a long time, while it's partner, 
the bottom quark, was experimentally found already four years after its theoretical prediction \cite{PhysRevLett.39.252}. 

To produce a top quark one needs to concentrate an immense amount of energy 
into a small region of space. This is done at the accelerator experiments. 
The most powerful accelerator nowadays is the Large Hadron Collider (LHC), CERN. 
Delivering proton-proton collisions with a center-of-mass energy of 7 TeV in 2011 and 8 TeV in 2012, 
it became a real top quark factory. The large number of events with top quarks
produced at the LHC gives us a unique possibility to study precisely the properties of the heaviest quark. 

The top quarks are dominantly produced together with an anti-top quark, which is called in the 
following $t\bar{t}$ production. Top quarks decay before they could hadronize. 
Each top quark of the pair decays to a $W$ boson and a $b$-quark almost exclusively. 
The W boson has several decay channels. In this work only the ``dilepton" channel, 
where both $W$ bosons from the two top quarks decay to a lepton and an (anti)neutrino, is studied. 
For this purpose the $19 fb^{-1}$ data sample with 8 TeV center-of-mass energy taken in 2012 was analyzed.

This work represents the first measurement of the normalized double differential (2D) top pair production
cross section at the LHC. The double differential $t\bar{t}$ production cross sections provide a stringent test of the 
Standard Model of Particle Physics, as they allow to study the $t\bar{t}$ production dynamics in unprecedented
detail. For the measurement of the double differential production cross sections a new kinematic reconstruction
of the $t\bar{t}$ events was implemented, providing an accurate and unbiased determination of the momenta of the
top and the anti top quarks.

This measurement is done in totally ten different 2D variable combinations
connected to the top kinematics. A comparison to theory predictions is also performed.

In order to contribute to future measurements with higher statistics during
the next years runs of the LHC, a part of my work was connected to the studies of irradiated
prototype silicon sensors and readout chips for the Phase-I upgrade of the CMS pixel detector.

This work is structured the following way. Chapter \ref{chapt:SM} gives an introduction to the theoretical
knowledge behind the measurements performed in this analysis. It gives a brief overview of the Standard
Model of particle physics and top quark physics in particular.

In chapter \ref{chap:exp_setup} the experimental setup used to obtain the data for this analysis is described.
It is delivering information about the LHC accelerator and the CMS detector, giving a brief description of each detector part.

As an extension to the previous chapter, chapter \ref{chapt:pixel} is describing the studies performed for the
upgrade of the barrel pixel part of the CMS detector. The upgraded detector is planned to be operated starting from 2016.

Simulation of all the physical processes and detector performance was performed to enable studies of the detector effects.
Additionally the simulated samples allow the comparison of the results of the measurements to the theoretical models, 
which are implemented in the simulation. An overview of the simulation
models and techniques exploited for the analysis described in this thesis is presented in chapter \ref{chapt:MC}.

The reconstruction and selection criteria applied for the real detector data and to the simulated events are described
in chapter \ref{chapt:event_selection}. After selecting the objects, like leptons, jets and missing transverse energy, which 
fulfill the criteria, described in chapter \ref{chapt:event_sel}, the $t\bar{t}$ candidate has to be reconstructed out
of them. The procedure of the full kinematic reconstruction of the $t\bar{t}$ system in the dileptonic final state is introduced
in chapter \ref{chapt:kinReco}.

Having the fully reconstructed event kinematics, the double differential $t\bar{t}$ production cross sections are calculated
as described in chapter \ref{chapt:xsec}. This section gives an overview of the data unfolding, shows the cross section determination 
and delivers the results of the cross section calculations with statistical uncertainties. The systematic uncertainties (sources and 
values) are described in chapter \ref{chapt:syst}.

The discussion of the results of the analysis final conclusions are summarized in chapter \ref{chapt:conc}.