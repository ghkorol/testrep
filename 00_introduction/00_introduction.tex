\chapter{Introduction}

Nowadays the heaviest known elementary elementary particle is the top quark. 
Being a point particle, it is as heavy as the atom of gold. 
The search for the heaviest particle lasted more then two decades and successfully 
ended in March 1995 at the Fermi National Accelerator Laboratory, Fermilab, where 
the discovery of the top quark was announced \cite{PhysRevLett.74.2626}. 

The third generation of quarks, which includes top and bottom quarks, 
was predicted in 1973 by Kobayashi and Maskawa \cite{Kobayashi:1973fv}. However, the huge mass of the 
top quark was not expected, so it couldn't be discovered for a long time, while it's partner, 
the bottom quark, was experimentally found already four years after its theoretical prediction. 

To produce a top quark one needs to concentrate an immense amount of energy 
into a small region of space. This is done at the accelerator experiments. 
The most powerful accelerator nowadays is the Large Hadron Collider (LHC), CERN. 
Delivering proton-proton collisions with a center-of-mass energy of 8 TeV in 2012, 
it became a real top quark factory. The large number of events with top quark, 
which were produced at the LHC, gives us a unique possibility to study precisely the properties of the heaviest quark. 

The top quarks are dominantly produced in pairs, decaying before they could hadronize. 
Each top quark of the pair decays to a $W$ boson and a $b$-quark almost exclusively. 
The W boson has several decay channels. In this work only the ``dilepton" channel, 
where both $W$ bosons from the two top quarks decay to a lepton and an (anti)neutrino, is studied. 
For this purpose the $19 fb^{-1}$ data sample with 8 TeV center-of-mass energy taken in 2012 was analyzed.

This work represents the first measurement of the normalized double differential top pair production
cross section at the LHC. The double differential $t\bar{t}$ production cross sections provide a stringent test of the 
Standard Model of Particle Physics, as they allow to study the $t\bar{t}$ production dynamics in unprecedented
detail. For the measurement of the double differential production cross sections a new kinematic reconstruction
of the $t\bar{t}$ events was implemented, providing an accurate and unbiased determination of the momenta of the
top and the anti top quarks.

This measurement is done in totally XXX different variables
connected to the top kinematics. A comparison to theory predictions is also performed.

In order to contribute to the future measurements with higher statistics during
the next years runs of the LHC, a part of my work was connected to the studies of irradiated
prototype silicon sensors and readout chips for the Phase-I upgrade of the CMS pixel detector.

This work is structured the following way. The chapter \ref{chapt:SM} gives an introduction to the theoretical
knowledge behind the measurements performed in this analysis. It gives a brief overview of the Standard
Model of particle physics and top quark physics in particular.