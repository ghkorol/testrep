\chapter{Introduction}

Nowadays the heaviest known elementary particle is the top quark which is as heavy as an atom of gold. 
The search for this particle lasted more then two decades and ended successfully 
in March 1995 at the Fermi National Accelerator Laboratory, Fermilab, where 
the discovery of the top quark was announced \cite{PhysRevLett.74.2626}. 

The third generation of quarks, which consists of top and bottom quarks, 
was predicted in 1973 by Kobayashi and Maskawa \cite{Kobayashi:1973fv}. However, the huge mass of the 
top quark was not expected, so it couldn't be discovered for a long time, while it's partner, 
the bottom quark, was experimentally found already four years after its theoretical prediction \cite{PhysRevLett.39.252}. 

To produce a top quark one needs to concentrate an immense amount of energy 
into a small region of space. This is done at the accelerator experiments. 
The most powerful accelerator nowadays is the Large Hadron Collider (LHC), CERN. 
Delivering proton-proton collisions with a center-of-mass energy of 7 TeV in 2011 and 8 TeV in 2012, 
it became a real top quark factory. The large number of events with top quarks
produced at the LHC gives us a unique possibility to study precisely the properties of the heaviest quark. 

The top quarks are dominantly produced together with an antitop quark, which is called in the 
following $t\bar{t}$ production. Top quarks decay before they could hadronize. 
Each top quark of the pair decays to a $W$ boson and a $b$-quark almost exclusively. 
The W boson has several decay channels. In this work only the ``dilepton" channel, 
where both $W$ bosons from the two top quarks decay to an (anti)lepton and an (anti)neutrino, is studied. 
For this purpose the 19 fb$^{-1}$ data sample with 8 TeV center-of-mass energy taken in 2012 was analyzed.

This work represents the first measurement of the normalized double differential (2D) top pair production
cross section at the LHC. The double differential $t\bar{t}$ production cross sections provide a stringent test of the 
Standard Model of Particle Physics, as they allow to study the $t\bar{t}$ production dynamics in unprecedented
detail. For the measurement of the double differential production cross sections a new kinematic reconstruction
of the $t\bar{t}$ events was implemented, providing an accurate and unbiased determination of the momenta of the
top and the antitop-quarks.

This reconstruction was used for single differential cross section measurements \cite{Khachatryan:2015oqa} which were 
published recently. In these measurements it was observed that the transverse momentum spectrum of the 
top quarks is softer than in several Standard Model predictions based on calculations up to Next-to-Leading Order (NLO) QCD. 
The study of 2D cross sections allows to further investigate the origin of this discrepancy. For this purpose the measurement 
is done in nine different 2D variable combinations connected to the top kinematics and results are compared to various model 
predictions. The object selection presented in this work closely follows the strategy of the single differential measurement.

In order to contribute to future measurements with higher statistics during
the next runs of the LHC, a part of my work was connected to the studies of irradiated
prototype silicon sensors and readout chips for the Phase-I upgrade of the CMS pixel detector.

This thesis is structured the following way. Chapter \ref{chapt:SM} provides an introduction to the theoretical
knowledge behind the measurements performed in this analysis. It gives a brief overview of the Standard
Model of particle physics and top quark physics in particular.

Chapter \ref{chap:exp_setup} describes the experimental setup used to obtain the data for this analysis.
It provides information about the LHC accelerator and the CMS detector, giving a brief introduction to each detector part.

Chapter \ref{chapt:pixel} elucidates the studies performed for the
upgrade of the barrel pixel part of the CMS detector. The upgraded detector is planned to be operated starting from 2016.

Simulations of the $pp$ collisions and of the detector response were used to study detector effects for the $t\bar{t}$ 
double differential cross sections measurements. Additionally, the simulated samples allow a comparison of the results of 
the measurements to the theoretical models implemented in the simulations. An overview of the simulation models and 
techniques exploited for the analysis described in this thesis is presented in chapter \ref{chapt:MC}.

The reconstruction of the collision events collected by the CMS detector is described
in chapter \ref{chapt:event_selection}. After selecting objects like leptons, jets and missing transverse energy, which 
fulfill the event selection criteria described in chapter \ref{chapt:event_sel}, the $t\bar{t}$ candidate has to be reconstructed out
of them. The procedure of the full kinematic reconstruction of the $t\bar{t}$ system in the dileptonic final state is introduced
in chapter \ref{chapt:kinReco}.

Having the fully reconstructed event kinematics, the double differential $t\bar{t}$ production cross sections are determined
as described in chapter \ref{chapt:xsec}. This section gives an overview of the data unfolding and explains the cross section evaluation. 
The determination of the measurement of systematic uncertainties (sources and values) are described in chapter \ref{chapt:syst}. 
Chapter \ref{chapt:results} delivers the cross section results and the comparison to the theory models and the discussion of 
the presented results.

%The discussion of the results of the analysis final conclusions are summarized in chapter \ref{chapt:conc}.
A summary of the thesis and an outlook are provided in chapter \ref{chapt:conc}.