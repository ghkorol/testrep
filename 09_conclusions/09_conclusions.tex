\chapter{Summary and Outlook}\label{chapt:conc}

\section{Summary}

The top-quark is the heaviest known elementary particle. It provides possibilities for unique tests of the Standard
Model due to its properties. The top-quark has a lifetime, which is shorter than the typical hadronization time.
It decays before it can hadronize. Thus, studies of the top-quark provide the knowledge of bare quark 
properties.

In this work, normalized double differential cross sections for the $t\bar{t}$ production in the electron-muon final state 
in proton-proton collisions at 8 TeV are presented.
The results were obtained after analyzing a data set collected with the CMS detector at the LHC in 2012, which corresponds to a 
luminosity of 19.7 fb$^{-1}$.

A kinematic reconstruction procedure was employed to reconstruct the full final state of the $t\bar{t}$ pair. It is based on six constraints:
top- and antitop-quark mass constraint, $W^{\pm}$ mass constraint and the assumption that all the MET in the event is arising only
from the neutrinos from the top decay. These assumptions allow to solve the system of kinematic equations analytically. The kinematic reconstruction
algorithm requires lepton, antilepton, $b$-jet, $\bar{b}$-jet momenta and MET as an input. To account for possible fluctuations due 
to detector effects, the energies and directions of leptons and $b$-jets were smeared 100 times. The output of the kinematic reconstruction
algorithm are neutrino and antineutrino momenta.

The kinematic reconstruction algorithm is performing with 93\% efficiency. The reconstructed kinematics of the top- and antitop-quarks
show, according to MC simulations, no biases and trends. This kinematic reconstruction algorithm was also employed to obtain the 
results in the recent publication related
to the measurement of the single differential $t\bar{t}$ production cross sections in the dilepton channel at $\sqrt{s} = $8 TeV
at CMS \cite{Khachatryan:2015oqa}.

To account for the migrations between different bins, a two-dimensional unfolding is performed. In this work the TUnfold 
algorithm\cite{Schmitt:2012kp} (based on $\chi^{2}$ minimization and Tikhonov regularization \cite{Tikhonov:1963}) is utilized. 
The minimum global correlation coefficient method\cite{VBlobelT} is exploited to choose the 
optimal regularization strength. The unfolding is performed simultaneously in bins of two variables in which the cross
sections are measured.

The central topic of this work is the first measurement of the normalized double differential $t\bar{t}$ production cross 
sections at $\sqrt{s}$ = 8 TeV. This measurement is important to test Standard Model predictions. Detailed knowledge of the 
top-pair production properties is also important for searches beyond the SM.

The normalized double differential $t\bar{t}$ production cross sections were measured in bins of nine different combinations of
variables, which describe the top or top-pair kinematics. The precision of these measurements allows a visual comparison to different 
Standard Model predictions as implemented in $\MG+\PYTHIA$, $\Powheg+\PYTHIA$, $\Powheg+\HERWIG$ and $\MCNLO+\HERWIG$ Monte Carlo
event generators. In general, the Standard Model predictions agree with the measured double differential
$t\bar{t}$ production cross sections.

The results show that the top-quark transverse momentum is generally softer in data than in the MC predictions for the central
top-quark rapidity bins. The best description of the top-quark transverse momentum spectra is provided by the $\Powheg+\HERWIG$
model.

For the higher invariant masses of the top-quark pairs, The cross sections in bins of $p_{T}(t)$ and $|y(t)|$ are generally poorly 
described by the models.

The top and antitop quarks in data have a larger separation in $\eta$ than in Monte Carlo predictions. This effect is more pronounced in 
$\MG+\PYTHIA$. 
% This tendency may point to the dominance of
% soft component in radiation in the Monte Carlo models.

The relevant sources of systematic uncertainties for the normalized cross section measurement were taken into account in this analysis.
The result showed that the measurement is dominated by statistical uncertainties.

This measurement confirmed the observations made previously in single differential $t\bar{t}$ production cross section measurement
at $\sqrt{s}$ = 8 TeV\cite{Asin2014Auth} in dileptonic final state. The same tendency to the softer $p_{T}$ of the top-quark in 
data is seen similarly to what has been presented in this work. However, the double differential studies reveal more: they show that
the softer $p_{T}(t)$ is related to the effect that at high $M(t\bar{t})$ the pseudorapidity separation $\Delta\eta(t\bar{t})$ is larger 
in data than in MC (which is also reflected by a broader $y(t)$ distribution in data than in MC). It seems that the discrepancy might
be related to the mixture of matrix element hard radiation and parton showering. It will be the tasks mainly for the 
theorists to relate this discrepancies to deficiencies in their models and to improve their predictions.

\section{Outlook}

The measurement of double differential $t\bar{t}$ production cross section, as presented in this work, is dominated by the statistical
uncertainty. There is a room for improvement with exploiting the upcoming data set, which is going to be collected at the high-energy LHC
run with the collision center-of-mass energy of $\sqrt{s}$ = 13-14 TeV. A more intense top-pair production is expected, which will ensure
a higher statistics in the data samples to be analyzed. A better precision gained in this way will allow to confirm the tendencies 
observed in this analysis and to perform the test of Standard Model predictions in more detail.

One could also measure the $t\bar{t}$ production cross section in a fiducial range at particle level. This can be done by taking out the 
acceptance correction in eq. \ref{eq:epsanal}. By doing so one avoids the kinematic extrapolation from fiducial to total cross section
range represented by $\mathcal{A}$. This reduces the dependence of the measurement on the MC which was used to calculate the acceptance.

On the theory side it will be very interesting to compare NNLO models to the double differential data, once they are available for the
double differential observables.

It would be also intersting to try including the measured cross sections
in a proton PDF fit. Here the presented measurements (sec. results) as
function
of the observable $\times 1$ might be particular interesting because
they provide (in the leading order QCD picture) access to the PDFs at
different momentum fractions.
With the presented tt data also large momentum fractions ($> 0.4$) are
being probed with at least some sensitivity and this
could help to constrain the gluon density in this region.